En este ultimó capitulo se plantean las conclusiones derivadas del trabajo
realizado en el proyecto de grado y se establecen aspectos que pueden ser
retomados como trabajo futuro.

\section{Conclusiones}

% Este proyecto de grado se planteo como objetivo mejorar la propuesta de
% coordinación para la exploración multirobot de \cite{wurm2008coordinated} en la
% que se establece que la exploración debe realizase maximizando la distribución
% de los robots sobre las habitaciones y corredores (segmentos), los cuales son
% identificados a partir de un GVD. Para esto se evaluaron varios aspectos de la
% propuesta en cuestión y del problema de exploración multirobot en general
% relevando una serie de potenciales mejoras. Estas fueron implementadas como
% parte de una solución al problema de exploración multirobot \footnote{Disponible en línea:\\
% \url{https://gitlab.fing.edu.uy/federico.ciuffardi/pgmappingcooperativo}} con la cual se
% experimento dentro de un simulador con el propósito de validar su impacto.

Este proyecto de grado se planteo como objetivo mejorar la propuesta de
exploración multirobot de \cite{wurm2008coordinated} donde se establece que la
exploración debe llevarse a cabo maximizando la distribución de los robots
sobre las habitaciones y corredores (segmentos), los cuales son identificados a
partir de un GVD. Para esto se evaluaron varios aspectos de la propuesta en
cuestión y del problema de exploración multirobot en general relevando una
serie de potenciales mejoras. Estas fueron implementadas como parte de una
solución al problema de exploración multirobot \footnote{Disponible en línea:\\
\url{https://gitlab.fing.edu.uy/federico.ciuffardi/pgmappingcooperativo}} con
la cual se experimento dentro de un simulador con el propósito de validar su
impacto.

En general, a partir de los resultados obtenidos de la experimentación se
concluye que la solución desarrollada logra resolver el problema de exploración
multirobot de forma satisfactoria en términos de los mapas generados.

Se puede concluir que es posible realizar de forma incremental, tanto la
construcción del GVD, como la obtención de información secundaria relacionada a
este que es necesaria para identificar los segmentos. Esto se logra adaptando
el algoritmo presentado en \cite{Lau2013} para que se incluya dicha
información. Adicionalmente los datos experimentales permiten concluir que la
construcción incremental del GVD acelera significativamente los tiempos de
construcción del GVD y de exploración frente a realizar una construcción no
incremental como en la propuesta original. Esto es una mejora considerable en
tanto hace viable aplicar la estrategia de coordinación en la exploración de
entornos de mayor tamaño o que requieran mapas con un mayor nivel de detalle.

También se concluye que el espacio desconocido puede ser considerado de
distintas maneras en la construcción del GVD. En especial que puede considerase de
forma de limitar la construcción del GVD al espacio conocido, lo cual lleva a
una reducción de los tiempos de construcción del GVD y de exploración, con
respecto a realizar la construcción en todo el espacio, como es usual.

% tambien se concluye que al construir el gvd el espacio desconocido puede
% considerase de forma de limitar la construcción al espacio conocido, lo cual
% lleva a una reduccion de los tiempos de construcción del gvd y de
% exploración, con respecto a realizar la construcción en todo el espacio como
% es usual.

En relación a la identificación de objetivos, según los resultados
experimentales se concluye que utilizar todas las fronteras como objetivos
tiene un impacto negativo considerable en los tiempos de exploración y en menor
medida en la distancia que recorren los robots, frente a reducir el numero de
objetivos simplificando las fronteras en sus partes mas significativas.
Adicionalmente se mencionan dos métodos de simplificación de fronteras la
simplificación basada en K-Mean que se encuentra presente en el estado del arte y
la simplificación basada en cubrimiento propuesta en este proyecto como mejora
de la anterior. Los resultados experimentales permiten concluir que la
simplificación basada en cubrimiento logra una mejor reducción de objetivos que la
simplificación basada en K-Means, pero tiene un rendimiento inferior en
términos de costo computacional. 

Adicionalmente existen dos potenciales mejoras sobre las cuales únicamente se
puede concluir que funcionan correctamente por formar parte de la solución
realizada, pero que su impacto no fue aislado y estudiado en la
experimentación. La primera es el modelo de planificación jerárquica que
combina la planificación sobre el GVD y sobre todo el espacio libre, que
debería implicar reducciones en las distancias recorridas frente a navegar
únicamente sobre el GVD y de costos computacionales frente a calcular los
caminos directamente sobre todo el espacio libre. La segunda es el algoritmo de
asignación de objetivos propuesto como alterativa al método húngaro como es
utilizado en el trabajo original de \cite{wurm2008coordinated}. La ventaja del
algoritmo propuesto es que este logra distribuir los robots sobre los segmentos
considerando el numero de objetivos que existen en estos, lo cual evita dejar
robots ociosos por asignar mas robots a un segmento que los objetivos que se
encuentran en él.


% La conclusion principal es que el objetivo se cumplio de forma satisfactoria,
% varias mejoras relacionadas a los aspectos de asignación de tareas, construcción
% del GVD y planificación, fueron presentadas y validadas con la solución
% dearrollada. 

% Las mejoras estan asociadas construcción del GVD y identificación de objetivos.

% La principal mejora consiste en hacer uso de un algoritmo incremental para
% contruccion del GVD en lugar de uno no incremental como en la propuesta
% original.

% el desarrollo una solucion al problema de
% exploracion multirobot basada en \cite{wurm2008coordinated} 

% Uno de
% los aspectos principales de la exploración multirobot 

% En este proyecto se trabajo sobre el problema de exploración multirobot. 

% En este proyecto se describio y 

% Varios aspectos de la exploración
% multirobot fueron analizados derivando en propuestas 

% La solucion desarrollada a

% \todo[inline]{seguir con las dos mejoras, ir mezclando las conclusiones de las pruebas capaz.}

% El trabajo desarrollado trata sobre el problema de exploración utilizando un
% grupo de robots, más específicamente sobre como lograr distribución eficiente
% de los robots en el espacio para una mayor coordinación en la exploración. Para
% esto, en lugar de distribuir los robots sobre las fronteras entre el espacio
% conocido y el desconocido, como es usual, la propuesta apunta a identificar
% porciones del espacio similares a habitaciones o corredores, llamados
% segmentos, para luego distribuir los robots primero sobre dichos segmentos y
% luego sobre las fronteras de los mismos.

% punteo
% * La exploración es un problema complejo:
%   * involucr varios subprobleaas de la robtica que deben interactuar entre si para lograr
% * La

\section{Trabajo a futuro}

Esta sección se dedica a comentar las lineas de trabajo que surgen del proyecto
realizado y pueden ser retomadas a futuro.

\subsection{Experimentacion en la realidad}

Una de estas líneas de trabajo es la de realizar pruebas en la realidad con el
proposito de validar los resultados obtenidos de los experimentos simulados.
Esto no es una tarea trivial ya que, ademas de requerir de hardware y de un
entorno en donde hacer las pruebas, también implica remover las hipotesis de
comunicación ideal y de localización trivial de los robots, ya que estas
hipotesis no son viables en la realidad. Esto hace necesario alterar la
asignación de tareas de forma que los problemas potenciales de comunicación
sean considerados y evitados durante la exploración. Adicionalmente requiere
integrar una solución de localización que permita obtener la ubicación de los
robots en un escenario real.

\subsection{Identificación de objetivos}
% esta relacionada al
% metodo de identificación de objetivos que funciona simplificacando fronteras
% basandose en el cubrimiento. Como se comento en la seccion de conclusiones,
% dicho metodo obtiene una reduccion de objetivos mejo

% ue se abre desde este proyecto

% Se intuye que existe la

% posiblidad de mejorar el rendimiento computacional del 

También se abren dos líneas de trabajo que profundizan en el estudio y
desarrollo de metodos de identificación de objetivos. La primera esta orientada
al anlaisis del metodo de simplificación de fronteras basado en el cubrimiento
en busca de posibles optimizaciones que logren mejorar su rendimiento
computacional, que actualmente es su mayor desventaja. La segunda consiste en
el desarrollo de variantes del metodo antes mencionado que en lugar de elegir
las fronteras sigificativas de forma abitraria desde los conjuntos de fronteras
candidatas lo hagan con algun criterio especifico, como por ejemplo elegir la
frontera candidata con mayor ganancia de información \cite{amorin2019novel}.
\todo{de tener una ref al metodo de affinity propagation lo agrego como trabajo futuro}

% La tercera es
% el estudio de otros metodos identificación de objetivos, como por ejemplo la simplificación.

De seguir alguna de estas direcciones también sería conveniente diseñar nuevas
pruebas que se concentren en evaluar la identificación de objetivos, ya que por
ejemplo en el entorno utlizado la mayoria de habitaciones son pequeñas y no son
propensas a generar fronteras con una complejidad suficiente como para
presentar un desafio para los metodos de simplificación estudiados.

 % analizar la posibilidad de optimizar 

% Como se comento en la seccion de conclusiones,
% dicho metodo obtiene una reduccion de objetivos mejo

% Comentar que el entorno en el que se realizan las pruebas pone a prueba los
% algorimos de simplificacion principalmente en la habitacion grande del centro,
% siendo las demas demasiado pequeñas como para favorecer que se generen
% fronteras con las condiciones de ser largas y serperteantes, de forma de
% generar las condiciones necesaria para que la simplificacion de K-Means
% funcione peor. Quizas el generar un conjunto distinto de pruebas viene bien
% junto a hacerlo mas eficiente y tambien compara contra affinity propagation

% Cuidado con la simplificacion de fronteras y los obstaculos que pueden llegar a
% ocluir las fornteras (modificar la def de cubrimiento para que no permita
% oclucsiones)



\subsection{Mapa topológico}

En la solucion desarrollada la extension espacial de los segmentos es la unica
información necesaria y que se tiene disponible relacionada al mapa topologico.
Por lo que una direccion de trabajo posible seria la de procesar esta
información para obtener un mapa topológico completo. Por ejemplo un grafo que
contenga un vertice por segmento y las conexines entre dichos vertices implique
puertas entre los segmentos.

Evaluar los mapas topológicos generados por la solucion desarrollada con las
ideas presentadas en la sección \ref{sec:evalTop} también se presenta como una
linea de trabajo interesante. Al igual que el estudio, implementación y
validación de tecnicas que proponen mejorar la segmentación como las que se
describen en \cite{Thrun1998,Liu2015,wurm2008coordinated}.

% * Mejorar la segmentacion, por ejemplo con:
%   * Trun y su mergeo de segmentos 
%   * lo descrito en liu ming y siegwart spectral clustering

% * Evaluar la segmentacion incremental con lo mencionado en la seccion de criterios del "estado del arte"

% * Hacer la tencica de filtrado de puntos criticos de wurm (la de vecino de grado 3) de forma incremental


% tiene consite de sus segmentos sueltas y el gvd que
%     permite navegar entre estas.
%     * no es necesaria otra para lo que hice pero estaria bueno como herramienta
%       generar un verdadero mapa topologico

% Auque en la solucion del proyecto desarrollado se identifican los segmentos del entorno

% es la de probar distintas formas de sementacion, estudiar el impacto de cada 
% * Consolidar una la generacion un mapa topologico verdadero:
%   * ahora el mapa topologico se compone de regiones sueltas y el gvd que
%     permite navegar entre estas.
%     * no es necesaria otra para lo que hice pero estaria bueno como herramienta
%       generar un verdadero mapa topologico
%   * La idae seria reducir los segmentos vertices ubicandos en sus centros de
%     masa y las conexiones entre ellos (a travez del gvd/mapa metrico) a aristas
%     punto a punto entre dichos vertices.


\subsection{Asignación de objetivos y Planificación}

Realizar pruebas que permitan comparar el impacto del algorimo de asignación de
objetivos introducido contra con otros algorimos de asignación de objetivos,
como por ejemplo el metodo húngaro, es otra posible direccion de trabajo que
puede ser retomada a futuro.

De forma similar otro trabajo futuro posible es el de validar el impacto de la
planificación jerarquica presentada en este proyecto comparandola con otras
estrategias alterativas de planificación. 

\subsection{Potenciales mejoras de rendimiento}

% varios trabajos a futuro que se .. .de esto

Existen varias mejoras potenciales de rendimiento que pueden ser estudidadas,
implementadas y validadas como trabajo fururo. A continuacion se presentan
algunos ejemplos.

Un ejemplo sería el paralelizar la contrucción del GVD y la identificación de
objetivos en la etapa de obtención de información.

Otro ejemplo consiste cambiar algoritmo de segmentación no incremental por una variante
incremental de este, como la presentada en \cite{Liu2015}. 

Los algorimos de simplificación de frontera son no incrementales y cambiarlos
por variantes incrementales tambien seria un ejemplo. Aunque en este caso no se
encontro información al respecto por lo que se deberia estudiar la posibilidad
de desarrollar las variantes incrementales.

% Ver posibilidades de un filtrado de fronteras (obtener fronteras sigificativas) de forma incremental


\subsection{Reusabilidad}

Otro trabajo futuro posible es el de refcatorizar, organizar y documentar cada
modulo de la solucion ya disponible en linea, con el proposito de proporcionar
dichos modulos como nodos de ROS independientes, siguiendo los estandares
propuestos por la comunidad de dicha herramienta. De esta forma se facilita la
reutilizacion de lo desarrollado en este proyecto con lo que se lograra
alcanzar a mas investigadores y estudantes.

Relacionado a la anterior adaptar la solución actual que utliza la primera
verision de ROS (ROS 1) a su segunda versión (ROS 2), puede ser otra linea de
trabajo de interes pensando en la reusabilidad a fururo esto ya que ROS 1
termina su soporte en el 2025.

% * Hacer librerias con los timpos de datos utliados y algormios desarrollados (agnositcos de ros 1)
%   * Esto se hizo (no seria trabajo futuro)

% * Hacer paquetes de ros 1 que sigan los estandares, sean modulares y resutilizables 
%   * Ahora esta todo junto dentro de un paquete pero los modulos son
%     relativametne independientes entre si y podrian modularizarse para su
%     reutlizacion y combinacion con otros paquetes

% * Hacer lo mismo del punto anterior con ros2:
%   * Discutir que esto es lo mas razonable ya que:
%     * ROS 1 termna su soporte pronto
%     * ROS 2 es ampliamente superior cuando se trata de multirobot (no es centralizado)

% \subsection{Remocion de hipotesis}

% X Se podria trabajar para remover la hipotesis de comunicacion ideal, esto introduce el desafio de considerar los kkkkk

% X comunicacion ahora es ideal, si quisieramos simular mejores comuniciaiones nosotros deberiamos investigar co-simulacion (simulador de fisicas gazebo y simulador dee redes).
%     * Ver de comentar cocosim

% X ubiacion ahora es ideal podria ver que desafios introduce el uso la remocion de esta restriccion


% * Hipotesis de saber las dimenciones del mapa estaria buena removerla , dar ideas

% * Hay que comentar homogeneidad como restriccion que estamos tomando:
%   * Ahora las fronteras accesibles en la grilla son accesibles por todos los robots 
%     * Por ser homogeneos y cierculares los robots se puede engordar los obstaculos por el radio del robot y eso asegura lo del punto antesrior.
%   * De lo contrario si los tamanio 
%   * tipos de fronteras dependiendo de la accesibilidad : de percepcion y de movilidad
%     * se podria asignar dimendsiones minimas de lso robots a las fronteras (para un camino tengo los vertices y cada uno tiene distancia minima a los obstaculos entoces la minima distancia a un obstaculo es tu dimencion maxima si no te vas a chochar). 


%   * Por ahora los entornos se considerarn estaticos, y en mayor parte lo son (solo los demas robots son dinamicos). Levantar esta restriccion tambien es un trabajo a futuro interesante. Ver de modelar puertas personas , etc y como manejarlas. Puertas como cosas que se pueden abrir. Personas como cosas a esquivar pero que no queneran segmentos etc.


