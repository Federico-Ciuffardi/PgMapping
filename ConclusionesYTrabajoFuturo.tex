En este proyecto de grado se planteo el objetivo de mejorar la solución del
problema de exploración multi-robot propuesta en \cite{wurm2008coordinated},
donde se propone una estrategia de coordinación para  entornos estructurados
que consiste en explorar maximizando la distribución de los robots sobre los
segmentos (habitaciones y corredores), los cuales se identifican a partir de un
GVD. Para esto se evaluaron varios aspectos de la propuesta en cuestión y del
problema de exploración multi-robot en general relevando una serie de
potenciales mejoras. Estas fueron implementadas como parte de una solución al
problema de exploración multi-robot\footnote{Disponible en línea:\\
\url{https://gitlab.fing.edu.uy/federico.ciuffardi/pgmappingcooperativo}} con
la cual se experimento dentro de un simulador con el propósito de validar su
impacto.

En este ultimó capitulo se plantean las conclusiones derivadas del trabajo
realizado en el proyecto de grado y se establecen líneas de trabajo que pueden ser
retomadas en el futuro.

\section{Conclusiones}

% Este proyecto de grado se planteo como objetivo mejorar la propuesta de
% coordinación para la exploración multi-robot de \cite{wurm2008coordinated} en la
% que se establece que la exploración debe realizase maximizando la distribución
% de los robots sobre las habitaciones y corredores (segmentos), los cuales son
% identificados a partir de un GVD. Para esto se evaluaron varios aspectos de la
% propuesta en cuestión y del problema de exploración multi-robot en general
% relevando una serie de potenciales mejoras. Estas fueron implementadas como
% parte de una solución al problema de exploración multi-robot \footnote{Disponible en línea:\\
% \url{https://gitlab.fing.edu.uy/federico.ciuffardi/pgmappingcooperativo}} con la cual se
% experimento dentro de un simulador con el propósito de validar su impacto.

En general, a partir de los resultados obtenidos en la experimentación, se
concluye que la solución desarrollada resuelve el problema de exploración
multi-robot de forma satisfactoria en términos de los mapas generados. 

Por otro lado se puede concluir que es posible realizar la construcción del GVD
de forma incremental sin perder la información secundaria necesaria para
identificar los segmentos. Es decir, que es posible aplicar la estrategia de
coordinación de \cite{wurm2008coordinated} construyendo el GVD de forma
incremental. Esto se logró adaptando el algoritmo presentado en \cite{Lau2013}
para incluir la información necesaria. A su vez los datos experimentales permiten
concluir que la construcción incremental del GVD acelera significativamente los
tiempos de construcción del GVD y de exploración, frente a realizar una
construcción no incremental como en la propuesta original. Esto es una mejora
considerable en tanto hace viable aplicar la estrategia de coordinación en
entornos de mayor tamaño o que requieran mapas con un mayor nivel de detalle.

% lagf exploración de

También se concluye que el espacio desconocido puede ser considerado de
distintas maneras en la construcción del GVD. En especial que puede considerase
de forma de limitar la construcción del GVD al espacio conocido, lo cual lleva
a una reducción de los tiempos de construcción del GVD y de exploración, con
respecto a realizar la construcción en todo el espacio (conocido y desconocido)
como es usual.

% También se concluye que el espacio desconocido puede ser considerado de
% distintas maneras en la construcción del GVD. En especial que puede considerase de
% forma de limitar la construcción del GVD al espacio conocido, lo cual lleva a
% una reducción de los tiempos de construcción del GVD y de exploración, con
% respecto a realizar la construcción en todo el espacio, como es usual.


% tambien se concluye que al construir el gvd el espacio desconocido puede
% considerase de forma de limitar la construcción al espacio conocido, lo cual
% lleva a una reduccion de los tiempos de construcción del gvd y de
% exploración, con respecto a realizar la construcción en todo el espacio como
% es usual.

En relación a la identificación de objetivos, según los resultados
experimentales se concluye que utilizar todas las fronteras como objetivos
tiene un impacto negativo considerable en los tiempos de exploración y en menor
medida en la distancia que recorren los robots, frente a reducir el número de
objetivos simplificando las fronteras a sus partes más significativas.
Respecto a la simplificación de fronteras se mencionan dos métodos, la
simplificación basada en K-Means que se encuentra presente en el estado del
arte, y la simplificación basada en cubrimiento que es propuesta en este proyecto como
mejora de la anterior. Los resultados experimentales permiten concluir que la
simplificación basada en cubrimiento logra una mayor reducción de objetivos que
la simplificación basada en K-Means, pero lleva más tiempo.
% pero tiene un rendimiento inferior en términos de costo computacional.

Adicionalmente existen dos mejoras potenciales sobre las cuales únicamente se
puede concluir que funcionan correctamente por formar parte de la solución
implementada, pero que su impacto no fue aislado y estudiado experimentalmente.
La primera es el modelo de planificación que combina jerárquicamente la
planificación sobre el GVD, y la planificación sobre todo el espacio libre.
Esta forma de planificar debería implicar reducciones en las distancias
recorridas frente a navegar únicamente sobre el GVD y de costos computacionales
frente a calcular los caminos directamente sobre todo el espacio libre. La
segunda es el algoritmo de asignación de objetivos que fue propuesto como
alterativa al utilizado en el trabajo original de \cite{wurm2008coordinated}.
La principal ventaja de este algoritmo es que logra maximizar la distribución
de los robots sobre los segmentos evitando asignar más robots a un segmento que
los objetivos contenidos en él.

% como si puede
% ocurrir en la propuesta original por asignarse más robots a un segmento que los
% objetivos que se encuentran en él.


% La conclusion principal es que el objetivo se cumplio de forma satisfactoria,
% varias mejoras relacionadas a los aspectos de asignación de tareas, construcción
% del GVD y planificación, fueron presentadas y validadas con la solución
% dearrollada. 

% Las mejoras estan asociadas construcción del GVD y identificación de objetivos.

% La principal mejora consiste en hacer uso de un algoritmo incremental para
% contruccion del GVD en lugar de uno no incremental como en la propuesta
% original.

% el desarrollo una solucion al problema de
% exploración multi-robot basada en \cite{wurm2008coordinated} 

% Uno de
% los aspectos principales de la exploración multi-robot 

% En este proyecto se trabajo sobre el problema de exploración multi-robot. 

% En este proyecto se describio y 

% Varios aspectos de la exploración
% multi-robot fueron analizados derivando en propuestas 

% La solucion desarrollada a

% \todo[inline]{seguir con las dos mejoras, ir mezclando las conclusiones de las pruebas capaz.}

% El trabajo desarrollado trata sobre el problema de exploración utilizando un
% grupo de robots, más específicamente sobre como lograr distribución eficiente
% de los robots en el espacio para una mayor coordinación en la exploración. Para
% esto, en lugar de distribuir los robots sobre las fronteras entre el espacio
% conocido y el desconocido, como es usual, la propuesta apunta a identificar
% porciones del espacio similares a habitaciones o corredores, llamados
% segmentos, para luego distribuir los robots primero sobre dichos segmentos y
% luego sobre las fronteras de los mismos.

% punteo
% * La exploración es un problema complejo:
%   * involucr varios subprobleaas de la robtica que deben interactuar entre si para lograr
% * La

\section{Trabajo a futuro}

Esta sección se dedica a comentar las líneas de trabajo que surgen del proyecto
realizado y pueden ser retomadas a futuro.

\subsection{Experimentación en la realidad}

Una de estas líneas de trabajo es la de realizar experimentos en la realidad
con el propósito de validar los resultados obtenidos en los experimentos
simulados. Esto no es una tarea trivial ya que, además de requerirse hardware y
un entorno en donde hacer las pruebas, también se deben remover las hipótesis de
comunicación ideal y de localización trivial de los robots, dado que estas
no son viables en la realidad. Esto hace necesario alterar la
implementación de forma que los potenciales problemas de comunicación
sean considerados y evitados durante la exploración \cite{amigoni2017multirobot}. Y por otro lado requiere
integrar una solución de localización que permita estimar la ubicación de los
robots \cite{slam}. % en un escenario real.

\subsection{Identificación de objetivos}
% esta relacionada al
% método de identificación de objetivos que funciona simplificando fronteras
% basándose en el cubrimiento. Como se comento en la secón de conclusiones,
% dicho método obtiene una reducción de objetivos mejo

% posibilidad de mejorar el rendimiento computacional del 

También se abren dos líneas de trabajo que profundizan en el estudio y
desarrollo de métodos de identificación de objetivos. La primera está orientada
al análisis del método de simplificación de fronteras basado en el cubrimiento
en busca de posibles optimizaciones que logren mejorar su rendimiento
computacional, que actualmente es su mayor desventaja. La segunda consiste en
el desarrollo de variantes del método antes mencionado, que en lugar de
determinar de forma arbitraria cual frontera es la significativa del conjunto
de candidatas, se determine con algún criterio específico, como por ejemplo elegir la
frontera candidata con mayor ganancia de información \cite{Amorin2019}.

% La tercera es
% el estudio de otros métodos identificación de objetivos, como por ejemplo la simplificación.

De seguir alguna de estas direcciones también sería conveniente diseñar nuevas
pruebas que se concentren en evaluar la identificación de objetivos, ya que por
ejemplo en el entorno utilizado la mayoría de habitaciones son pequeñas y no son
propensas a generar fronteras con una complejidad suficiente como para
presentar un desafió a los métodos de simplificación estudiados.

 % analizar la posibilidad de optimizar 

% Como se comento en la sección de conclusiones,
% dicho método obtiene una reducción de objetivos mejo

% Comentar que el entorno en el que se realizan las pruebas pone a prueba los
% algoritmos de simplificación principalmente en la habitación grande del centro,
% siendo las demás demasiado pequeñas como para favorecer que se generen
% fronteras con las condiciones de ser largas y serpentean tes, de forma de
% generar las condiciones necesaria para que la simplificación de K-Means
% funcione peor. Quizás el generar un conjunto distinto de pruebas viene bien
% junto a hacerlo más eficiente y también compara contra affinity propagation

% Cuidado con la simplificación de fronteras y los obstáculos que pueden llegar a
% ocluir las fronteras (modificar la def de cubrimiento para que no permita
% oclusiones)



\subsection{Mapa topológico}

En la solución desarrollada la extensión de los segmentos es la única
información necesaria y que se tiene disponible relacionada al mapa topológico.
Por lo que una dirección de trabajo posible sería la de desarrollar un método
para procesar dicha información con el propósito obtener un mapa topológico
completo, como por ejemplo un grafo que contenga un vértice por segmento y las
aristas entre dichos vertices implique la adyacencia entre los segmentos.

Evaluar los mapas topológicos generados a partir de la solución desarrollada
con las ideas presentadas en la sección \ref{sec:evalTop} también se presenta
como una línea de trabajo interesante. Al igual que el estudio, implementación
y validación de técnicas que proponen mejorar la segmentación como las que se
describen en \cite{Thrun1998,Liu2015,wurm2008coordinated}.

% * Mejorar la segmentación, por ejemplo con:
%   * Trun y su mergo de segmentos 
%   * lo descrito en lie Ming y siege wart spectral clustering

% * Evaluar la segmentación incremental con lo mencionado en la sección de criterios del "estado del arte"

% * Hacer la técnica de filtrado de puntos críticos de warm (la de vecino de grado 3) de forma incremental


% tiene consiste de sus segmentos sueltas y el GVD que
%     permite navegar entre estas.
%     * no es necesaria otra para lo que hice pero estaría bueno como herramienta
%       generar un verdadero mapa topológico

% Aoque en la solución del proyecto desarrollado se identifican los segmentos del entorno

% es la de probar distintas formas de segmentación, estudiar el impacto de cada 
% * Consolidar una la generación un mapa topológico verdadero:
%   * ahora el mapa topológico se compone de regiones sueltas y el GVD que
%     permite navegar entre estas.
%     * no es necesaria otra para lo que hice pero estaría bueno como herramienta
%       generar un verdadero mapa topológico
%   * La idea sería reducir los segmentos vertices ubicándoos en sus centros de
%     masa y las conexiones entre ellos (a través del GVD/mapa métrico) a aristas
%     punto a punto entre dichos vertices.


\subsection[Algoritmo de asignación de objetivos y Planificación]{Algoritmo de asignación de objetivos y\\ Planificación}

Otra posible dirección de trabajo que puede ser retomada a futuro es la de
realizar pruebas que permitan comprobar el impacto del algoritmo de asignación
de objetivos introducido en este proyecto, comparándolo contra con otros algoritmos de
asignación de objetivos, como por ejemplo el método húngaro como es utilizado
en \cite{wurm2008coordinated}.

De forma similar otro trabajo futuro posible es el de validar el impacto de la
planificación jerárquica presentada en este proyecto comparándola con otras
estrategias alterativas de planificación. 

\subsection{Potenciales mejoras de rendimiento}

% varios trabajos a futuro que se .. .de esto

Existen varias mejoras potenciales de rendimiento que pueden ser estudiadas,
implementadas y validadas como trabajo futuro. A continuación se presentan
algunos ejemplos.

Un ejemplo sería el paralelizar la segmentación y la identificación de
objetivos en la etapa de obtención de información.

Otro ejemplo consiste en cambiar el algoritmo de segmentación no incremental por una variante
incremental de este, como la presentada en \cite{Liu2015}. 

Los algoritmos de simplificación de frontera son no incrementales y cambiarlos
por variantes incrementales también es un un ejemplo. Aunque en este caso no se
encontró información al respecto por lo que se debería estudiar si esto es
posible.

% Ver posibilidades de un filtrado de fronteras (obtener fronteras significativas) de forma incremental


\subsection{Reusabilidad}

Otro trabajo futuro posible es el de refactorizar, organizar y documentar cada
módulo de la solución desarrollada, con el propósito de proporcionar dichos
módulos como nodos de \gls{ROS} independientes, que sigan los estándares propuestos
por la comunidad de dicha herramienta. De esta forma se facilita la
reutilización de lo desarrollado en este proyecto permitiendo
alcanzar a más investigadores y estudiantes.

Actualizar la solución actual que utiliza la primera versión de \gls{ROS} (\gls{ROS} 1)
para utilizar en su lugar a su segunda versión (\gls{ROS} 2), puede ser otra línea de
trabajo de interés pensando en la reusabilidad a futuro, ya que \gls{ROS} 1 termina
su soporte en el 2025.

% * Hacer librerias con los timpos de datos utliados y algormios desarrollados (agnositcos de \gls{ROS} 1)
%   * Esto se hizo (no sería trabajo futuro)

% * Hacer paquetes de \gls{ROS} 1 que sigan los estandares, sean modulares y resutilizables 
%   * Ahora esta todo junto dentro de un paquete pero los modulos son
%     relativametne independientes entre si y podrian modularizarse para su
%     reutlizacion y combinacion con otros paquetes


% * Hacer lo mismo del punto anterior con ros2:
%   * Discutir que esto es lo más razonable ya que:
%     * \gls{ROS} 1 termna su soporte pronto
%     * \gls{ROS} 2 es ampliamente superior cuando se trata de multi-robot (no es centralizado)

% \subsection{Remocion de hipotesis}

% X Se podria trabajar para remover la hipotesis de comunicacion ideal, esto introduce el desafio de considerar los kkkkk

% X comunicacion ahora es ideal, si quisieramos simular mejores comuniciaiones nosotros deberiamos investigar co-simulacion (simulador de fisicas gazebo y simulador dee redes).
%     * Ver de comentar cocosim

% X ubiacion ahora es ideal podria ver que desafios introduce el uso la remocion de esta restriccion


% * Hipotesis de saber las dimenciones del mapa estaria buena removerla , dar ideas

% * Hay que comentar homogeneidad como restriccion que estamos tomando:
%   * Ahora las fronteras accesibles en la grilla son accesibles por todos los robots 
%     * Por ser homogeneos y cierculares los robots se puede engordar los obstaculos por el radio del robot y eso asegura lo del punto antesrior.
%   * De lo contrario si los tamanio 
%   * tipos de fronteras dependiendo de la accesibilidad : de percepcion y de movilidad
%     * se podria asignar dimendsiones mínimas de lso robots a las fronteras (para un camino tengo los vertices y cada uno tiene distancia mínima a los obstaculos entoces la mínima distancia a un obstaculo es tu dimencion máxima si no te vas a chochar). 


%   * Por ahora los entornos se considerarn estaticos, y en mayor parte lo son (solo los demas robots son dinamicos). Levantar esta restriccion tambien es un trabajo a futuro interesante. Ver de modelar puertas personas , etc y como manejarlas. Puertas como cosas que se pueden abrir. Personas como cosas a esquivar pero que no queneran segmentos etc.


