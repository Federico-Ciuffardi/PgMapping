La exploración es un problema clásico de la robótica móvil autónoma que
consiste en utilizar un robot para obtener información de un entorno
desconocido, a través de sus sensores, con el objetivo de generar un mapa que lo
represente. La exploración es una parte fundamental en tareas de limpieza
\cite{luo2002real}, operaciones de búsqueda y rescate \cite{Liu2015}, misiones
extra planetarias \cite{schuster2019towards}, entre otras tareas en las cuales
se desconozca el entorno y sea ineficiente, o directamente inviable teleoperar
a un robot.

La exploración suele ser una tarea altamente paralelizable, al ser usual que en
un mismo instante de tiempo existan varios lugares del entorno de los cuales un
robot puede extraer información novedosa. Esto causa que sea atractivo el uso
de varios robots para acelerar la exploración, y en este caso el problema pasa
a conocerse como el de exploración multi-robot.

Sin embargo, al tener varios robots dedicados a la tarea de exploración es
necesario algún mecanismo de coordinación que logre explotar el paralelismo.
Para esto se debe evitar situaciones subóptimas como que varios robots realicen
trabajo redundante al explorar los mismos lugares, %desaprovechando sus capacidades sensoriales
o que estos interfieran entre sí causando perdidas de tiempo en desvíos para
evitar colisiones.

Los entornos estructurados como edificios, hogares y otras construcciones
humanas son entornos que pueden dividirse en segmentos, como habitaciones
y corredores. En \cite{wurm2008coordinated} se introduce una estrategia de
coordinación para la exploración multi-robot en entornos estructurados que
consiste en llevar a cabo la exploración maximizando la distribución de los
robots sobre los segmentos. La estrategia se fundamenta en que las
exploraciones de segmentos diferentes suelen ser independientes entre sí, por
lo cual asignar robots a diferentes segmentos ayuda a evitar redundancia en la
exploración. Como también en que los segmentos pueden ser demasiado pequeños
para que un segundo robot acelere su exploración. Y adicionalmente en que de
haber más de un robot explorando un mismo segmento estos pueden bloquearse
entre sí mientras intentan abandonarlo a la vez al finalizar su exploración.

El proyecto de grado se propone mejorar la propuesta de
\cite{wurm2008coordinated}, para esto se evaluaron varios aspectos de la
propuesta en cuestión y del problema de exploración multi-robot en general,
relevando una serie de potenciales mejoras que resultan en las seis
contribuciones que se describen a continuación.

\section{Contribuciones}\label{sec:cont}

La estrategia de coordinación requiere identificar los segmentos del entorno,
lo cual se conoce como segmentación. Uno de los métodos más populares y
eficientes para segmentar, que es también utilizado en la propuesta original, se
basa en utilizar una estructura llamada Diagrama Generalizado de Voronoi (GVD
por sus siglas en inglés). Dado que para aplicar la estrategia de coordinación
los segmentos deben mantenerse actualizados durante la exploración, el GVD
también debe mantenerse actualizado. En la propuesta original el GVD se
mantiene actualizado aplicando un algoritmo no incremental que
reconstruye completamente el GVD en cada actualización. Dado que gran parte del GVD se
mantiene igual con respecto a su última actualización, la reconstrucción
completa se considera ineficiente. Visto esto la \textbf{primera contribución}
consiste en adaptar e integrar a la propuesta original un algoritmo incremental
que optimice la actualización del GVD solo cambiando las partes necesarias para
que la última versión disponible del GVD quede actualizada. 

% Por otro la definicion de GVD no establece como tratar las porciones
% desconocidas de un entorno parcialmente explorado. Por lo tanto la segunda
% contribucion consiste en un estudio de como se trata dichas porciones
% durante la contrucción del GVD en las propuestas del estado del arte (dado que
% esto no se considera de forma explicita) y la propuesta de una forma
% novedosa que reduce la construcción a las areas conocidas en lugar de contruir
% sobre todo el entorno (conocido y desconocido) como se hace en el resto de
% trabajos.

Por otro lado la definición de GVD no establece como tratar las porciones
desconocidas de un entorno parcialmente explorado. La \textbf{segunda contribución}
consiste en una forma novedosa de tratar con dichas porciones durante la
construcción del GVD, que logra optimizar la construcción reduciéndola a las porciones 
conocidas en lugar de construir sobre todo el entorno (conocido y desconocido)
como se hace usualmente.

Uno de los principales aspectos de la exploración multi-robot es el de
determinar a que lugares se debe enviar a los robots a explorar. Para esto se
suele primero identificar dichos lugares, conocidos como objetivos de
exploración, para luego asignar los objetivos identificados a los
robots.

Respecto a la identificación de objetivos de exploración, identificar grandes
cantidades de objetivos no es deseable en tanto complejiza la posterior tarea de
asignarlos a los robots. La \textbf{tercera contribución} es un
método novedoso que busca reducir la cantidad de objetivos identificados
eliminado objetivos redundantes basándose en las capacidades sensoriales de los
robots. 

Respecto a la asgnación de objetivos de exploración, en el contexto de la
estrategia de coordinación de \cite{wurm2008coordinated}, se debe maximizar la
distribución de los robots sobre los segmentos. Para lograr esto la propuesta
original primero asigna robots a los segmentos, y luego distribuye los
objetivos de cada segmento sobre sus robots asignados. El problema es que
durante la asignación de los robots sobre los segmentos no se considera el
número de objetivos que hay en cada segmento, y por lo tanto es posible
asignar más robots a un segmento que los objetivos que hay disponibles en este,
dejando robots ociosos que podrían ser necesarios en otro segmento. La \textbf{cuarta contribución} consiste en un
algoritmo de asignación de objetivos que logra maximizar la distribución de los
robots sobre los segmentos considerando los objetivos que existen en cada
segmento para evitar robots ociosos.

Luego de que un robot recibe un objetivo de exploración este debe moverse hacia
él, esto introduce la necesidad de planificar caminos hacia los objetivos. En el
contexto del trabajo desarrollado hay dos métodos de planificar que son de
especial interés: planificar sobre el GVD, y planificar sobre todo el espacio
libre. Planificar sobre el GVD es más rápido, y genera caminos más
seguros, mientras que  planificar sobre todo el espacio libre lleva a caminos más
cortos. La \textbf{quinta contribución} es la propuesta de un método de planificación
que combina los métodos antes descritos de forma jerárquica para obtener la
velocidad de la planificación sobre el GVD y el largo de los caminos
planificados sobre todo el espacio libre.

% Las potenciales mejoras buscan acelerar el rendimiento del
% proceso de identificar los segentos del entorno. 

% gvd incremental y otra optimizacion concerniente a como considerar el espacio
% desconocido, cosa que no se especifica 


% . La primera es integración de la contrucción incremental de
% un Diagrama Generalizado de Voronoi (GVD por sus siglas en inglés)k
% \section{Principales aportes}

% Estudiar el impacto al de utlizar una construcción incremental del GVD en la
% tecnica de coordinación que se propone en \cite{wurm2008coordinated}. 

% Propuesta y estudio de un algoritmo novedoso para identificación de objetivos
% basado en la simplificación de fronteras a partir del rango de .

% Para este problema
% se comentan posibles alternativas, pasando por la que esta presente en el
% estado del arte y una técnica novedosa que permite una mayor eficiencia.

Las contribuciones antes mencionadas se implementaron como parte de una
solución al problema de exploración multi-robot. Esta solución fue utilizada para
realizar pruebas dentro de un simulador con el propósito de validar y estudiar
la utilidad las contribuciones. La solución implementada es la \textbf{sexta
contribución} del proyecto y se encuentra disponible en
línea\footnote{\url{https://gitlab.fing.edu.uy/federico.ciuffardi/pgmappingcooperativo}\\Accedido por última vez: 25/02/2022.}.

\section{Organización del documento}
El resto del documento se organiza de la siguiente manera. El capitulo
\ref{cha:marco} introduce problemas, artículos, herramientas y conceptos que
conforman la base del trabajo desarrollado. En el capitulo \ref{cha:central} se
describe la solución al problema multi-robot implementada, incluyendo las
potenciales mejoras que constituyen las contribuciones del proyecto de grado. El
capitulo \ref{cha:exp} se dedica especificar las pruebas realizadas, presentar
sus resultados y analizarlos. Por último en el capitulo \ref{cha:concl} se
plantean las conclusiones del proyecto de grado y se mencionan líneas de
trabajo futuro que surgen a partir del trabajo realizado.


