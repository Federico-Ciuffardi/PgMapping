La exploración es un problema clásico de la robótica
móvil autónoma que consiste en utilizar un robot con el objetivo de obtener
información de un entorno desconocido generando un mapa que lo represente. La
exploración es fundamental en tareas de limpieza \cite{luo2002real}, operaciones
de búsqueda y rescate \cite{Liu2015}, misiones extra planetarias
\cite{schuster2019towards}, entre otras tareas en las cuales se desconozca
el entorno y sea ineficiente o directamente inviable teleoperar a un robot.

\section{Principales aportes}
Propuesta y estudio de un algoritmo novedoso para identificación de objetivos
basado en la simplificacion de fronteras.

Estudiar el impacto al de utlizar una construcción incremental del GVD en la
tecnica de coordinación que se propone en \cite{wurm2008coordinated}. 

Adicionalmente se estudia un aspecto critico que no esta explicito en el estado
del arte, el problema de como como tratar las porciones desconocidas al
contruir un GVD en un entorno parcialemnte explorado. Para este problema se
comentan posibles alternativas, pasando por la que esta presente en el estado
del arte y una técnica novedosa que permite una mayor eficiencia.


\section{Organizacion del documento}
El documento se organiza de la siguiente manera.



