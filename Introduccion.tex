La exploración es un problema clásico de la robótica móvil autónoma que
consiste en utilizar un robot con el objetivo de obtener información de un
entorno desconocido con sus sensores para generar un mapa que lo represente. La
exploración es una parte fundamental en tareas de limpieza \cite{luo2002real},
operaciones de búsqueda y rescate \cite{Liu2015}, misiones extra planetarias
\cite{schuster2019towards}, entre otras tareas en las cuales se desconozca el
entorno y sea ineficiente, o directamente inviable teleoperar a un robot.

La exploración suele ser una tarea altamente paralelizable, al ser usual que en
un mismo instante de tiempo existan varios lugares del entorno de los cuales un
robot puede obtener información novedosa. Esto causa que sea actractivo el uso
de varias robots para acelerar la exploración. Al usar varios robots el
problema pasa a conocerse como el de exploración multirobot.

Sin embargo al tener varios robots dedicados en la tarea de exploración es
necesario algun mecanismo de coordinacion que logre explotar el paralelismo en
la exploración. Para esto se debe evitar situaciones suboptimas como que varios
robots realicen trabajo redundante al explorar los mismos lugares
desaprovechando sus capacidades sensoriales, como tambien que estos interfieran
entre sí, causando perdidas de tiempo en desvíos para evitar colisiones.

En \cite{wurm2008coordinated} se propone una tecnica de coordinación en la
exploración multirobot que establece que la exploración debe llevarse a cabo
maximizando la distribución de los robots sobre las habitaciones y corredores
(segmentos), que componen a un entorno estructurado. Esta tecnica se basa en
que las exploraciones de segmentos diferentes suelen independientes entre sí,
mientras que en que los segmentos pueden ser demasiado pequeños para que un
segundo robot acelere su exploración, y que de haber mas de un robot explorando
un mismo segmento estos pueden bloquearse entre sí mientras intentan
abandonarlo.

El proyecto de grado se propone mejorar la propuesta de
\cite{wurm2008coordinated}, para esto se evaluaron varios aspectos de la
propuesta en cuestión y del problema de exploración multirobot en general
relevando una serie de potenciales mejoras. Estas fueron implementadas como
parte de una solución al problema de exploración multirobot
\footnote{Disponible en línea:\\
\url{https://gitlab.fing.edu.uy/federico.ciuffardi/pgmappingcooperativo}} con
la cual se experimento dentro de un simulador con el propósito de validar su
impacto.

\section{Principales aportes}

Integración de la contrucción incremental de un Diagrama Generalizado de
Voronoi (GVD por sus siglas en ingles) necesario para identificar los segentos
del entorno.

Estudiar el impacto al de utlizar una construcción incremental del GVD en la
tecnica de coordinación que se propone en \cite{wurm2008coordinated}. 

Propuesta y estudio de un algoritmo novedoso para identificación de objetivos
basado en la simplificacion de fronteras a partir del rango de .

Adicionalmente se estudia un aspecto critico que no esta explicito en el estado
del arte, el problema de como como tratar las porciones desconocidas al
contruir un GVD en un entorno parcialemnte explorado. Para este problema se
comentan posibles alternativas, pasando por la que esta presente en el estado
del arte y una técnica novedosa que permite una mayor eficiencia.

\section{Organizacion del documento}
El documento se organiza de la siguiente manera.
El capitulo \ref{cha:marco} \ref{cha:central} \ref{cha:exp} \ref{cha:concl}


