La exploración es un problema clásico de la robótica móvil autónoma que
consiste en utilizar un robot con el objetivo de obtener información de un
entorno desconocido con sus sensores para generar un mapa que lo represente. La
exploración es una parte fundamental en tareas de limpieza \cite{luo2002real},
operaciones de búsqueda y rescate \cite{Liu2015}, misiones extra planetarias
\cite{schuster2019towards}, entre otras tareas en las cuales se desconozca el
entorno y sea ineficiente, o directamente inviable teleoperar a un robot.

La exploración suele ser una tarea altamente paralelizable, al ser usual que en
un mismo instante de tiempo existan varios lugares del entorno de los cuales un
robot puede extraer información novedosa. Esto causa que sea actractivo el uso
de varias robots para acelerar la exploración. Al usar varios robots el
problema pasa a conocerse como el de exploración multirobot.

Sin embargo, al tener varios robots dedicados en la tarea de exploración es
necesario algun mecanismo de coordinacion que logre explotar el paralelismo.
Para esto se debe evitar situaciones suboptimas como que varios robots realicen
trabajo redundante al explorar los mismos lugares, desaprovechando sus
capacidades sensoriales, como tambien que estos interfieran entre sí, causando
perdidas de tiempo en desvíos para evitar colisiones.

En \cite{wurm2008coordinated} se propone una estrategia de coordinación para la
exploración multirobot que consiste en llevar a cabo la exploración maximizando
la distribución de los robots sobre las habitaciones y corredores (llamados
segmentos), que componen a un entorno estructurado. La estrategia se fundamenta
en que las exploraciones de segmentos diferentes suelen ser independientes
entre sí, por lo cual asignar robots a diferentes segemtnos ayuda a evitar
redundacia en la exploración. En que los segmentos pueden ser demasiado
pequeños para que un segundo robot acelere su exploración. Y en que de haber
mas de un robot explorando un mismo segmento estos pueden bloquearse entre sí
mientras intentan abandonarlo.

% El principal objetivo de este proyecto de grado es mejorar la propuesta de
% \cite{wurm2008coordinated}. Luego de un analisis de la propuesta en cuestión y
% del problema de exploración multirobot en general se relevo una serie de
% potenciales mejoras. 

El proyecto de grado se propone mejorar la propuesta de
\cite{wurm2008coordinated}, para esto se evaluaron varios aspectos de la
propuesta en cuestión y del problema de exploración multirobot en general
relevando una serie de potenciales mejoras. El primer aspecto que para el cual se propone mejoras esta relacionado a la
identificación de los segmentos del entorno (segmentación). Uno de los metodos
mas populares y eficientes para segmentar, que es también utlizado en la
propuesta origial basa en utlizar una estructura llamada diagrama generalizado
de voronoi (GVD por sus siglas en ingles). Dado que para aplicar la estrategia
los segmetos deben mantenerse actualizados durante la exploración, el GVD
tambien debe mantenerse actualizado. En la propuesta original el GVD se mantine
actualizado reconstruyendolo completamente segun sea necesario, aplicando un
algoritmo no incremental. Dado que gran parte del GVD se mantiene igual con
respecto a su ultima actualizacion, la recontruccion completa se considera
ineficiente. Visto esto la primera mejora potencial asociada a la segmentación
consiste en aplicar un algoritmo incremental para lograr mantener el GVD
actualizado solo cambiando las partes necesarias para que la ultima version
disponible del GVD quede actualiazada. Por otro lado dado que la definicion de
GVD no establece como tratar las porciones desconocidas de un entorno
parcialemnte explorado, se propone una forma novedosa de tratar con dichas
porciones que busca reducir el procesamiento a las areas conocidas en lugar de
en todo el entorno (conocido y desconocido) como se hace en la propuesta
original.

% Las potenciales mejoras buscan acelerar el rendimiento del
% proceso de identificar los segentos del entorno. 

% gvd incremental y otra optimizacion concerniente a como considerar el espacio
% desconocido, cosa que no se especifica 


% . La primera es integración de la contrucción incremental de
% un Diagrama Generalizado de Voronoi (GVD por sus siglas en ingles)k
% \section{Principales aportes}

% Estudiar el impacto al de utlizar una construcción incremental del GVD en la
% tecnica de coordinación que se propone en \cite{wurm2008coordinated}. 

% Propuesta y estudio de un algoritmo novedoso para identificación de objetivos
% basado en la simplificacion de fronteras a partir del rango de .

% Para este problema
% se comentan posibles alternativas, pasando por la que esta presente en el
% estado del arte y una técnica novedosa que permite una mayor eficiencia.

Las mejoras potenciales previamente descritas fueron implementadas como parte
de una solución al problema de exploración multirobot \footnote{Disponible en
línea:\\
\url{https://gitlab.fing.edu.uy/federico.ciuffardi/pgmappingcooperativo}} con
la cual se realizaron experimentos dentro de un simulador con el propósito de
validar su impacto.


Aca comentar  como resultaron las mejoras

\section{Organizacion del documento}
El documento se organiza de la siguiente manera.
El capitulo \ref{cha:marco} \ref{cha:central} \ref{cha:exp} \ref{cha:concl}


