\section{Hipotesis de trabajo}
\begin{enumerate}
  \item La comunicaciones son sin perdida y de rango infinito
  \item Cada robot conoce en todo momento su ubicacion (posicion y orientacion).
  \item El entrorno a explorar es cerrado.
\end{enumerate}


\section{mejora sobre wurm}
Wurm aparentemente no considera que el numero de robots asigandos a un segmento debe ser menor que el numero de fronteras que tiene ya que no tiene sentido asignar a un robot  a un segmento si eseete no tiene objetivos disponibles. Habria que ver bien como funciona el metodo hungaro, pero creo que es una asignacion robot-segmento que no considera otra cosa que el costo, por lo tanto el num de fronteras de un segmento no se estaria considerando



- Mencionar el uso de las fronteras seignificativsa de \cite{amorin2019novel} como heuristica de coordinacion y que para evitar acumular robots en segmentos

