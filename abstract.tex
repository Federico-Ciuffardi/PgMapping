La exploración es un problema fundamental de la robótica móvil autónoma que
consiste en utilizar un robot con el objetivo de obtener información de un
entorno desconocido con sus sensores para generar un mapa que lo represente.

Determinar el siguiente lugar al cual el robot debe moverse para obtener
información del entorno es uno de los principales aspectos de la exploración
conocido como el problema de asignación objetivos de exploración, donde por
\say{objetivo de exploración} se entiende uno de estos lugares convenientes. 

Por motivos de eficiencia y robustez la exploración suele llevarse a cabo con
más de un robot. Al utilizar varios robots es deseable que la asignación de
objetivos se lleve a cabo siguiendo una estrategia de coordinación para que los
robots cooperen de forma optima, evitando explorar los mismos lugares o
obstaculizarse entre sí.

Los entornos estructurados normalmente están compuestos de habitaciones y
corredores (conocidos como segmentos). Para este tipo de entornos una
estrategia de coordinación consiste en llevar a cabo la exploración maximizando
la distribución de los robots sobre los segmentos. 

El proyecto se plantea como objetivo mejorar dicha estrategia coordinación,
para esto se realizan cinco propuestas que buscan mejorar aspectos específicos
de la estrategia y del problema de exploración multirobot en general.

Dos de las propuestas se concentran en acelerar la construcción de una
estructura llamada Diagrama Generalizado de Voronoi (GVD por sus siglas en
ingles), que es esencial mantener actualizada para reconocer los segmentos
sobre los cuales distribuir a los robots durante la exploración. La primera
propuesta optimiza la actualización solo modificando las partes desactualizadas 
de la ultima version del GVD. La segunda reduce la contracción del GVD a las
partes conocidas del entorno, ahorrando el tiempo de construir sobre las
partes desconocidas.

La tercera propuesta consiste en forma novedosa de identificar los objetivos
que se basa en las capacidades sensoriales de los robots para evitar
identificar objetivos redundantes.

La cuarta propuesta es algoritmo de asignación de objetivos de exploración que
logra asignar los objetivos identificados maximizando la distribución de los
robots sobre los segmentos. La principal ventaja de este algoritmo es que la
asignación considera la cantidad de objetivos que existen en un segmento.

La quinta y ultima propuesta consiste en un método de planificar caminos hacia
los objetivos de exploración que hace uso del GVD (ya disponible por ser usando
en la identificación de segmentos) para optimizar los tiempos de planificación
pero permitiendo que los caminos no estén completamente sobre el GVD evitando
generar caminos innecesariamente largos.

Las propuestas se implementaron como parte de una solución al problema de
exploración multirobot completa con la cual se realizaron pruebas dentro de un
simulador. Los resultados de las pruebas en general indican que las propuestas
logran su objetivo de mejorar los resultados, siendo las propuestas asociadas a
la aceleración de la contracción del GVD las de mayor impacto.
