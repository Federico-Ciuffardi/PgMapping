La exploración es un problema fundamental de la robótica móvil autónoma que
consiste en utilizar un robot con el objetivo de obtener información de un
entorno desconocido con sus sensores para generar un mapa que lo represente. La
exploración es una parte fundamental en tareas de limpieza, operaciones de
búsqueda y rescate, misiones extra planetarias, entre otras tareas en las
cuales se desconozca el entorno y sea ineficiente, o directamente inviable
teleoperar a un robot.

Por motivos de eficiencia y robustes la exploración suele llevarse a cabo con
más de un robot. Para optimizar el uso de los robots es necesario algún
mecanismo de coordinación para que los robots cooperen de forma optima,
evitando explorar los mismos lugares o interferir entre sí.

Una estrategia de coordinación para la exploración multirobot consiste en
llevar a cabo la exploración maximizando la distribución de los robots sobre
las habitaciones y corredores (llamados segmentos), que componen a un entorno
estructurado. 

El proyecto de grado en primer lugar se propone mejorar dicha estrategia
coordinación a partir de acelerar la contruccion de una estructura llamada
grafo generalizado de voronoi, que es escencial para reconocer los segmentos
sobre los cuales distribuir a los robots durante la exploración.

Por otro lado durante la exploración es necesario identificar lugares a los
cuales enviar a los robots conocidos como objetivos de exploración. 

% y del problema de exploración multirobot en general.
% . Las principales mejoras potenciales
% tratadas a lo largo del proyecto se

% El primer aspecto que para el cual se propone mejoras esta relacionado a la
% identificación de los segmentos del entorno (segmentación).

% La estrategia de coordinación requiere identificar los segmentos del entorno,
% lo cual se conoce como segmentación. Uno de los métodos más populares y
% eficientes para segmentar, que es también utilizado en la propuesta original, se
% basa en utilizar una estructura llamada Diagrama Generalizado de Voronoi (GVD
% por sus siglas en ingles). Dado que para aplicar la estrategia de coordinación
% los segmentos deben mantenerse actualizados durante la exploración, el GVD
% también debe mantenerse actualizado. En la propuesta original el GVD se
% mantiene actualizado aplicando un algoritmo no incremental, es decir que se
% reconstruye completamente en cada actualización. Dado que gran parte del GVD se
% mantiene igual con respecto a su ultima actualización, la reconstrucción
% completa se considera ineficiente. Visto esto la primera contribución
% consiste en adaptar e integrar a la propuesta original un algoritmo incremental
% que optimize la actualización del GVD solo cambiando las partes necesarias para
% que la ultima version disponible del GVD sean actualizadas. 

% Por otro lado la definición de GVD no establece como tratar las porciones
% desconocidas de un entorno parcialmente explorado. La segunda contribución
% consiste en una forma novedosa de tratar con dichas porciones durante la
% construcción del GVD, que logra optimizar la construcción reduciéndola a las areas
% conocidas en lugar de construir sobre todo el entorno (conocido y desconocido)
% como se hace usualmente.

% Uno de los principales problemas de la exploración multirobot consiste en
% determinar los lugares a los que enviar a los robots a explorar. Para este
% problema se deben identificar dichos lugares, conocidos como objetivos de
% exploración, para luego distribuir dichos objetivos entre los robots.

% Respecto a la identificación de objetivos de exploración, se tiene que
% identificar grandes cantidades de objetivos puede ser computacionalmente
% restrictivo. La tercera contribución es un método novedoso que busca reducir la
% cantidad de objetivos identificados eliminado objetivos redundantes basándose en
% las capacidades sensoriales de los robots. 

% Respecto a la distribución de objetivos de exploración, en el contexto de la
% estrategia de coordinación de  esta debe asegurar que
% los objetivos sean distribuidos de forma que maximizar la distribución de los
% robots sobre los segmentos. Para lograr esto la propuesta original primero
% asigna robots a los segmentos y luego distribuye los objetivos de dicho segmento
% sobre los robots asignados a dicho segmento. El problema es que esto lo hace sin
% considerar los objetivos que hay en cada segmento por lo tanto se puede dar la
% posibilidad de asignar más robots a un segmento lo que los objetivos que en el
% hay disponibles dejando robots  ociosos. La cuarta contribución consiste en un
% algoritmo de asignación de objetivos que logra maximizar la distribución de los
% robots sobre los segmentos, a la vez que considera los objetivos de cada
% segmento para evitar robot ociosos.

% Luego de que un robot recibe un objetivo de exploración este debe moverse hacia
% él, esto introduce la necesidad de planificar caminos hacia los objetivos. En el
% contexto del trabajo desarrollado hay dos métodos de planificar que son de
% especial interés: planificar sobre el GVD, y planificar sobre todo el espacio
% libre. La planificación sobre el GVD es la más rápida, y genera caminos más
% seguros, mientra que planificar sobre todo el espacio libre lleva a caminos más
% cortos. La quinta contribución es la propuesta de un método de planificación
% que combina los métodos antes descritos de forma jerárquica para obtener la
% velocidad de la planificación sobre el GVD y el largo de los caminos
% planificados sobre todo el espacio libre.

% Las contribuciones antes mencionadas se implementaron como parte de una
% solución al problema de exploración multirobot. Esta solución fue alisada para
% realizar pruebas dentro de un simulador con el propósito de validar y estudiar
% la utilidad las contribuciones. La solución implementada se encuentra
