La exploración es un problema fundamental de la robótica móvil autónoma que
consiste en utilizar un robot para obtener información de un entorno
desconocido, a través de sus sensores, con el objetivo de generar un mapa que lo
represente. Por motivos de eficiencia y robustez la exploración suele llevarse
a cabo con más de un robot, en este caso el problema se conoce como el de
exploración multi-robot.

Uno de los principales problemas de la exploración es determinar los lugares a
los cuales los robot\todo{robots} deben moverse para obtener información del entorno. Esto
se conoce como el problema de asignación \todo{de }objetivos de exploración, donde por
\say{objetivo de exploración} se entiende uno de estos lugares. Cuando se
utiliza más de un robot es deseable que la asignación de objetivos se lleve
a cabo siguiendo una estrategia de coordinación para evitar que los robots
exploren los mismos lugares o que se obstaculicen entre sí.

Los entornos estructurados como edificios, hogares y otras construcciones
humanas son entornos que pueden dividirse en segmentos, como\todo{sobra un como} como habitaciones
y corredores. En este tipo de entornos una posible estrategia de coordinación
es la de llevar a cabo la exploración maximizando la distribución de los
robots sobre los segmentos. 

El proyecto se planteó como objetivo mejorar dicha\todo{se pierde un poco a qué se refiere 'dicha'. quizás agregar info diciendo que lo que se pretenden mejorar son implementaciones previas} estrategia coordinación,
resultando en cinco propuestas que buscan mejorar\todo{se repite mejorar} aspectos específicos de la
estrategia y del problema de exploración multi-robot en general.

Dos de las propuestas se concentran en acelerar la construcción de una
estructura llamada Diagrama Generalizado de Voronoi (GVD por sus siglas en
inglés), que es esencial mantener actualizada para reconocer los segmentos
sobre los cuales distribuir a los robots durante la exploración. La primera
propuesta optimiza la actualización solo modificando las partes desactualizadas 
de la última versión del GVD. La segunda reduce la construcción del GVD a las
partes conocidas del entorno, ahorrando el tiempo de construir sobre las
partes desconocidas.

La tercera propuesta consiste en \todo{una} forma novedosa de identificar los objetivos de exploración
que se basa en las capacidades sensoriales de los robots para evitar
identificar objetivos redundantes.

La cuarta propuesta es\todo{no se si el verbo es 'ser' o debería ser 'introduce'} un algoritmo de asignación de objetivos de exploración
que logra maximizar la distribución de los robots sobre los segmentos evitando
asignar más robots a un segmento que los objetivos que contenidos en él.

La quinta y última propuesta consiste en un método de planificar caminos hacia
los objetivos de exploración que hace uso del GVD (ya disponible por ser usando
en la identificación de segmentos) para reducir los tiempos de planificación
pero a su vez permite que los caminos no sean completamente sobre el GVD evitando
planificar caminos innecesariamente largos.

Las propuestas se implementaron como parte de una solución al problema de
exploración multi-robot completa con la cual se realizaron pruebas dentro de un
simulador. Los resultados de las pruebas en general indican que las propuestas
logran su objetivo de mejorar los resultados\todo{se repite resultados!}, siendo las propuestas asociadas a
la aceleración de la construcción del GVD las de mayor impacto.
\todo[inline]{la última oración, a modo de síntesis de conclusiones, me dejó gusto a poco.}
\todo[inline]{a modo de resumen y para que verifiques en el resto del doc, noté temas con la redacción, y en mucho menor mendida cosas de contenido. revisar ortografía, repetición de palabras, plural/singular, género}
