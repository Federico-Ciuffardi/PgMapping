La exploración es un problema fundamental de la robótica móvil autónoma que
consiste en utilizar un robot para obtener información de un entorno
desconocido, a través de sus sensores, con el objetivo de generar un mapa que lo
represente. Por motivos de eficiencia y robustez la exploración suele llevarse
a cabo con más de un robot, en este caso el problema se conoce como el de
exploración multi-robot.

Uno de los principales aspectos de la exploración es determinar a que lugares
los robots deben moverse para obtener información del entorno. Para esto se
suele primero identificar dichos lugares, conocidos como objetivos de
exploración, para luego asignar los objetivos identificados a los robots.
Cuando se utiliza más de un robot es deseable que la asignación de objetivos se
lleve a cabo siguiendo una estrategia de coordinación para evitar que los
robots exploren los mismos lugares o que se obstaculicen entre sí.

Los entornos estructurados como edificios, hogares y otras construcciones
humanas son entornos que pueden dividirse en segmentos, como habitaciones
y corredores. En este tipo de entornos una posible estrategia de coordinación 
es la de llevar a cabo la exploración maximizando la distribución de los
robots sobre los segmentos. 

El presente proyecto de grado se planteó como objetivo mejorar la propuesta
original de la estrategia de coordinación antes mencionada, resultando en cinco
contribuciones sobre aspectos específicos de la estrategia y del problema de
exploración multi-robot en general.

Dos de las contribuciones se concentran en acelerar la construcción de una
estructura llamada Diagrama Generalizado de Voronoi (GVD por sus siglas en
inglés), que es esencial mantener actualizada para reconocer los segmentos
sobre los cuales distribuir a los robots durante la exploración. La primer
contribución optimiza la actualización solo modificando las partes desactualizadas 
de la última versión del GVD. La segunda reduce la construcción del GVD a las
partes conocidas del entorno, ahorrando el tiempo de construir sobre las
partes desconocidas.

La tercer contribución consiste en una forma novedosa de identificar los objetivos de exploración
que se basa en las capacidades sensoriales de los robots para evitar
identificar objetivos redundantes.

La cuarta contribución introduce un algoritmo de asignación de objetivos de exploración
que logra maximizar la distribución de los robots sobre los segmentos a la vez que evita
asignar más robots a un segmento que los objetivos que contenidos en él.

La quita contribución consiste en un método para planificar caminos hacia
los objetivos de exploración que hace uso del GVD (ya disponible por ser usado
en la identificación de segmentos) para reducir los tiempos de planificación,
pero que a su vez permite que los caminos no sean completamente sobre el GVD evitando
planificar caminos innecesariamente largos.

Se implementó una solución al problema de exploración multi-robot que incluye
las cinco contribuciones antes mencionadas, siendo esta implementación una sexta
contribución. La solución desarrollada se puso a prueba dentro de un simulador. Los
resultados de las pruebas en general confirman que las contribuciones logran su
objetivo de mejorar la propuesta original, siendo las contribuciones asociadas
a la aceleración de la construcción del GVD las de mayor impacto.

% \todoerror[inline]{Como lo puedo mejorar? | la última oración, a modo de síntesis de conclusiones, me dejó gusto a poco.}
% \todo[inline]{a modo de resumen y para que verifiques en el resto del doc, noté temas con la redacción, y en mucho menor mendida cosas de contenido. revisar ortografía, repetición de palabras, plural/singular, género}
