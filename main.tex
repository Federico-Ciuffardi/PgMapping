% = Document class =
% \documentclass[oneside]{book}
\documentclass{report}

% Avoid blank page before chapter in appendix 
% https://tex.stackexchange.com/questions/24066/start-new-chapter-on-same-page/24068#24068 
% \usepackage{etoolbox}
% \makeatletter
% \patchcmd{\chapter}{\if@openright\cleardoublepage\else\clearpage\fi}{}{}{}
% \makeatother

% = Packages =
\usepackage{todonotes}
\newcommand{\todoerror}[2][]{\todo[color=red!70, #1]{#2}}
\newcommand{\todowarn}[2][]{\todo[color=orange, #1]{#2}}
\newcommand{\todoremark}[2][]{\todo[color=yellow!100!black, #1]{#2}}
\newcommand{\todohint}[2][]{\todo[color=green!20, #1]{#2}}
\newcommand{\tododone}[2][]{\todo[color=green!50, #1]{#2}}

% == silence warnings ==
\usepackage{silence}

% Extract of minitoc documentation:
% Some packages alter the sectionning commands, like \part. Most of them should
% be loaded before the minitoc package. The hyperref package, even if it is
% loaded before the minitoc package (as recommended), alters the sectionning
% commands in an \AtBeginDocument, so this message is always printed when you use
% the hyperref package with minitoc, but then it is harmless.
\WarningFilter{minitoc(hints)}{W0023}
\WarningFilter{minitoc(hints)}{W0028}
\WarningFilter{minitoc(hints)}{W0030}
\WarningFilter{minitoc(hints)}{W0024}

% == FONTS ==
\usepackage[T1]{fontenc}
\usepackage[spanish]{babel}
\usepackage[utf8]{inputenc}

% == CAPTION AND QUOTES ==
\usepackage{dirtytalk}
\usepackage[center]{caption}

% == TOC == 
\usepackage{minitoc}
\addto{\captionsspanish}{
  \renewcommand{\mtctitle}{Contenidos}
}
\dominitoc

% == MISC ==
% \usepackage[top= 2.75cm, bottom = 2.50cm, left   = 3.00cm, right  = 2.50cm]{geometry}
\usepackage[toc,page]{appendix}
\usepackage{float}
\usepackage{enumitem} 
\usepackage{colortbl}
% \restylefloat{table}
% \usepackage{enumitem}

% == PDF, URL AND HYPERLINK ***
\usepackage{verbatim}
\usepackage[breaklinks=true]{hyperref}
\usepackage{breakcites}
\hypersetup{
    colorlinks,
    citecolor=black,
    filecolor=black,
    linkcolor=black,
    urlcolor=blue
}

% == MATH ==
\usepackage{amsmath}
\usepackage{mathtools}
\DeclareMathOperator*{\argmin}{argmin}
\newcommand{\mli}[1]{\mathit{#1}}
\usepackage{ mathrsfs }
\usepackage{amssymb}
\usepackage{ upgreek }
\newcommand{\smallsim}{\smallsym{\mathrel}{\sim}}

% small
\makeatletter
\newcommand{\smallsym}[2]{#1{\mathpalette\make@small@sym{#2}}}
\newcommand{\make@small@sym}[2]{%
  \vcenter{\hbox{$\m@th\downgrade@style#1#2$}}%
}
\newcommand{\downgrade@style}[1]{%
  \ifx#1\displaystyle\scriptstyle\else
    \ifx#1\textstyle\scriptstyle\else
      \scriptscriptstyle
  \fi\fi
}
\makeatother
% == CODE ==
\usepackage[ruled,vlined,spanish,onelanguage,linesnumbered]{algorithm2e}
\usepackage{listings}

% == TABLE ==
\usepackage{multicol}
\usepackage{multirow}
\usepackage{adjustbox}
\usepackage{tabularx,ragged2e}
% \renewcommand\tabularxcolumn[1]{m{#1}}% for vertical centering text in X column
\renewcommand\tabularxcolumn[1]{>{\Centering}m{#1}}
\newcolumntype{C}[1]{>{\centering\arraybackslash}m{#1}}
% \renewcommand\tabularxcolumn[1]{p{#1}}% for vertical centering text in X column

% == FIGURE PACKAGES ==
\makeatletter
\newcommand*{\centerfloat}{%
  \parindent \z@
  \leftskip \z@ \@plus 1fil \@minus \textwidth
  \rightskip\leftskip
  \parfillskip \z@skip}
\makeatother
\usepackage{graphicx,wrapfig,lipsum}
\usepackage[caption=false]{subfig}

% == COLOR PACKAGES ==
\usepackage{xcolor}

% = DEFINITIONS =
\hyphenation{op-tical net-works semi-conduc-tor}
\renewcommand\_{\textunderscore\allowbreak}

\definecolor{mBlack}{rgb}{0,0,0}
\definecolor{mGreen}{rgb}{0,0.6,0}
\definecolor{mGray}{rgb}{0.5,0.5,0.5}
\definecolor{mPurple}{rgb}{0.58,0,0.82}
\definecolor{mBlue}{rgb}{0.25,0.25,1}
\definecolor{backgroundColour}{rgb}{0.98,0.98,0.98}
\lstset{
    language=Pascal,
    literate={á}{{\'a}}1
        {ã}{{\~a}}1
        {é}{{\'e}}1
        {ó}{{\'o}}1
        {í}{{\'i}}1
        {ñ}{{\~n}}1
        {¡}{{!`}}1
        {¿}{{?`}}1
        {ú}{{\'u}}1
        {Í}{{\'I}}1
        {Ó}{{\'O}}1
}
\lstdefinestyle{CStyle}{
    backgroundcolor=\color{backgroundColour},   
    commentstyle=\color{mGreen},
    keywordstyle=\color{magenta},
    numberstyle=\small\color{mGray},
    stringstyle=\color{mBlue},
    %basicstyle=\footnotesize,
    breakatwhitespace=true,         
    breaklines=true,                 
    %captionpos=b,
    frame=l,
    %keepspaces=true,                 
    numbers=left,                    
    numbersep=7pt,                  
%    showspaces=false,                
%    showstringspaces=false,
%    showtabs=true,
    literate={\ \ }{{\ }}1,
    tabsize=1,
    language=pascal,
    escapeinside={(*} {*)},
    columns=fullflexible,
    breakautoindent=false,
    framerule=1pt,
    xleftmargin=15pt,
    xrightmargin=0pt,
    breakindent=0pt,
    resetmargins=true
}

% = Cover =

\title{%
\begin{figure}[H]
\vspace{-2.5cm}
  \subfloat{\hspace{0.075\textwidth}}
  \subfloat{\includegraphics[clip=true, width=0.15\textwidth]{logos/udelar.jpg}}
  \subfloat{\hspace{0.2\textwidth}}
  \subfloat{\includegraphics[clip=true, width=0.15\textwidth]{logos/logo-fing.png}}
  \subfloat{\hspace{0.2\textwidth}}
  \subfloat{\includegraphics[clip=true, width=0.15\textwidth]{logos/inco.png}}
\end{figure}
\vspace{0.25cm}
\huge
Exploración multi-robot basada en grillas de ocupación probabilística y digramas de Voronoi\\
}

\author{%
\Large
\vspace{0.5cm}\\
Estudiante: Federico Ciuffardi\\
\vspace{0.5cm}\\
Tutor: Facundo Benavides%
}

\date{}

% \setcounter{secnumdepth}{6}

% = DOCUMENT =
\begin{document}
\pagenumbering{roman}

\maketitle

% == RESET PAGE NUMBER ==
\thispagestyle{empty}
\setcounter{page}{1}


\enlargethispage{1\baselineskip}
\begin{abstract} % facundo
La exploración es un problema fundamental de la robótica móvil autónoma que
consiste en utilizar un robot con el objetivo de obtener información de un
entorno desconocido con sus sensores para generar un mapa que lo represente. La
exploración es una parte fundamental en tareas de limpieza, operaciones de
búsqueda y rescate, misiones extra planetarias, entre otras tareas en las
cuales se desconozca el entorno y sea ineficiente, o directamente inviable
teleoperar a un robot.

Determinar el siguiente lugar al cual el robot debe moverse para obtener
información del entorno es uno de los principales problemas de la exploración
conocido como el problema de asignación objetivos de exploración, donde por
\say{objetivo de exploración} se entiende uno de estos lugares convenientes. 

Por motivos de eficiencia y robustes la exploración suele llevarse a cabo con
más de un robot. Al utilizar varios robots es deseable que la asignación de
objetivos se lleve a cabo siguiendo una estrategia de coordinación para que los
robots cooperen de forma optima, evitando explorar los mismos lugares o
obstaculizarse entre sí.

Los entornos estrucurados normalmente estan compuestos de habitaciones y
corredoses (conocidos como segmentos). Para este tipo de entronos una
estrategia de coordinación consiste en llevar a cabo la exploración maximizando
la distribución de los robots sobre los segmentos. 

El proyecto se plantea como objetivo mejorar dicha estrategia coordinación,
para esto se realizan cinco propuestas que bucan mejorar aspectos especificos
de la estrategia y del problema de exploracion multirobot en general.

Dos de las propuestas se concentran en acelerar la contruccion de una
estructura llamada grafo generalizado de voronoi (GVD por sus siglas en
ingles), que es escencial mantener actualizada para reconocer los segmentos
sobre los cuales distribuir a los robots durante la exploración. La primera
propuesta opimiza la actualizacion solo modificando las partes desactualizadas
de la ultima version del GVD. La segunda reduce la contruccion del GVD a las
partes conocidas del entorno, ahorrando el tiempo de construir sobre las
paretes desconocidas.

La tercera propuesta consiste en forma novedosa de identificar los objetivos
que se basa en las capacidades sensoriales de los robots para evitar
identificar objetivos redundantes.

La cuarta propuesta es algoritmo de asignación de objetivos de exploración que
logra asignar los objetivos identificados maximizando la distribución de los
robots sobre los segmentos. La principal ventaja de este algoritmo es que la
asigancion considera la cantidad de objetivos que existen en un segmento.

La quinta y ultima propuesta consiste en un metodo de planificar caminos hacia
los objetivos de exploración que hace uso del GVD (ya disponible por ser usando
en la identificion de segmentos) para opimizar los tiempos de planificacion
pero permitiendo que los caminos no esten completamnete sobre el GVD evitando
generar caminos inecesriamente largos.

Las propuestas se implementaron como parte de una solucion al problema de
exploración multirobot completa con la cual se realizaron pruebas dentro de un
simulador. Los resultados de las pruebas en general indican que las propuestas
logran su objetivo de mejorar los reultados, siendo las propuestas asociadas a
la aceleracion de la contruccion del GVD las de mayor impacto.

\end{abstract}

% == TOC ==
\hfuzz=10pt 
\tableofcontents
\hfuzz=0pt 

\listoffigures  % facundo
\listoftables   % facundo
\listofalgorithms % facundo

% == CONTENTS == 
\chapter{Introducción}\label{cha:Intro}
\pagenumbering{arabic}
\hfuzz=10pt 
\minitoc
\hfuzz=0pt 
La exploración es un problema clásico de la robótica móvil autónoma que
consiste en utilizar un robot con el objetivo de obtener información de un
entorno desconocido con sus sensores para generar un mapa que lo represente. La
exploración es una parte fundamental en tareas de limpieza \cite{luo2002real},
operaciones de búsqueda y rescate \cite{Liu2015}, misiones extra planetarias
\cite{schuster2019towards}, entre otras tareas en las cuales se desconozca el
entorno y sea ineficiente, o directamente inviable teleoperar a un robot.

La exploración suele ser una tarea altamente paralelizable, al ser usual que en
un mismo instante de tiempo existan varios lugares del entorno de los cuales un
robot puede extraer información novedosa. Esto causa que sea atractivo el uso
de varias robots para acelerar la exploración. Al usar varios robots el
problema pasa a conocerse como el de exploración multirobot.

Sin embargo, al tener varios robots dedicados en la tarea de exploración es
necesario algún mecanismo de coordinación que logre explotar el paralelismo.
Para esto se debe evitar situaciones suboptimas como que varios robots realicen
trabajo redundante al explorar los mismos lugares, desaprovechando sus
capacidades sensoriales, como también que estos interfieran entre sí, causando
perdidas de tiempo en desvíos para evitar colisiones.

En \cite{wurm2008coordinated} se introduce una estrategia de coordinación para la
exploración multirobot que consiste en llevar a cabo la exploración maximizando
la distribución de los robots sobre las habitaciones y corredores (llamados
segmentos), que componen a un entorno estructurado. La estrategia se fundamenta
en que las exploraciones de segmentos diferentes suelen ser independientes
entre sí, por lo cual asignar robots a diferentes segmentos ayuda a evitar
redundancia en la exploración. En que los segmentos pueden ser demasiado
pequeños para que un segundo robot acelere su exploración. Y en que de haber
más de un robot explorando un mismo segmento estos pueden bloquearse entre sí
mientras intentan abandonarlo.

% El principal objetivo de este proyecto de grado es mejorar la propuesta de
% \cite{wurm2008coordinated}. Luego de un analisis de la propuesta en cuestión y
% del problema de exploración multirobot en general se relevo una serie de
% potenciales mejoras. 

El proyecto de grado se propone mejorar la propuesta de
\cite{wurm2008coordinated}, para esto se evaluaron varios aspectos de la
propuesta en cuestión y del problema de exploración multirobot en general
relevando una serie de potenciales mejoras, que resultan en las contribuciones
que se describen a continuación.

% . Las principales mejoras potenciales
% tratadas a lo largo del proyecto se

% El primer aspecto que para el cual se propone mejoras esta relacionado a la
% identificación de los segmentos del entorno (segmentación).

\section{Contribuciones}

La estrategia de coordinación requiere identificar los segmentos del entorno,
lo cual se conoce como segmentación. Uno de los métodos más populares y
eficientes para segmentar, que es también utilizado en la propuesta original, se
basa en utilizar una estructura llamada Diagrama Generalizado de Voronoi (GVD
por sus siglas en ingles). Dado que para aplicar la estrategia de coordinación
los segmentos deben mantenerse actualizados durante la exploración, el GVD
también debe mantenerse actualizado. En la propuesta original el GVD se
mantiene actualizado aplicando un algoritmo no incremental, es decir que se
reconstruye completamente en cada actualización. Dado que gran parte del GVD se
mantiene igual con respecto a su ultima actualización, la reconstrucción
completa se considera ineficiente. Visto esto la primera contribución
consiste en adaptar e integrar a la propuesta original un algoritmo incremental
que optimize la actualización del GVD solo cambiando las partes necesarias para
que la ultima version disponible del GVD sean actualizadas. 

% Por otro la definicion de GVD no establece como tratar las porciones
% desconocidas de un entorno parcialmente explorado. Por lo tanto la segunda
% contribucion consiste en un estudio de como se trata dichas porciones
% durante la contrucción del GVD en las propuestas del estado del arte (dado que
% esto no se considera de forma explicita) y la propuesta de una forma
% novedosa que reduce la construcción a las areas conocidas en lugar de contruir
% sobre todo el entorno (conocido y desconocido) como se hace en el resto de
% trabajos.

Por otro lado la definición de GVD no establece como tratar las porciones
desconocidas de un entorno parcialmente explorado. La segunda contribución
consiste en una forma novedosa de tratar con dichas porciones durante la
construcción del GVD, que logra optimizar la construcción reduciéndola a las areas
conocidas en lugar de construir sobre todo el entorno (conocido y desconocido)
como se hace usualmente.

Uno de los principales problemas de la exploración multirobot consiste en
determinar los lugares a los que enviar a los robots a explorar. Para este
problema se deben identificar dichos lugares, conocidos como objetivos de
exploración, para luego distribuir dichos objetivos entre los robots.

Respecto a la identificación de objetivos de exploración, se tiene que
identificar grandes cantidades de objetivos puede ser computacionalmente
restrictivo. La tercera contribución es un método novedoso que busca reducir la
cantidad de objetivos identificados eliminado objetivos redundantes basándose en
las capacidades sensoriales de los robots. 

Respecto a la distribución de objetivos de exploración, en el contexto de la
estrategia de coordinación de \cite{wurm2008coordinated} esta debe asegurar que
los objetivos sean distribuidos de forma que maximizar la distribución de los
robots sobre los segmentos. Para lograr esto la propuesta original primero
asigna robots a los segmentos y luego distribuye los objetivos de dicho segmento
sobre los robots asignados a dicho segmento. El problema es que esto lo hace sin
considerar los objetivos que hay en cada segmento por lo tanto se puede dar la
posibilidad de asignar más robots a un segmento lo que los objetivos que en el
hay disponibles dejando robots  ociosos. La cuarta contribución consiste en un
algoritmo de asignación de objetivos que logra maximizar la distribución de los
robots sobre los segmentos, a la vez que considera los objetivos de cada
segmento para evitar robot ociosos.

Luego de que un robot recibe un objetivo de exploración este debe moverse hacia
él, esto introduce la necesidad de planificar caminos hacia los objetivos. En el
contexto del trabajo desarrollado hay dos métodos de planificar que son de
especial interés: planificar sobre el GVD, y planificar sobre todo el espacio
libre. La planificación sobre el GVD es la más rápida, y genera caminos más
seguros, mientra que planificar sobre todo el espacio libre lleva a caminos más
cortos. La quinta contribución es la propuesta de un método de planificación
que combina los métodos antes descritos de forma jerárquica para obtener la
velocidad de la planificación sobre el GVD y el largo de los caminos
planificados sobre todo el espacio libre.

% Las potenciales mejoras buscan acelerar el rendimiento del
% proceso de identificar los segentos del entorno. 

% gvd incremental y otra optimizacion concerniente a como considerar el espacio
% desconocido, cosa que no se especifica 


% . La primera es integración de la contrucción incremental de
% un Diagrama Generalizado de Voronoi (GVD por sus siglas en ingles)k
% \section{Principales aportes}

% Estudiar el impacto al de utlizar una construcción incremental del GVD en la
% tecnica de coordinación que se propone en \cite{wurm2008coordinated}. 

% Propuesta y estudio de un algoritmo novedoso para identificación de objetivos
% basado en la simplificacion de fronteras a partir del rango de .

% Para este problema
% se comentan posibles alternativas, pasando por la que esta presente en el
% estado del arte y una técnica novedosa que permite una mayor eficiencia.

Las contribuciones antes mencionadas se implementaron como parte de una
solución al problema de exploración multirobot. Esta solución fue alisada para
realizar pruebas dentro de un simulador con el propósito de validar y estudiar
la utilidad las contribuciones. La solución implementada se encuentra disponible
en
línea\footnote{\url{https://gitlab.fing.edu.uy/federico.ciuffardi/pgmappingcooperativo}}
y es la sexta contribución de este proyecto.

\section{Organización del documento}
El resto del documento se organiza de la siguiente manera. En capitulo
\ref{cha:marco} introduce problemas, artículos, herramientas y conceptos que
conforman la base del trabajo desarrollado. En el capitulo \ref{cha:central} se
describe la solución al problema multirobot implementada incluyendo las mejoras
potenciales. El capitulo \ref{cha:exp} se dedica especificar las pruebas
realizadas, presentar sus resultados y analizarlos. Por ultimo en el capitulo
\ref{cha:concl} se plantean las conclusiones del proyecto de grado y se
mencionan líneas de trabajo futuro que surgen a partir del trabajo realizado.




\chapter{Marco de trabajo}\label{cha:marco}
\hfuzz=10pt 
\minitoc
\hfuzz=0pt 

\section{Estado del arte}

%%%%%%%
% Desde acá TSCF
%%%%%%%
% \section{Investigación previa}
% Con el motivo de introducirnos y orientar el trabajo, se investigaron cuatro artículos relacionados al problema de la exploración multi-robot. A continuación se describirá cada uno de los artículos, haciendo especial énfasis en las características de los mismos que se relacionan con el trabajo realizado.
\subsection{A novel stop criterion to support efficient Multi-robot mapping}\cite{amorin2019novel}
El articulo modela e implementa una solución al el problema de coordinación y exploración de un entorno desconocido a partir de un sistema de robots homogéneos, con el objetivo de analizar un criterio de terminación novedoso basado en la ganancia de información esperada.

% El articulo consta de varias secciones: la representación del entorno, los componentes(robot y estación central) y comunicaciones entre ellos, método de identificación de tareas para cada robot y el criterio de finalización.

\subsubsection{Representación del entorno}
El entorno $E$ se define como un espacio de trabajo plano limitado limitado $E\subseteq R^2$. Para que sea computacionalmente tratable dicho entorno $E$ está representado por una estructura de grilla de ocupación donde cada celda $c$ puede pertenecer a tres estados probabilísticos diferentes $S = \{f, o, u\}$, representando libre, ocupado y desconocido, respectivamente. Para realizar las asignaciones solo se considerara el centro de la celda.

\subsubsection{Componentes y comunicaciones}
La flota se compone de $N$ robots móviles circulares rígidos con capacidad para tomar mediciones en toda la circunferencia alrededor con un radio $r$. Existe una estación central encargada de la tarea de reconocimiento de objetivos y asignación de los mismos. Los robots serán quienes recopilen la información del entorno y la central la encargada de crear un mapa global a partir de esta. 

En el articulo, se consideran comunicaciones ideales, suponiendo que los robots no tienen restricciones de comunicación (por ejemplo, sin errores ni pérdidas con ancho de banda y alcance ilimitados). %Cabe aclarar que las comunicaciones inalámbricas son importantes en el contexto de exploración multi-robot.

\subsubsection{Método de identificación y asignación de tareas}

Las celdas que llevan a espacios desconocidos, consideradas como libres y que son adyacentes a celdas desconocidas, son denominadas celdas frontera (fronteras), en tanto conforman una frontera entre espacios libres y desconocidos. Las fronteras son objetivos potenciales de exploración ya que si el robot alcanzara dicha celda este podría recopilar nueva información dentro del radio de sensado. Sin embargo, el tratamiento de todas las fronteras como tareas de exploración diferentes podría ser computacionalmente prohibitivo. Por lo tanto, para reducir la carga, se intenta determinar las celdas frontera mas significativas, para lograr esto se hace lo siguiente, primero, las fronteras se dividen en conjuntos disjuntos $F_k$. 
Luego, se obtienen las fronteras mas significativas de cada conjunto $F_k$ considerando el radio de sensado de un robot de forma que si dos robots van a dos celdas frontera significativas distintas de $F_k$, sus radios de sensado no se solapen, esto se hace utilizando el algoritmo K-Means. Finalmente las fronteras consideradas significativas para algún $F_k$ se convierten en celdas objetivo, también conocidas como tareas de exploración.

La asignación de tareas se resuelve organizando una subasta entre robots. Cuando un robot completa una tarea (llegando a una celda objetivo), este envía una notificación a la estación central. La Estación Central comparte la última versión del mapa con la flota e inicia un proceso de subasta solicitando postores. Durante un corto período de tiempo, cada robot no asignado es notificado, pudiendo enviar una oferta; cada oferta consiste en una lista de pares tarea-utilidad, donde la utilidad considera tanto la estimación de ganancia de información como el costo asociado a la ruta necesaria para cumplir con la tarea. Finalmente la central, a partir de las listas de todos los postores, resuelve las asignaciones de forma voraz, para luego notificar a cada robot su tarea asignada.
% Dada la falta de conocimiento sobre el entorno, la mejor opción para los robots es visitar los lugares donde la ganancia de información puede ser potencialmente mayor, sin dejar de considerar el esfuerzo necesario llegar al objetivo, por lo tanto, los robots priorizarán las tareas por el coeficiente entre la ganancia de información esperada y el esfuerzo esperado necesario para completar la tarea.

El valor de la ganancia de información de cada frontera resultante debe ser estimado, en el artículo se presenta un estimador de ganancia, el cual se basa en el rango de visión del robot y el entorno conocido. 

\subsubsection{Criterio de parada}

El criterio según el cual detener la exploración es un aspecto importante de la exploración multi-robot. Una de las opciones sería la configuración del sistema para que, al llegar a un porcentaje de cubrimiento del entorno, la exploración sea detenida. Esta medida en la práctica no es efectiva, solo sirve en caso de tener conocimiento del mapa global. Otra opción es que el sistema verifique de forma autónoma si existen espacios accesibles desde donde cualquier robot pueda recopilar nueva información, y detener la exploración de lo contrario.

El criterio de parada propuesto en el articulo, se basa en que durante la exploración, los robots obtienen conocimiento sobre el entorno que mejora continuamente sus habilidades de predicción. Teniendo en cuenta que una suposición básica es que la flota explora entornos limitados, las hipótesis son dos, la primera es que existe un momento a partir del cual la estimación de la ganancia de información se puede hacer con precisión (es decir, el error cometido por el estimador tiende a cero). Y la segunda es que dicho momento puede determinarse en línea mediante robots que calculan su propio error en sus estimaciones de ganancia de información. Dicho esto el criterio de parada es que dada una cierta tolerancia cada robot da por concluida su participación en la exploración cuando para $n$ objetivos seguidos, dicha tolerancia es mayor a la diferencia entre la estimación de ganancia de información que se hizo para un objetivo y la ganancia de información real. La idea subyacente es que tan pronto como los robots obtengan suficiente información sobre el entorno como para poder estimar la porción inexplorada con precisión, estos podrán detenerse.

\subsection{Coordinated Multi-Robot Exploration using a Segmentation of the Environment}\cite{wurm2008coordinated}
El articulo describe y analiza un enfoque de coordinación para la exploración multi-robot que se basa en la distribución eficiente de los robots al explorar el entorno, teniendo en cuenta la estructura de dicho entorno. Para lograr la distribución el entorno se divide en segmentos, por ejemplo, correspondientes a habitaciones y corredores. Luego, las asignaciones de robots a objetivos se hacen considerando que los robots deben ser distribuidos de forma uniforme sobre los segmentos identificados. 

% Y en lugar de considerar todas las fronteras entre áreas desconocidas y exploradas como ubicaciones objetivo, se envían los robots a los segmentos individuales con la tarea de explorar el área correspondiente.
En el articulo se describen dos tareas principales: la segmentación del mapa y la asignación de objetivos a los robots.

\subsubsection{Segmentación del mapa}
% La segmentación de mapas se basada en la partición de un grafo.  
% , como lo son los grafos de Voronoi
En el articulo el mapa es representado como un grafo de Voronoi, para calcular el grafo de Voronoi $G(m) = (V, E)$ asociado a un mapa $m$, se considera el conjunto $O_p (m)$ que contiene para cada punto $p$ en el espacio libre $C$ de $m$ el conjunto de puntos de obstáculos más cercanos. El conjunto $V$ de nodos del grafo de Voronoi es definido como el conjunto de puntos $p$ en $C$ para los cuales hay al menos dos puntos de obstáculo a distancia mínima, existiendo una arista entre dos nodos si sus puntos correspondientes en $m$ son adyacentes:
\[
V = \{p \in C : |O_p (m)| \geq 2\}
\]
\[
E = \{(p, q) : p, q \in V, p \text{ adjacent } q \text{ in } m\}
\]

El grafo de Voronoi se puede generar a partir de una grilla de ocupación, aplicándole la transformación de distancia euclidiana\cite{meijster2002general} se obtiene como resultado un mapa de distancia que mantiene para cada celda de la grilla la distancia al obstáculo más cercano, a partir del cual es posible determinar $V$ y $E$ como se describió anteriormente.
% (o de interés para explorar, normalmente corredores o puertas)
%, los cuales normalmente se ubican en puertas 

La segmentación de mapas se basa en la partición del grafo utilizado para representar el mapa, y se realiza a partir de los denominados puntos críticos, definidos como puntos que cumplen con ser mínimos locales según la distancia a los obstáculos. La idea es utilizar los puntos críticos para reconocer pasajes entre dos habitaciones o entre habitaciones y corredores. Dado que la forma de los pasajes suelen generar puntos críticos en ellos, el problema pasa a ser el de evitar reconocer falsos positivos (puntos críticos que no estén en un pasaje). Con el motivo de evitar falsos positivos los puntos críticos se restringen a los nodos del grafo que sean mínimos locales según la distancia a los obstáculos, de grado 2 (tengan dos aristas), que tengan un vecino de grado 3 (un nodo de intersección entre varios caminos) y que conduzcan a áreas desconocidas.
Finalmente los pasajes reconocidos determinan una partición del grafo que a su vez determina los segmentos, que por estar entre pasajes serán corredores y habitaciones.

\subsubsection{Asignación de objetivos}
% Los entornos interiores son en general estructurados, por ejemplo, los edificios generalmente se dividen en habitaciones a las que se puede llegar por pasillos.
Los objetivos serán, como es usual, las fronteras entre las partes desconocidas y conocidas del mapa.

Asignar más de un robot a un mismo segmento en muchos casos, puede ser una desventaja, ya que el segmento podría ser demasiado pequeño para que un segundo robot acelere su exploración, aunque inicialmente haya más de una frontera en el mismo o, por otro lado, al terminar la exploración de un segmento, los robots pueden incluso bloquearse entre sí mientras intentan abandonarlo, lo que aumentará el tiempo de exploración. Por lo tanto, el objetivo es distribuir uniformemente a los robots sobre los segmentos.

Para resolver la asignación se propone utilizar el método húngaro, a partir de los costos $C_{s}^{i}$ definidos como el costo de llegar a la celda frontera más cercana del segmento $s$ con el robot $i$, descontando un valor constante en caso de que el robot ya se encuentre en el segmento. El método húngaro da soluciones que evitan tener más de un robot por segmento mientras esto sea posible, y asigna más de un robot a un segmento en el caso de que haya más robots que segmentos. Esta descripción corresponde al algoritmo \ref{alg:asignacionobjetivos}.

\begin{algorithm}
\SetAlgoLined
    Determinar segmentos $S = \{s_{1} , ..., s_{n} \}$ del mapa\\
    Determinar el conjunto de fronteras objetivo para cada segmento\\
    \For{cada robot $i$}{
        \For{cada segmento $s \in S$}{
                Computar el costo $C_{s}^{i}$\\
                Descontar al costo $C_{s}^{i}$ si el robot $i$ ya se encontraba en $s$\\
            }
    }
    Asignar robots a los segmentos usando el método húngaro\\
    \For{cada segmento $s \in S$}{
        Asignar robots a las fronteras objetivas en $s$ utilizando el método húngaro\\
    }
    \caption{Asignación de objetivos}
    \label{alg:asignacionobjetivos}
    
\end{algorithm}

Usando este enfoque, cada uno de los corredores es explorado completamente por uno de los robots revelando rápidamente la estructura de un edificio, mientras que se asignarán otros robots a las habitaciones accesibles desde los corredores a medida que estas se vayan detectando.

\subsection[Voronoi-Based Space Partitioning for Coordinated Multi-Robot Exploration]{Voronoi-Based Space Partitioning for Coordinated\\ Multi-Robot Exploration} \cite{wu2007voronoi}

% Este artículo describe y evalúa la extensión de un algoritmo de exploración multi-robot y muestra que al reemplazar el modelo de mapa original (una grilla de ocupación) con una representación poligonal más compacta y flexible, el nuevo enfoque aumenta significativamente la eficiencia de la etapa más costosa del algoritmo original, que es la división de áreas desconocidas en tantas regiones como robots. El algoritmo de agrupación K-Means original se sustituye por un algoritmo de partición basado en Voronoi aplicado a polígonos.
Las grillas de ocupación es una de las representaciones de mapas mas utilizadas en el contexto de exploración multi-robot, sin embargo, estas grillas no son apropiadas para entornos que son muy grandes o cuyos límites no están bien delimitados desde el comienzo de la exploración. En contraste, las representaciones poligonales no tienen tales limitaciones.

El artículo propone utilizar una representación poligonal en la cual el mapa consiste en la union de conjunto de polígonos cerrados de forma y tamaño arbitrario, que pueden estar libres, no explorados u obstaculizados. 

El desempeño de la representación propuesta se comprueba contra una estrategia de exploración multi-robot que hace uso de grillas de ocupación, de lo cual se concluye que una representación poligonal es más compacta, flexible y logra aumentar significativamente la eficiencia.

\subsubsection{Representación poligonal}
Inicialmente, todo el mapa estará constituido por un solo polígono desconocido. A partir de lo sensado por cada robot, se incluye nueva información sobre entorno a la representación agregando polígonos libres y ocupados que son substraídos de los polígonos desconocidos a los que pertenecían.

Los objetivos de exploración se determinan a partir de las aristas entre polígonos desconocidos y polígonos libres denominadas como aristas frontera que son análogas a las celdas frontera de las grillas de ocupación.

El trabajo propone asignar tareas a partir de la division el entorno desconocido en tantas regiones como robots existan en la flota, para luego asignar a cada robot la tarea de explorar una region diferente. El entorno se divide a partir de un algoritmo que adapta al algoritmo K-Means para funcionar a partir de la representación poligonal propuesta, este algoritmo se explica en la sección\ref{subsubsec:particionamientovoronoi}.

%La division del entorno se logra adaptando el algoritmo K-means para funcionar en utilizado para encontrar los centroides correspondientes a las agrupaciones de áreas desconocidas, ya que se asume que estos serán los puntos más provechosos para que los robots exploren.

Finalmente, la planificación del camino del robot es realizada en el interior de los polígonos libres, cosa que se puede hacer a partir de la aplicación de cualquier algoritmo de descomposición celular, por ejemplo el que se explica en \cite{schachter1978decomposition}.
   
\subsubsection{Particionamiento basado en polígonos}\label{subsubsec:particionamientovoronoi}
El diagrama de Voronoi\cite{fortune1987sweepline} de un conjunto de puntos 2D, también conocidos como sitios, $C_{i} , 1 \leq i \leq K$, es una partición de ese espacio en $K$ regiones convexas disjuntas conocidas como células Voronoi. Cada región $V_i$ está definida por los puntos en el espacio que están más cerca de $C_{i}$ que a cualquier otro $C_{j}$, $j\neq i$. 

Sea $n$ numero de robots, el resultado deseado es obtener un conjunto de $K=n$ celdas de Voronoi para las cuales, sus centroides (centros de masa) y sus sitios estén a una distancia menos que cierta tolerancia $\varepsilon$. % cerradas que están globalmente delimitadas por el polígono a dividir

La generación de estas celdas se realiza a partir del algoritmo \ref{alg:particionamientovoronoi} que adapta K-Means (algoritmo que encuentra clusters en un mapa discretizado) a partir de diagramas de Voronoi.

Notar que el algoritmo trabaja con celdas de Voronoi restringidas, estas son las celdas resultantes de un diagrama de Voronoi restringidas a los limites del mapa conocidos por hipótesis.

\begin{algorithm}
\SetAlgoLined
    Elegir aleatoriamente $K$ puntos $C_i$, $1 \leq i \leq K$, contenidos en los polígonos correspondientes a las regiones desconocidas actuales en el mapa.\\
    Calcular el diagrama de Voronoi asociado con el conjunto actual de $C_i$.\\
    Restringir las celdas del diagrama de Voronoi a los polígonos desconocidos actuales.\\
    Determinar el centro de masa $M_i$ de cada celda Voronoi restringida.\\
    \eIf{$C_i - M_i < \varepsilon\ \forall i$}{
        Saltar al paso 11.\\
    }{
        Sustituir cada $C_i$ por su $M_i$ correspondiente.\\ Volver al paso 2.\\
    }
    
    Dividir el conjunto de polígonos desconocidos en $K$ regiones disjuntas según las celdas de Voronoi restringidas encontradas.\\
    \caption{Particionamiento basado en Voronoi}
    \label{alg:particionamientovoronoi}
    
\end{algorithm}

%%%%%%%
% Desde acá modulo taller (redactado)
%%%%%%%
% hasta acá redactado

\subsection{Incremental reconstruction of generalized Voronoi diagrams on grids}

El principal problema con este acercamiento es que el GVD se construye basa en determinar obstáculos independientes para luego calcular el GVD. Esto es un problema porque no es claro en como pasar de celdas obstaculizadas a obstáculos que están compuestos por estas (buscar mas problemas, ineficiencia?, nuestro código no lo soportaba de antes).  El siguiente artículo propone una solución que evita esta necesidad y solo basta con tener un grilla de ocupación.

\subsection{Efficient grid-based spatial representations for robot navigation in dynamic environments}
Articulo del cual obtuvimos varias las características de nuestra implementación incremental

\subsection{Sensor-based exploration: Incremental construction of the hierarchical generalized Voronoi graph}

Obtener de este las definiciones de GVD GVG VD etc

Ademas en este se muestra que HGVG es un roadmap y como GVD = HGVG en dimension 2 entonces se tiene que GVD es un roadmap y esto justifica su uso para la navegación.

Se puede explicar que es un roadmap y para que sirve.
%https://docs.google.com/document/d/1yFqYCa47JHWA1tYXfWRbPnJMDWjv4nsER7b-64Yx2YM/edit#heading=h.w7bvjvihzhqy
%https://docs.google.com/document/d/1pxqyMlNkoJkrIP6SBvMu3rOycAUBt72uPNH_S8o-HYw/edit#heading=h.l9jal9qqz7kt

%%%%%%%
% Desde acá proyecto de grado redactado
%%%%%%%

\subsection{Distributed Multi-robot Exploration Based on Scene Partitioning and Frontier Selection}
% Al considerase una solución para el problema de exploración hay dos aspectos centrales a resolver, identificar posibles objetivos de exploración y como priorizar estos objetivos de forma de priorizar los mas relevantes.

Este articulo propone una solución distribuida al problema de exploración multi-robot%, que al igual que otras ya comentadas partición el entorno aunque en este caso dicha 

\subsubsection{Formulación del problema}
La variante del problema de exploración multi-robot tratada en este articulo es la de explorar un entorno desconocido $W\in R^2$ utilizando un conjunto de robots $R={R_1,R_2,...,R_n}$. Se utilizan celdas de ocupación, por lo que el entorno se descompone en un conjunto de celdas $C={c_1,c_2,...,c_n}$. Cada robot $R_i$ tiene:
\begin{enumerate}
  \item Un mapa $M_i \in CxL$ donde $L$ son las diferentes estados en los que puede estar una celda. 
  \item Una posición inicial $q_{init}^{i}\in R^2 \times [-\pi,\pi]$. 
  \item Un sensor $S_i$ para explorar el entorno desconocido.
\end{enumerate}

Adicionalmente asumen que existe siempre un cuadro delimitador de $W$ que puede ser establecido, aunque sea de forma aproximada, antes de que la exploración comience, comunicación perfecta (sin perdida de rango infinito) y que cada robot tiene acceso a su posición respecto a un marco de referencia común.

\subsubsection{Descripción del sistema}
El sistema tiene un diseño modular, este consta de 4 módulos que se encuentran en cada robot, el modulo de distribución de información, el modulo de asignación de zonas, el modulo de adquisición de información y el modulo de exploración. Estos se describen en las siguientes secciones. 

\paragraph{Modulo de distribución de información}
Este modulo se encarga de distribuir a los demás robots la posición del robot al que pretence, las ubicaciones que conoce de los demás robots y su mapa $M_i$. 

\paragraph{Modulo de asignación de zonas }
El entorno es segmentado en zonas y la exploración cada una de estas es asignada a un robot. El modulo de asignación de zonas es el responsable de realizar la segmentación en zonas y su posterior asignación a los robots. La segmentación genera tantas zonas $E_i$ como robots $R_i$ que $E_i$ se define en la ecuación \ref{ec:zones} a partir de la posición $C_i$ de $R_i$. .
\begin{equation}\label{ec:zones}
  E_i=\{c_j:d_g(C_i,c_j)\leq d_g(V_i(k),c_j) \forall i \neq k , c_j \in U_i\}
\end{equation}
Donde $U_i \subset M_i$ son la celdas desconocidas del mapa asociado al robot $R_i$ y $d_g(c',c'')$ es la distancia geodésica, que indica el camino compuesto de centros de celdas mas corto entre dos celdas $c'$ y $c''$ que respecta la conectividad entre las celdas que lo componen.

Cada zona $E_i$ es asignada al robot $R_i$ y este explorara las celdas de $E_i$ según lo determine el \hyperref[par:estar:moduloexp]{modulo de exploración}.

\paragraph{Modulo de adquisición de información}
El propósito de este es el de actualizar el mapa $M_i$ de el robot $R_i$ según la información obtenida por su sensor $S_i$ y su posición $C_i$.

\paragraph{Modulo de exploración}\label{par:estar:moduloexp}
El modulo de exploración esta compuesto por tres submódulos. 

El primero de ellos es el submódulo de asignación de objetivos, como su nombre lo indica, este se encarga de seleccionar un objetivo de todos los pertenecientes a la zona asignada $E_i$ para asignarlo al robot $R_i$. El objetivo seleccionado $G_i$ es el que alcanza el menor valor para la función de peso $f_p$ que se computa como \ref{ec:weight}.
\begin{equation}\label{ec:weight}
  f_p(c_j) = k_d.d_g(C_i,c_j) + k_a.\phi_i(c_j)
\end{equation}
Donde $k_d$ y $k_a$ son dos constantes positivas que determinan el peso de cada sumando, $d_g(C_i,c_j)$ hace referencia a la distancia geodésica mencionada anteriormente, $\phi_i(c_j)$ es el angulo entre la orientación del robot y el vector con origen en la posiciones del robot $C_i$, y fin en el centro de celda $c_j$. 

Luego esta el submódulo de planificación de ruta, este se encarga de generar una ruta que le permita al robot llegar al objetivo asignado. Esto se logra a partir del algoritmo $A*$, para la ejecución de este se asume que las celdas desconocidas son libres. En el caso de encontrarse un obstáculo en la ruta planificada se replanifica ejecutando nuevamente el algoritmo $A*$ en la version mas reciente del mapa.

El ultimo es el submódulo de ejecución de trayectoria que se encarga de generar los comandos de actuación para seguir una ruta planeada.

\subsection{Learning metric-topological maps for indoor mobile robot navigation}
En este articulo se propone un método de segmentación del espacio a partir de mapa representado con una grilla de ocupación. El concepto de segmento es equivalente al de habitaciones y corredores en entornos estructurados, como lo son las casas y oficinas. La idea principal del algoritmo de segmentación presentado es que las entradas y salidas de los segmentos son pasajes estrechos, y por lo tanto encontrar estos pasajes estrechos permite delimitar segmentos. En este caso el objetivo de segmentar el espacio en habitaciones y corredores, es el de generar un mapa topológico.

\subsubsection{Mapas topológicos}
Los mapas topológicos son mapas simplificados al punto de solo incluir regiones de interés y conexiones entre estas. En este caso las regiones de interés serán los segmentos. 

La simplicidad de los mapas topológicos es útil para ayudar a los humanos a entender un entorno (por esto es usual ver mapas topológicos de sistemas de transporte) y permiten instrucciones que nos resultan naturales, por ejemplo, "ir a la habitación A". Otra ventaja es que por su simplicidad estos suelen ser mas compactos y por lo tanto pueden utilizarse para desarrollar técnicas de planificación eficiente.

\subsubsection{Algoritmo de segmentación}
Para introducir el algoritmo de segmentación se definen algunos conceptos fundamentales que se se mencionaran a lo largo del informe:
\begin{enumerate}
  \item Puntos base: Dado un punto en el espacio, los puntos base es el conjunto de los puntos de espacio ocupado que están a la misma minima distancia.
  \item Despeje: Es la distancia de un punto en el espacio a sus puntos base.
  \item Diagrama Generalizado de Voronoi (GVD): Conjunto de puntos del espacio libre que tienen al menos dos puntos base.
  \item Puntos críticos: Puntos pertenecientes al GVD que a su vez son mínimos locales según el despeje.
  \item Lineas criticas: Segmentos de linea que conectan a un punto critico con sus puntos base.
\end{enumerate}

Como trabaja sobre una grilla de ocupación por lo que se trabaja con celdas en lugar de con puntos. El algoritmo comienza por la construcción del GVD asociado la grilla de ocupación, posteriormente se encuentran los puntos críticos los cuales se ubican en el centro de pasajes estrechos, para luego obtener las lineas criticas asociadas a cada punto critico. Las lineas criticas estarán ubicadas a lo largo de cada pasaje estrecho y por lo tanto serán los limites entre cada segmento obteniendo así la segmentación deseada.

\subsection[Incremental Topological Segmentation for Semi-structured Environments using discretized GVG]{Incremental Topological Segmentation for Semi-\\structured Environments using discretized GVG}

Este articulo se centra en la descripción de un método de segmentación incremental del entorno que se basa en la propuesta por Thrun \cite{Thrun1998}. Para lograr la incrementalidad se desarrollaron variantes incrementales para la generación del GVD y para la detección de segmentos.

El algoritmo de generación incremental del GVD se basa en el desarrollado en \cite{kalra2009incremental}. 

La detección incremental de puntos críticos se hace solo considerando la definición básica de Thrun, sin considerar las técnicas de remoción de falsos positivos propuestas por Wurm \cite{wurm2008coordinated}, esto facilita la tarea de detección pero impacta negativamente los resultados. Para lograr la mencionada detección incremental de puntos críticos, se recalculan los mínimos locales según los cambios del GVD (nuevos nodos, nodos removidos, cambios de despeje) de cada incremento, específicamente se recalcula en los cambios y sus vecinos de primer y segundo grado. Al detectar nuevo punto critico se calcula la linea critica asociada.

Finalmente es necesario, a partir de las lineas criticas, determinar de forma incremental a que segmento pertenece cada celda. Para esto, se elije una celda de todas las celdas afectadas en el incremento y se la agrega a una nueva region, sus celdas vecinas que estén dentro del mapa, no estén ocupadas y no sean cruzadas por una linea critica serán agregada...



% \begin{algorithm}
% \SetAlgoLined
%     Determinar segmentos $S = \{s_{1} , ..., s_{n} \}$ del mapa\\
%     Determinar el conjunto de fronteras objetivo para cada segmento\\
%     \For{cada robot $i$}{
%         \For{cada segmento $s \in S$}{
%                 Computar el costo $C_{s}^{i}$\\
%                 Descontar al costo $C_{s}^{i}$ si el robot $i$ ya se encontraba en $s$\\
%             }
%     }
%     Asignar robots a los segmentos usando el método húngaro\\
%     \For{cada segmento $s \in S$}{
%         Asignar robots a las fronteras objetivas en $s$ utilizando el método húngaro\\
%     }
%     \caption{Asignación de objetivos}
%     \label{alg:asignacionobjetivos}
    
% \end{algorithm}


% hasta acá redactado


\subsection[Incremental contour-based topological segmentation for robot exploration]{Incremental contour-based topological\\ segmentation for robot exploration}
\subsubsection{Resumen}
Habla de mapas topológicos, estos son los mapas que se generan a partir de la segmentación del entorno. 
Específicamente se centra en describir y evaluar un algoritmo de segmentación para la contracción de mapas topológicos en 2D: Segmentación topológica basada en el contorno de los obstáculos del mapa (Contour Based Topological Segmentation).
También se describe una variante incremental para poder usarse en tiempo real para la exploración multi-robot.

\subsubsection{Notas generales}
Interesante tener en cuenta la existencia de diferentes métodos para segmentar el espacio.

Este se evalúa contra la segmentación humana.

Se describe una variante incremental para poder usarse en tiempo real para la exploración multi-robot.

En la parte de trabajos relacionados habla de la segmentación basada en GVD. <- importante

Interesante el trabajo relacionado por describir los tipos de segmentación

Funcionamiento:
\begin{enumerate}
  \item De grid 2D based map a un conjunto de polígonos:
  \begin{enumerate}
     \item De grid a imagen binaria
     \item A partir de esa imagen usando el modo árbol de la función de la biblioteca OpenCV $findCountours$ que devuelve hacer jerarquía de contornos según que contornos (padres) contienen a otros (hijos)
  \end{enumerate}
\item Luego se usa la función DuDe\_segment toma los polígonos del paso anterior para obtener la descomposición del espacio en segmentos. Esto se hace para cada polígono encontrado en 1. Y el resultado final es la union de estas segmentaciones. La descomposición DuDe se explica en: 

  \url{http://masc.cs.gmu.edu/wiki/Dude2D}
\end{enumerate}

Obtienen buenos resultados, la version incremental permite tiempo real, solo tiene un parámetro a tunear (maxima convexidad, usado en el DuDe\_segment), flexible para entornos estructurados y no estructurados. Y es independiente del tamaño de la grid, solo importan las proporciones del contorno.

\subsection[Coordinated Multi-Robot Exploration: Out of the Box Packages for ROS]{Coordinated Multi-Robot Exploration: Out of the\\ Box Packages for ROS}
\subsubsection{Resumen}
Se presentan y evalúan paquetes de ROS para la exploración multi-robot coordinada. Estos paquetes tienen el objetivo de ofrecer funcionalidades básicas para tener una solución completa pero simple para el problema de la exploración multi-robot permitiendo una configuración completamente distribuida, apuntando a que grupos de investigación puedan utilizarlos. Los paquetes son:
\begin{itemize}
  \item ad hoc communication between robots,
  \item construction of global maps from local maps
  \item exploration of unknown environments
\end{itemize}

\subsubsection{Notas}
\paragraph{WIRELESS AD HOC COMMUNICATION}

ROS permite comunicación ínter proceso local a través de un maestro usando el patron de comunicación duplicación/subscripción. El maestro se encarga de manejar los publicadores y los subscriptores a  tópicos de ROS ( canales de comunicación entre procesos ). Para hacer un sistema multi-robot en este caso seria necesario que todos los robots se conecten inalámbricamente a un solo maestro y solo este maestro es el encargado de establecer canales de comunicación entre procesos. Esto significa que dos procesos deben comunicarse con el maestro para poder luego comunicarse entre si y en el caso de desconectarse deben recaer nuevamente en el maestro para lograr una re conexión. Esto es malo ya que el maestro es un punto único de fallo y porque dos procesos que corren localmente en un robot deben comunicarse con el maestro para poder comunicarse entre ellos.

La solución presentada por el articulo es tener un maestro por robot para manejar la comunicación ínter proceso local y manejar la comunicación global entre robots con un paquete propuesto por ellos. 

Su paquete se basa en generar una red Ad-Hoc usando un protocoló similar a AODV (protocolo conocido para generar redes ad-hoc). Sus features son:
\begin{itemize}
\item Routing: allows robots to communicate via multiple hops to other robots which are not in the immediate neighbor-hood.
\item Multicast: allows to transmit from one robot to multiple robots, especially useful for dissemination of map data, for example.
\item ARQ: stands for automatic repeat request. If a frame is not acknowledged within a specified time, the sender will automatically repeat the transmission. The Ad Hoc Communication package supports both hop-by-hop ARQ and end-to-end ARQ.
\item Segmentation: allows to split data packets into multiple frames of smaller size. The IEEE 802.11a/$g$/$n$ MAC layer supports payloads up to 2304 octets [12]. At the receiver side the frames are ordered and combined.
\item Ordering: ensures that frames and packets are delivered in the correct order.
\end{itemize}

El uso de este paquete es:
Cuando un robot A se quiere comunicar con un robot B no lo hacen de forma directa si no que utilizan el un service call del paquete que debe incluir:
\begin{itemize}
  \item destino, datos, tópico en el cual publicar el dato al llegar al destino
\end{itemize}

Esto causa que el paquete envié el dato junto a los metadatos necesarios en un frame MAC a través de un raw socket hasta B

B al recibir el frame el paquete lo interpreta y lo publica en el tópico indicado en los metadatos.

De esta manera se conserva parcialmente el manejo de tópicos de ros. Hay transparencia en la subscripción pero no me queda claro si hay transparencia en la publicación (es la service call implícita al publicar en un tópico)

\paragraph{MAP MERGER}

El map merger es el encargado de recolectar mapas locales de los robots y combinarlos en un solo mapa global. El mapa global es utilizado por los robots para navegar, explorar y coordinar.

El Map merger puede ser centralizado o distribuido:
\begin{itemize}
\item si es centralizado los mapas locales deben enviarse a la central, ser combinados ahí y luego distribuirse el mapa global resultante a los robots
\item si no es centralizado los mapas locales deben enviarse a todos los robots para que cada uno de estos lo combine y obtenga su propio mapa global
\end{itemize}

Este modulo esta inspirado en $map\_stich$ pero extiende y mejora varios aspectos.

Proceso de combinado
\begin{itemize}
  \item El paquete combina dos mapas $M_1$ y $M_2$ en un solo mapa global juntando dichos mapas de forma de maximizar la areas que se solapan.

  \item La transformación entre los sistemas de coordenadas que debe hacerse para juntar los mapas se calcula con OpenCV:
  \begin{itemize}
    \item Se convierten las grillas de ocupación en bitmaps
    \item Se utiliza la función de OpenCV $estimateRigitTrasnform$ que intenta emparejar patrones entre los mapas.
  \end{itemize}
\end{itemize}

Al basarse en solapamientos se requiere asumir que los robots van a empezar en la misma posición (por ejemplo una entrada) para que los mapas locales solapen desde un principio.

Features:
\begin{itemize}
\item Map updates are triggered if changes in local maps are detected.
\item New robots are added and maps are exchanged automatically if the Ad Hoc Communication package reports a new robot in the system.
\item Robot positions are transmitted in regular intervals. The package automatically converts other robots’ positions to the correct coordinate system.
\item Integration Ad Hoc Communication package allows the wireless exchange of maps between robots right out of the box. All topics are preconfigured.
\end{itemize}

\paragraph{EXPLORATION}

Este paquete:
\begin{enumerate}
\item Identifica fronteras basado en el mapa (global) actual.
  \begin{itemize}
     \item No se explica como, seguramente est en la referencia que mencionan (5)
  \end{itemize}
\item Selecciona fronteras a ser exploradas ( la exploración termina si no hay mas fronteras para explorar ).
\
  \begin{itemize}
     \item Las fronteras objetivo se priorizan según la distancia euclídea entre la pos del robot y la frontera 
     \item Las fronteras se agrupan
     \item Menciona que usar un camino real seria un drawback por consumir mas recursos y que es propenso errores con los path planners de ros actuales?
  \end{itemize}
\item Los robots deben coordinarse para reducir el tiempo de exploración.
  \begin{itemize}
     \item se hace una subasta para determinar que robot sigue que objetivo a través del método húngaro (igual que Wurm)
  \end{itemize}
\end{enumerate}

Features:
\begin{itemize}
\item Integration Ad Hoc Communication package and
\item Integration Map Merger package allow out of the box deployment. All topics and settings are pre-configured.
\item Coordination exploration including frontier identification and coordinated assignment as described above.
\item Bid interpolation aims to interpolates bids of other robots which did not send their bids for an auctioned cluster.
\end{itemize}

\subsubsection{Definiciones}
Ad-hoc network: An ad hoc network is one that is spontaneously formed when devices connect and communicate with each other. The term ad hoc is a Latin word that literally means "for this," implying improvised or impromptu. Ad hoc networks are mostly wireless local area networks (LANs).The devices communicate with each other directly instead of relying on a base station or access points as in wireless LANs for data transfer co-ordination. Each device participates in routing activity, by determining the route using the routing algorithm and forwarding data to other devices via this route. 


\subsubsection{Ideas}
Los paquetes presentados pueden ser útiles:
\begin{itemize}
\item WIRELESS AD HOC COMMUNICATION: es una solución al funcionamiento centralizado del sistema de tópicos de ROS.
\item MAP MERGER se presenta como una solución a la combinación de mapas cuyo código puede llegar a resultar útil dependiendo de que tan bien se pueda adaptar la solución actual
\item EXPLORATION resulta simple en comparación a las funcionalidades que provee nuestro paquete pero de igual, resaltando el uso de el método húngaro para la resolución de subastas el cual seria interesante considerar para nuestro paquete (comparar ordenes del método actual).
\end{itemize}

El código no se actualiza hace 5 años y no hay releases para versiones recientes de ros, de igual manera es posible extraer código para reutilizar en nuestro proyecto.

\subsection[Distributed matroid constrained submodular maximization for multi-robot exploration: theory and practice]{Distributed matroid constrained submodular maximization for multi-robot exploration: theory and\\ practice}
\subsubsection{Resumen}
Este articulo describe el problema de exploración multi-robot y se centra en la descripción de "distributed sequential greedy assignment (DSGA)", un algoritmo que resuelve de forma eficiente la asignación de objetivos de exploración de forma distribuida.

\subsubsection{Notas}
Informative planning problems of this form are known to be NP-Hard 

\url{https://www.jmlr.org/papers/volume9/krause08a/krause08a.pdf}

Entonces rather than attempt to find an optimal solution in possibly exponential time, we seek approximate solutions with bounded suboptimality that can be found efficiently in practice.

Para probar la eficiencia hacen un modelo que considera la incertidumbre del entorno y la modela, siendo el objetivo de la exploración reducir la entropía del mapa (concepto que cuantifica la incertidumbre).

Se tratan conceptos como la entropía, las funciones submodulares.

\url{https://www.youtube.com/watch?v=yEnYXCAj4WY}

Y matroides

\url{https://www.youtube.com/watch?v=XcSzR_tpHYE}

Me costo bastante entender las partes iniciales, pero logre entender por arriba, parece que la complejidad crece en las siguientes partes.

\subsection{Communication - Efficient Planning and Mapping for Multi - Robot Exploration in Large Environments}
\subsubsection{Resumen}
El aspecto principal es la utilización de una representación alternativa a las grillas de ocupación llamada "Gaussian mixture model (GMM)". Dicha representación es mas eficiente en tamaño que las grillas de ocupación, por lo que se usan para los mapas globales y para que los robots compartan la información global del mapa. Por otro lado cada robot usa el GMM para mantener un mapa local denso (grilla de ocupación) para usar en el planning. 

\subsubsection{Notas}

The perception system uses a Gaussian mixture model (GMM) that accurately represents detailed geometry and efficiently captures empty volumes and surfaces. 

The resulting GMM has a small memory footprint and communicating updates across a network of robots requires relatively little bandwidth compared to the volume of novel data.

Samples from the GMM are, in turn, used to maintain a dense local map for use in planning.

A receding horizon planner maximizes information gain over sequences of camera views, and a terminal cost based on distances to highly informative views provides global spatial reasoning and ensures complete exploration.

Robots maintain a library of such views by sampling and updating views locally and share updates with the rest of the team. 

Para planificar se clasifican vistas ("over sequences of camera views") según su ganancia de información, estas vistas son compartidas entre los robots para tener un modelo global. Las vistas consideradas son las "vistas informativas" (son significativamente informativas).
Los robots se mueven a las fronteras o vistas basados en criterios como la distancia y la información. Aunque también toma encenta en la secuencia de observaciones del trayecto.
Se define un costo terminal basado en el camino mas corto a una vista informativa para mantener una compatibilidad con el mapa local que los robot extraen del GMM para el planning.

El trabajo no aplica coordinación entre los robots.

Seria interesante entender como funciona la representación alternativa

"Gaussian mixture model (GMM)" y como esta puede ser utilizada para tener comunicaciones eficientes.

\subsection[Learning to Cooperate via an Attention-Based Communication Neural Network in Decentralized Multi-Robot Exploration]{Learning to Cooperate via an Attention-Based\\ Communication Neural Network in Decentralized Multi-Robot Exploration}

\subsubsection{Resumen}
En el marco de la exploración multi robot, este trabajo se centra en la parte de la cooperación/coordinación. 

Según los autores en muchos entornos del mundo-real, en especial los altamente dinámicos, son muy complejos para que los humanos diseñen estrategias eficientes y descentralizadas.

Dicho esto, presentan un método de coordinación basado en redes neuronales con mecanismos de atención. Le llaman Attention-based Communication neural network ($CommAttn$). Esta les permite a los robot aprender estrategias de cooperación a través de la comunicación explicita. El mecanismo de atención se introduce para que los robots puedan determinar con que otros robots es necesario comunicarse.

\subsubsection{Notas}
Interesante pero no seria el foco del proyecto


\chapter{Parte Central}\label{cha:central}
\hfuzz=10pt 
\minitoc
\hfuzz=0pt 

\section{Solución desarrollada}
Durante el proyecto se desarrollo una solución al problema de exploración
multirobot. Esto requiere la soluciones para una gran cantidad de problemas
relacionados a diversos aspectos de la exploración multirobot, este proyecto se
concentrara en algunos de ellos.

% se a partir de la información sensorial de los
El problema construir un mapa de forma cooperativa robots se soluciona construyendo una grilla de ocupación con
el algoritmo de actualización presentado en \cite{stachniss2009robotic}.

% Se presenta una solución para las dos partes asociadas al

Con respecto al problema de asignación de tareas (sección
\ref{sec:exploracion}) se presentan soluciones para cada una de sus partes.
Para la identificación de objetivos se propone una técnica novedosa que
determina \todo{FEDE: estaria bueno aca poner un adjetivo que resuma como es el
algorimo que desarrolle. De forma: determinista, geometrica, otro?} como
objetivos a un subconjunto de los puntos fronterizos. Con respecto a la
asignación de objetivos a robots, esta se resuelve de forma coordinada
aplicando una variante de lo presentado en la sección \ref{subsec:wurmCoord}.

La asignación requiere de la construcción de un mapa topológico, para esto se
implementa la técnica descrita en la sección \ref{subsec:mapaTopGVD},
construyendo el GVD de forma incremental con una variante del algoritmo
\emph{brushfire dinámico} (sección \ref{subsec:constGVDInc}).

Luego de asignado a un objetivo el robot deberá llegar hasta el, para esto es
necesario solucionar el este problema de planificación. La solución desarrollada 
aplica las ideas de planificación jerárquica presentadas en la sección
\ref{subsec:mapas}, con el motivo de permitir una planificación eficiente sin
resultar en caminos innecesariamente largos.


\section{Principales aportes}

Propuesta y estudio de un algoritmo novedoso para identificación de objetivos basado en la simplificacion de fronteras.% que no se basa en un algoritmo de clasificación no supervisada (p.e. K-Means).

Estudiar la coordinación que se propone en \cite{wurm2008coordinated} utilizando una construcción incremental del GVD.

Propuesta y estudio de un algoritmo de construcción incremental de un GVD basado en \emph{brushfire dinámico} según se describe en \cite{Lau2013} cambiando principalmente la forma de determinar las celdas pertenecientes al GVD.

Adicionalmente se estudia un aspecto critico que no esta explicito en el estado del arte\todo{FEDE: Kalra se ve que el mapa para entorno abierto es consistente con esta tecnica, en el codigo de lau tambien. Pero no se hace referancia a esto}, el problema asociado a como se tratan las celdas de estado desconocido en la construcción del GVD. Para este problema se comentan posibles alternativas, pasando por la que esta presente en el estado del arte y una técnica novedosa que permite una mayor eficiencia.

% Proponer un algoritmo novedoso para la simplificacion de conjuntos de fronteras que ... (no se que poner aca, pongo)
% Estudiar la coordinacion propuesta une wurm utilizando un  GVD incremental ya que de lo contrario no es escalable.
% Y que adicionalemtente se implemento una construiccion de GVD novedosa que basa su construiccion en sources y pseudosources (la idea intuitiva pero considerando los potenciales problemas de discretizacion)
% Llamado de atencion al problema de conectividad de un GVD al explorar donde en un mapa no solo hay obtaculos y celdas libres si no que tambien hay celdas desconocidas, las cuales no se consideran en los articulos del estado del arte. Estos se asume (no se explicita pero en Kalra se ve que el mapa para entorno abierto es consistente con esta tecnica, en el codigo de lau tambien) que implementan la tecnica de "los limites de mi mapa son obstaculos" y lo desconocido es libre. En este trabajo se propone una tecnica novedosa que soluciona este problema de forma mas eficiente evitando procesar todo el espacio desconocido mientras se va explorando.

\section{Hipótesis de trabajo}
La solución desarrollada tomara las siguientes hipótesis de trabajo.
\begin{enumerate}[label=(\roman*)]
  \item Cada robot conoce en todo momento su ubicación (posición y orientación).
  \item La comunicaciones son sin perdida y de rango infinito.
  \item El entorno a explorar es cerrado.
  \item Los robots son circulares e iguales entre sí.
  % \item El entorno es estático.
\end{enumerate}

Estas hipótesis tienen el motivo de simplificar algunos aspectos del
problema de exploración permitiendo que el trabajo se pueda enfocar en otros.

La hipótesis I resuelve trivialmente el problema de localización. La hipótesis
II simplifica la coordinación al no tener que considerarse en esta los posibles
problemas de comunicación que pueden haber de asignar robots a objetivos muy
distantes o con obstrucciones entre si. La hipótesis III es útil
principalmente para poder aplicar el criterio de parada que consiste en detener
la exploración al no tener más objetivos restantes (sección
\ref{sec:exploracion}). 

\section{Arquitectura}
La arquitectura se divide en dos partes robots y unidad central.  Cada una de
estas partes esta asociado a un hardware especifico, siendo los robots móviles
y la estación central estática. Los robots pueden ser uno o más mientras que la
estación central es única. Cada parte se compone de módulos los cuales cuales se
encargan de tareas específicas. 

En la figura \ref{fig:arquitectura} se resume la arquitectura en un diagrama en
el cual es posible ver los componentes, sus módulos dichos módulos, como se
distribuyen entre los componentes y una simplificación de las comunicaciones
que ocurren entre ellos.


\begin{figure}[H]
  \center
  \includegraphics[width=1\linewidth]{imagenes/arquitectura.png}
  \caption[Arquitectura de la solucion propuesta.]{Arquitectura de la solución propuesta. El modulo coloreado con azul es provisto por ROS. Los módulos no coloreados fueron  implementados en esta propuesta.}
  \label{fig:arquitectura}
\end{figure} 

En lo que resta de esta sección se comentará cada modulo, tanto sus tareas y como sus interacciones con el resto de los módulos.

\subsection{Move Base}
El modulo \emph{Move Base} consiste en una instancia del nodo ROS
\emph{move\_base} \cite{ROS-move_base} que provee una interfaz para configurar,
ejecutar y interactuar con el \emph{stack de navegación} de ROS
\cite{ROS-navigation}. 

El stack de navegación de ROS es un conjunto de nodos que tienen como propósito
que un robot pueda navegar el entorno hacia objetivos dentro del mismo. En la
figura \ref{fig:move_base} se muestra un diagrama que repesenta a los nodos que
componen al stack de navegación de ROS y sus interacciones.

\begin{figure}[H]
  \center
  \includegraphics[width=1\linewidth]{imagenes/move_base.png}
  \caption[Arquitectura del stack de navegación de ROS.]{Arquitectura del stack de navegación de ROS. Extraída de \cite{ROS-move_base}.}
  \label{fig:move_base}
\end{figure} 

Cuando se establece un objetivo de navegación este se trasmite al nodo
\emph{global\_planner} este se encarga de generar un plan de alto nivel
consistente de un numero de subobjetivos, que de seguirse en secuencia llevan
al robot al objetivo sin colisiones.

El camino generado por el \emph{global\_planner} es enviado al nodo
\emph{local\_planner} que se encarga de tomar el plan en alto nivel y
traducirlo a la secuencia las velocidades lineales y angulares que un robot
debe tener a lo largo del tiempo para seguir el global. A dicha secuencia de
velocidades se le conoce como plan local.

El stack de navegación permite utilizar distintas implementaciones de
\emph{global\_planner} y \emph{local\_planner}. En el trabajo desarrollado se
hace uso del \emph{global\_planner}\cite{ROS-global_planner} valga la
redundancia llamado \emph{global\_planner} y el \emph{local\_planner} llamado
\emph{teb\_local\_planner}\cite{ROS-teb_local_planner}.

Para generar sus planes tanto \emph{local\_planner} como \emph{global\_planner}
requieren de un mapa, de esto se encargan los nodos \emph{local\_costmap} (mapa
local) y \emph{global\_costmap} (mapa global) respectivamente. Ambos dos son
una instancia de una misma clase de nodo llamada \emph{costmap\_2d}
\cite{ROS-costmap_2d}, que dentro de sus funcionalidades esta la de de
construir una grilla de ocupación a partir de los datos sensoriales provistos
por los robots. Las principales diferencias entre el mapa global y el local son su
tamaño de celda (pequeño en el local y grande en el global), sus dimensiones
del mapa (el mapa global es el mapa completo mientras que el local es solo una
porción) y sus marcos de referencia (el mapa global suele estar fijo, el local se
centra en el robot).

El nodo \emph{recovery\_behaviors} permite ejecutar comportamientos de
recuperación de detectarse que el robot no esta avanzando de forma correcta al
objetivo. Para la solucion desarrollada solo se hace uso del comportamientos de
recuperación que consite en que el robot rote en el lugar.
%en el lugar y forzar que recalculen porciones del mapa.

% El \emph{global\_planner} hace uso de un mapa global, este mapa es provisto por 


% lograr esto cuando se establece un objetivo de navegación,


% de permitir que un robot pueda mover que a partir de información sobre la ubicación, sensores del robot, v 


\subsection{Combinador de mapas}
El modulo \emph{Combinador de mapas} es el encargado de mantener el mapa del entorno
explorado. Este recibe las actualizaciónes de los mapas globales que son
generados por el nodo \emph{global\_costmap} del stack de navegacion de cada
robot y las combina en un unico mapa que contenga toda la información
recopilada del entorno. Cuando el mapa combinado global se actualiza este
retrasmite lo actualizado a diversos componentes del sistema (ver figura
\ref{fig:arquitectura}) que utlizan el mapa del entorno explorado para llevar a
cabo alguna de sus tareas. 

\subsection{Controlador central}

El modulo \emph{Controlador central} lleva a cabo la identificacion de objetivos, es decir detectar en
el mapa actual los potenciales objetivos de exploracion. 

A su vez es el principal responsable de realizar la asignacion de objetivos.
Específicamente la asignacion de objetivos consiste en una subasta en la cual
este modulo actua como subastador. La subasta se puede resumir de la siguiente
manera, los objetivos de exploracion identificados son transmitidos desde el
controlador central a los robots los cuales valuan a dichos objetivos segun que
tan conveniente les es llegar a ellos. Los robots envian sus valuaciones a la
central la cual aplica una algorimo para determinar que objetivo le corresponde
a cada robot y posteriormente le informa a cada robot que objetivo le
corresponde.

% tiene el
% siguiente funciona en la cual  de objetivos consite en tomar los objetivos
% identificados y coordinar

% La prinicpal es la de llevar a cabo la identificación de tareas y ser el 

% e encarga de coordinar las subastas de segmentos, es decir, genera la
% información necesaria para esta, decide cuando comienza la subasta y cuando
% termina el plazo para ofertar, computa los resultados y se encarga de las
% comunicaciones necesarias para llevarla a cabo.

\subsection{Controlador de movimiento}
El modulo \emph{Contrador de movimiento} es como su nombre lo indica el modulo
que se encarga de controlar el movimiento del robot. Específicamente recibe
caminos compuesto por celdas de la grilla de ocupacion que en secuencia llevan
a un objetivo de exploracion, y se encarga de ir enviando objetivos de
navegacion al modulo \emph{Move Base} para que el camino sea ejecutado de forma
rapida evitando maniobras innecesarias.

Tambien lleva a cabo una capa superior de comportamientos de recuperacion
extendiendo los provistos por el modulo \emph{Move Base}.

Indica al \emph{Controlador del robot} si se completo el camino con exito, o
existe algun problema.

\subsection{Controlador del robot}
El \emph{Controlador del robot} es el modulo que se encarga de valuar los
objetivos cuando ocurre una subasta. A su vez se encarga de procesar las
asignaciones de objetivos determinadno el camino que lleva al objetivo y
enviandolo al modulo \emph{Controlador de movimiento}. 

Tambien es responsable por solicitar el inicio de una subasta al
\emph{Controlado central} cuando el \emph{Controlador de moviemiento} indica el
exito o el fracaso en seguir el camino asignado.

% Tambien pide la subasta.\todo{explicar mejor}

% \section{Ciclo de robot}

% \section{Ciclo de la central}
\section{Definiciones}
\subsection{Grillas de ocupacion}
En el contexto de este trabajo se utlizara las siguientes definiciones referentes a
grillas de ocupacion, introducidas en la seccion \ref{subsec:mapas}.

El conjunto $C\subseteq R^2$ esta conformado de los centros de cada celda de la
grilla de ocupacion. Las celdas se repreresentan segun sus centros y viceversa
sin ambigüedad, por lo tanto en lo que resta de este informe se usaran ambos
terminos de forma indistinta.

Se dice que cada celda $c\in C$ tiene asociada una probabilidad $P(c|m(1:k))$
de estar ocupada, donde $m(1:k)$ es el conjunto de medidas resultantes de algun
sensor desde el comienzo de la exploarcion.

La funcion $e : C \rightarrow E$ dado un centro de celda, devuelve uno de los
tres estados posibles $E=\{libre, ocupado, desconocido\}$ segun la probabilidad
asociada a $c$. En el contexto de este proyecto la funcion $e$ se define segun
(\ref{eq:estado}).
\begin{equation} 
  e(c)= 
  \left \{ 
    \begin{aligned}
       libre       &\ \ \ \text{ si}& P(c|m(1:k)) < 0.5 \\
       desconocido &\ \ \ \text{ si}& P(c|m(1:k)) = 0.5 \\
       ocupado     &\ \ \ \text{ si}& P(c|m(1:k)) > 0.5
    \end{aligned}
  \right .
  \label{eq:estado}
\end{equation}

La funcion $ady : C \rightarrow P(C)$ dada una celda devuelve el conjunto de
celdas adyacentes a la misma. La definicion de $ady$ utlizada en el contexto de
este proyecto se presenta en (\ref{eq:vecinos}) donde $n_1, n_2, ..., n_8$ se
corresponden a vecinos diagonales y horizontales de $c$ segun se muestran en la
figura \ref{fig:vecinos}.

\begin{equation} 
 ady(c)=\{n_i : 1\leq i \leq 8, n_i \in C\}
 \label{eq:vecinos}
\end{equation} 

\begin{figure}[H]
  \center
  \includegraphics[width=0.3\linewidth]{imagenes/vecinosSharp.png}
  \caption[Vecinos de una celda en una grilla de ocupación.]{Vecinos de una celda en una grilla de ocupación.}
  \label{fig:vecinos}
\end{figure} 

Notar que la relacion de adyacencia es simetrica por lo que $c_1 \in ady(c_2) \Rightarrow c_2 \in ady(c_1)$.
% This map is obtained from the occupancy probability grid by a simple clipping operation with a threshold of 0.5. The gray areas of the maximum-likelihood map correspond to cells that have not been sensed by the robot.

\subsection{Componentes conexas} \label{subsec:CompComp}
Una descomposicion en componentes conexas de un conjunto de
celdas $C$ es un conjunto $CC\in P(C)$ compuesto por $N$ conjuntos $C_i$ con
$i\in[1,N]$ tales que:
\begin{itemize}
  \item $\bigcup_{i=1}^{N}C_i = C$ 
  \item Para todo $i,j \in [1,N]$ $C_i\cap C_j = \emptyset$
  \item Para todo $i \in [1,N]$, para todo par $c_1,c_2 \in C_i$ se cumple que $c_1 \in ady(c_2)$.
  \item No existen $i,j \in [1,N]$ tales que existan $c_1 \in C_i$ y $c_2 \in C_j$ que cumplan con $c_1 \in ady(c_2)$ 
\end{itemize}

Un ejemplo de una descomposicion en componentes conexas se muestra en la figura \ref{fig:fronterasCompCon}.

\begin{figure}[H]
  \centering
  \subfloat[Las celdas pertenecientes a $C$ se marcan con azul.]{\includegraphics[clip=true, width=0.40\linewidth]{imagenes/compCon/a.png}}
  \qquad
  \subfloat[Cada componente conexa de $C$ se contornea con rojo.]{\includegraphics[clip=true, width=0.40\linewidth]{imagenes/compCon/b2.png}}

  \caption{Descompocicion en componentes conexas.}\label{fig:descCompCon}
\end{figure}

Es posible obtener las compnentes conexas de un conjunto cualquiera $C$ de
celdas con el algoritmo \ref{alg:compcon}

\begin{algorithm}[H]
\SetAlgoLined
  $CC := \emptyset$
  $pila :=$ Pila Vacia \\
  $i := 1$ \\
  $restantes := C$ \\
  \While{ $\neg restantes.vacia()$ } {
    $C_i := \emptyset $ \\
    $c :=$ elemento arbitrario de $Restantes$ \\

    % \tcp{DFS desde $c$ agregando las celdas visitadas a la componente conexa $C_i$}

    $C_i :=  C_i \cup \{c\}$ \\
    $restantes := restantes - \{c\}$ \\
    $pila.apilar(c)$ \\
    \While { $\neg pila.vacia()$ } {
      $c := pila.desapilar()$ \\
      \For{ $cA \in ady(c)$ } {
        \If{ $cA \in restantes$ } {
          $C_i :=  C_i \cup \{c\}$ \\
          $restantes := restantes - \{c\}$ \\
          $pila.apilar(c)$ \\
        }
      }
    }
    $CC := CC \cup C_i$ \\
    $i := i + 1$ \\
  }
  \Return $CC$ 

  \caption{Asignación de objetivos}
  \label{alg:compcon}
\end{algorithm}

Este algoritmo se resume en elegir una celda $c\in C$ que no este aun en una
componente conexa (linea 6), aplicar un \emph{depth-first search} (DFS)
partiendo $c$ agregado todas las celdas recorridas a una misma componente
conexa (lineas 7-19). Repetir dicho procedimiento hasta que todas las celdas
pertenezcan a alguna componente conexa (linea 4). Este algormitmo es analogo al
que esta presente en \cite{hopcroft1973algorithm}.

\section{Identificación de objetivos}
El problema de identificación de objetivos consiste en determinar los puntos
del espacio a los cuales es conveniente enviar robots para recolectar nueva
informacion sobre el entorno explorado. Estos puntos son los llamdos objetivos
de exploración. 
% En el contexto del trabajo desarrollado al utlizarse una grilla de ocupacion
% como mapa los 
% se hablara de celdas en lugar de puntos, existiendo una correspondencia entre
% cada celda y un unico punto en el espacio, su centro.

\subsection{Fronteras}
En \cite{yamauchi1998frontier} se propone que los lugares que permiten
recolectar la mayor cantidad de nueva informacion sobre el entorno son las
fronteras entre el espacio conocido y desconocido. Y que por lo tanto dichas
fornteras deben ser los objetivos de exploración.
Al utlizar una grilla de ocupacion como mapa, las fronteras se definen como las
celdas cuyo estado asociado es $libre$ y son adyacentes a una celda cuyo estado
asociado es $desconocido$ (figura \ref{fig:fronteras}).
Por lo tanto segun Yamauchi los objetivos de exploración seran las celdas
fronteras $F$ segun se definen en (\ref{eq:fronteras}).
\begin{equation} 
  F = \{ c \in C : e(c) = libre, \exists n \in ady(c), e(n) = desconocido  \}
  \label{eq:fronteras}
\end{equation}

\subsection{Fronteras simplificadas según K-Means}
En \cite{amorin2019novel} se argumenta que tratar todas las celdas fronteras
como objetivos de exploración diferentes podría ser computacionalmente
prohibitivo. Por lo tanto, para reducir el costo computacional, se intenta
reducir los objetivos de exploración a las celdas frontera mas representativas,
a las cuales se denominaran como fronteras significativas.

Para determinar las fronteras significativas, primero, las celdas fronteras $F$
se descomponen en sus componentes conexas $\mli{FC}=\{F_1,F_2,...F_N\}$
(seccion \ref{subsec:CompComp}), un ejemplo de este tipo de descomposicion se
puede ver en la figura \ref{fig:fronterasCompCon}.

Luego se determinan las fronteras significativas $\mli{FS}_i$ de cada
componente conexa $F_i\in \mli{FC}$. Esto se hace agrupando las fronteras de
$F_i$ con el algoritmo K-Means \cite{macqueen1967some}, y determinando una
frontera significativa por cada una de las $k$ agrupaciones, la
frontera mas cercana de $F_i$ al centroide de la agrupacion (una arbitraria de
las mas cercanas en el caso de que exista mas de una). Un ejemplo de las
fronteras significativas  $\mli{FS} = \bigcup_{i=1}^N \mli{FS_i}$ obtenidas con
este metodo se muestra en la figura \ref{fig:fronterasSig}.

El $k$ utilizado para ejecutar K-Means es el minimo que logra que para toda
frontera $f\in F_i$ existe $\mli{fs} \in FS_i$ tal que $d_{\mli{fs}}(f) <
rango$ siendo $rango$ el rango de los sensores del robot. Parafraseando, el
conjunto de fronteras significativas $\mli{FS}_i \subset F_i$ cumple con que
cada frontera esta dentro del rango del sensado de alguna frontera
significativa, cuando esto se cumple se dice que $\mli{FS}_i$ cubre a $F_i$, o
que $FS_i$ logra el cubrimiento. En la figura \ref{fig:fronterasSigCub} se
puede ver como las fronteras significativas obtenidas $FS$ logran el
cubrimiento, ya que todas los centros de las fronteras $F$ estan contenidos en
las circunferencias de radio $rango$ centradas en las frontera significativa
$FS$.

Para encontrar el minimo $k$ con las que se logra un $\mli{FS}_i$ que cubra a
$F_i$, se parte con $k=1$, si el resultado no logra cubrir incrementa $k$ y se
repite el proceso.


\begin{figure}[H]
  \centering
  \subfloat[Se identifican las frotneras,  marcadas con amarillo.]{\includegraphics[clip=true, width=0.40\linewidth]{imagenes/fronterasSig/a.png}\label{fig:fronteras}}
  \qquad
  \subfloat[Descompocicion de las fronteras en componentes conexas, cada componente conexa se indica con un color distinto.]{\includegraphics[clip=true, width=0.40\linewidth]{imagenes/fronterasSig/b.png}\label{fig:fronterasCompCon}}
  \qquad
  \subfloat[Frontera significativas (indicadas con verde) de cada componente conexa.]{\includegraphics[clip=true, width=0.40\linewidth]{imagenes/fronterasSig/c.png}\label{fig:fronterasSig}}
  \qquad
  \subfloat[Se logra el cubrimiento con un rango igual a 4 largos de celda.]{\includegraphics[clip=true, width=0.40\linewidth]{imagenes/fronterasSig/d.png}\label{fig:fronterasSigCub}}

  \caption[Proceso de simplificacion de froteras según K-Means.]{Proceso de simplificacion de froteras según K-Means. Cada figura corresponde a una etapa distinta para un mismo entorno parcialmente explorado, representado con una grilla, donde las celdas blancas son libres, las negras ocupadas, y para las grises se desconoce su estado. Basada en figuras de \cite{Amorin2019}.}\label{fig:ejemploFrontSig}
\end{figure}

% En la seccion anterior se presento un metodo que soluciona el siguiente
% problema. Dado un conjunto de  la descomposicion en componentes conexas de
% $F$, $\mli{FC}=\{F_1,F_2,...F_N\}$ se obteniene como salida el conjunto
% $\mli{FS} = \bigcup_{i=1}^N \mli{FS_i}$ que cumple con la restriccion de que
% para todo $i \in [1,N]$ $\mli{FS}_i$ cubre a $F_i$.

El problema descrito hasta el momento se puede resumir en el de  dado un
conjunto de fornteras $F$, obtener un conjunto de fornteras significativas
$\mli{FS}$ que cumplen con la restriccion de que $\mli{FS}$ cubre a $F$.
Recordando que el proposito de usar $\mli{FS}$ como objetivos de exploracion en
lugar de usar $F$ es reducir los objetivos de exploracion entoces es natural
pensar que la soluciones optimas reducen al minimo el $\mli{FS}$ resultante,
mientras mantienen la restriccion de cubrimiento.

% Analizando la restriccion de cubrimiento, se puede ver que un indicador de que
% una solucion es suboptima es que varias celdas $\mli{FS}$ se concentren,
% solapandose las circunferencias de radio $rango$ centradas en ellas. Por
% ejemplo 

Dado esto es posible ver que en el ejemplo presentado en
\ref{fig:ejemploFSKMMal} el resultado obtenido por el metodo presentado en esta
seccion no es optimo ya que existen fronteras significativas innecesarias para
el cubrimiento. 

\begin{figure}[H]
  \centering
  \subfloat[Fronteras significativas obtenidas segun el metodo basado en K-Means, con $rango = 5.6$.]{\includegraphics[clip=true, width=0.40\linewidth]{imagenes/fronterasigKMMal/caso1/a_sin_circ.png}}
  \qquad
  \subfloat[Dos de las fronteras significativas de la parte inferior derecha de (a) no son necesarias para lograr el cubrimiento.]{\includegraphics[clip=true, width=0.40\linewidth]{imagenes/fronterasigKMMal/caso1/b.png}}

  \caption[Simplificacion de fronteras suboptima resultante del metodo basado en K-Means.]{Simplificacion de froteras suboptima resultante del metodo basado en K-Means. Extraída de implementacion desarrollada\footnotemark.}\label{fig:ejemploFSKMMal}
\end{figure}
\footnotetext{Podria indicar el bag y el seg}

Resultados suboptimos similares se obtienen de forma consistente %al aplicarse la simplficacion 
sobre componentes conexas con forma serpenteante \todo{no se
  si este termino se entiende, me es dificil explicarlo bien, con forma de S?},
  asimetricas y con un largo mayor a $rango*8$ 
\todo[inline]{se entiende la idea del largo aplicada a esto? Creo que puede
quedar claro mirando la foto pero tengo mis dudas. La idea en realidad seria
decir que la solucion no es lo suficientemente simple, por ejemplo si con 1
sola frontera sig ya se cubre todo entonces no va haber problema sin importar
que tan serpenteante o asimetrica sea. Pero se me ocurre que por el largo puede quedar mas claro}
, empeorando (mayor cantidad de fornteras significativas
innecesarias para el cubrimiento) a medida que las componentes conexas de
fronteras son mas largas y las curvas son mas intrincadas.

% (comentarlos, distribuciones desparejas, acumulacion, no
% equidistantes) 

% Una explicacion posible a este comportamiento es que la distibucion de centroides en K-Means 

% Algo que se repite en estas situaciones es que suele haber zonas, en donde las
% fornteras significativas se acumulan, y zonas en las que estan muy dispersas
% (casi a $rango$ de distancia una de la otra, el maximo). Esto lleva a pensar que K-Means se tiende a ubicar concentrar los centroides en una zona,

Esto se presume que se debe principalmente a dos factores relacionados a
K-Means. (i) K-Means no considera la restriccion de cubrimiento para generar sus
resultados. La restriccion se fuerza ejecutando K-Means con el minimo $k$
(obtenido con prueba y error) segun el cual las fronteras significativas
resultantes logran el cubrimiento. (ii) Los centroides resultantes de K-Means
no son fronteras. Estos se deben traducir a fronteras posteriormente.

\todo{Hay una arbitrariedad asociada a k-means tambien lo cual puede ser criticado tambien, aunque deberia estudiar mejor el tema, quizas no criticarlo aca pero destacar que el otro metodo es menos opaco en como elige las front sig}
% Esto es la clase de pe la idea de obtener fornteras significativas es reducir el numero
% de objetivos de exploracion,

% El metodo basado en K-Means aunque funciona bien al aplicase en componentes conexas pequeñas, 

% El metodo descrito en la seccion resulta en un conjunto de fronteras
% significativas que cubren a las fronteras, el problema que se detecto
% experimentalmente es que en ciertos casos el resultado  la discribucion de las
% fronteras significativas es mala, se concentran mucho en ciertas porciones y se alejan 

\subsection{Fronteras simplificadas según su geometría}
Con esto en mente se desarrolla un metodo novedoso que considera los dos puntos
destacados anteriormente. (i) Tomando en cuenta el cubrimiento y su
optimizacion como parte fundamental del algorimo. (ii) Dando directamente
fronteras como resultado.

El algorimo al igual que el descrito en la seccion anterior comienza
descomponiendo a las fronteras $F$ en sus componentes conexas
$\mli{FC}=\{F_1,F_2,...F_N\}$ para luego obtener las fronteras significativas
$\mli{FS}_i$ para cada componente $F_i$, siendo el conjunto total de fronteras
significativas $FS$ la union $\bigcup^N_{i=0} \mli{FS}_i$

Para obtener las fronteras significativas de $F_i$ el algorimo parte mantiene
las fronteras sigficativas determinadas $\mli{FS}_i := \emptyset$ y el conjunto
todas las fronteras que resta cubrir $\mli{UF} := F_i$ (siglas del ingles
\emph{uncovered frontiers}), y a grandes rasgos consiste en:
\begin{enumerate}
  \item Elegir una frontera de $\mli{UF}$ como frontera significativa $\mli{fs}$.

  \item Actualizar $\mli{FS}$ agregando a $\mli{fs}$:

    $\mli{FS} := \mli{FS} \in \{\mli{fs}\}$

  \item Actualizar $\mli{UF}$ removiendo todas las fronteras cubiertas por $\mli{fs}$:

    $\mli{UF} := \mli{UF} - \{ f\in F : d_{\mli{fs}}(f) < rango\}$

  \item Si $\mli{UF} \neq \emptyset$ volver a 1. de lo contrario devolver $FS_i$.
\end{enumerate}

% De los pasos presentados el primero es el principal y tambien el mas complejo
% del algorimo.

Los pasos del 2 al 4 no presentan ambigüedad, siendo el primer paso en el que
resta aclarar, especificamente resta aclarar el criterio con el que se elige la
frontera significativa. La eleccion en un principio podria ser aleatoria y el
algormitmo daria resultados correctos, pero esto dejaria a la suerte la
optimalidad. La idea seria elegir una frontera significativa que cubra la mayor
cantidad de frotneras evitando cubrir nuevametne fronteras que ya sean
cubiertas por otra frotntera significativa.


DESCRIBIR algo a detalle, fotitos, etc.

% EL algorimo se basa en la idea de que una buena distribucion de fronteras
% significativas deberia mantener en su mayor parte una distancia de $rango*2$
% entre cada frontera significativa, ya que esta es la distancia maxima que se
% puede tener para asegurar qu se cumpla el cubrimiento. 


\section{Asignación de objetivos}
\todo{FEDE: Wurm aparentemente no considera que el numero de robots asigandos a un segmento debe ser menor que el numero de fronteras que tiene ya que no tiene sentido asignar a un robot  a un segmento si eseete no tiene objetivos disponibles. Habria que ver bien como funciona el metodo hungaro, pero creo que es una asignacion robot-segmento que no considera otra cosa que el costo, por lo tanto el num de fronteras de un segmento no se estaria considerando}

\subsection{Contruccion del GVD}

\subsection{Segmentacion}


\section{Planificación}

\section{Construcción cooperativa del mapa}


% \section{mejora sobre wurm}






\chapter{Experimentación}\label{cha:exp}
\hfuzz=10pt 
\minitoc
\hfuzz=0pt 

% En este capitulo se presentan y analizan los resultados experimentales de la
% solucion para el problema de al exploracion multirobot coordinada que fue
% desarrollada en este proyecto.

% Las pruebas realizadas se separan en tres secciones segun su proposito. La
% primera estudia el impacto de construir el GVD de forma incremental. La
% segunda compara las diversas tecnicas de identificacion de objetivos. Y la
% tercera analiza la influencia de las diversas formas de considerar el espacio
% desconocido. 


% Las pruebas realizadas tienen  propositos,  para estudiar el impacto de
% construir el GVD de forma incremental. Para comparar las diversas tecnicas de
% identificacion de objetivos. Y finalmente para analizar la influencia de las
% diversas formas de considerar el espacio desconocido. 

% , se presentan los resultados obtenidos y analizan
%  los resultados los cuales son analizados.

% Comentar que estos son cosas generales a todas las pruebas, que en cada una de
% las siguientes secciones se cambian paramtros aislados con el proposito de
% comparar su impacto. (ver capaz que crierios se van a comparar en cada una)

% generales de las pruebas realizadas. Estas 

Este capitulo esta dedicado a describir las pruebas realizadas y analizar los
resultados obtenidos. Las pruebas se ejecutaron de forma simuladas y consisten
en resolver una instacia del problema de la exploracion multirobot coordinada
con distintas soluciones.
Los resultados de las pruebas separan en cuatro secciones segun su proposito.
En la seccion \ref{sec:exp:cubcal} se evaluan los mapas contruidos en las
purebas. En la seccion \ref{sec:exp:inc} se estudia el impacto de construir el
GVD de forma incremental. En la seccion \ref{sec:exp:idobj} se compara las
diversas tecnicas de identificacion de objetivos. Y finalmente, en la seccion
\ref{sec:exp:desco} se analiza la influencia de las formas de considerar el
espacio desconocido. 

\section{Especificación de las pruebas}

A lo largo de esta seccion se especifican las pruebas realizadas en este
proyecto.

% * Pruebas realizadas, cosas generales:
%   * simulador
%   * hardware (de mi pc)
%   * Los robots utilizados
%     * sensor 
%     * robot en si
%   * El entorno de pruebas
%   * Criterios utlizados para analizar los resultados: 
%     * Referencias
%     * Explicar cuales son
%   * Cantidad de pruebas, promedio y varianza

% cuyos resultados se
% presentan a lo largo del capitulo.

% En esta seccion se plantean las caracteristicas generales que comparten todas
% las pruebas que se presentan a lo largo del capitulo.

% En lo que resta de esta seccion se especifican las caracteristicas de las
% pruebas llavadas a cabo. 

% A lo largo del capitulo se presentan diversas pruebas que buscan comparar ciertos
% aspectos de la solucion desarrollada en este proyecto. En esta seccion se
% especifican las caracteristicas generales que comparten todas las pruebas
% realizadas.

% En esta sección se establecen als car


% \subsection{Tipos de prueba}
% El proposito de las pruebas presentadas es evaluar el desempeño de diversos aspectos de una
% solucion al problema de la exploción multirobot. 
% Los tipos de prueba quedan definidos 

% Las pruebas tienen como proposito validar las variantes presentadas para la
% solucion de diviersos aspectos de la del problema de exploracion multirobot.
 
% Para lograr esto se definen pruebas consisten en reslover una instancia de
% dicho problema, cambiando la forma de resolver cada aspecto a validar.

% La instancia del problema de exploracion a resolver es el de explorar un
% entorno cerrado con una flota de robots, hasta que no exista espacio sin
% explorar.

% Debido a esto
% las pruebas se dividen en tres secciones, dependiendo del aspecto que se busca comparar.
% Dado esto las pruebas

% En la seccion \ref{sec:exp:inc} se compara la construccion incremental del GVD contra
% la no incremental, en la seccion \ref{sec:exp:idobj} se
% comparan los metodos de identificación de objetivos que se describen en la seccion
% \ref{sec:pc:idobj}, finalmente en la seccion 
%  la seccion \ref{sec:exp:desco} analiza la
% influencia de las diversas formas de considerar el espacio desconocido. 

% de objetivos que
% utiliza directamente las fronteras como objetivos, con las que se basan en
% simplicar dichas fronteras, tanto la simplificacion basada en K-Means, como la
% basada en cubrimiento. 

% Dado que se quiere comparar la utilidad entre las variantes para ciertas partes de
% la solución, se definen tipos de prueba según las variantes utilizadas. 
% En esta sección se describe el tipo de prueba que se denominara como $base$. Este consiste
% en una solucion del problema de exploracion multirobot que 
 % La solución base consiste en resolver la
% asignación de tareas basada en cubrimiento (seccion \ref{subsec:MiSimp}). La as La construccion de GVD
% incremental como es comentada en \ref{sec:MiConstGVD} con

\subsection{Simulador}
Las pruebas fueron simuladas con Gazebo \cite{gazebo}, el cual fue elegido a
partir del analisis comparativo entre varios simuladores candidatos realizado
en la sección \ref{sec:sim}). 

\subsection{Tipos de pruebas}
Las pruebas tienen como proposito validar ciertos aspectos de la
solucion al problema de exploracion multirobot propuesta en este proyecto.
 
Para lograr esto se define caso de prueba que permite evaluar el desempeño de
una solucion que resueve dicho problema. El caso de pruebas definido consiste
en reslover una instancia de dicho problema, especificamente en explorar un
entorno cerrado con una flota de robots, hasta que no exista espacio sin
explorar. El tipo de mapa generado en la exploración es una grilla de ocupacion
de un tamaño fijo suficiente para representar el entrono explorado.

Para poder validar los aspectos, se prueba la solución propuesta cambiando
unicamente el aspecto a validar. Los aspectos a validar son: la construccion
incremental del GVD (sección \ref{sec:MiConstGVD}). La identificación de
objetivos que simplifica las fronteras basandose en el cubrimiento (seccion
\ref{subsec:MiSimp}). Y la consideracion del espacio desconocido al construir
el GVD presentada en la seccion \ref{subsec:espDesc}, donde las celdas
desconocidas no propagan olas y el conjunto $\mli{UF}$ de celdas desconocidas
que son adyacentes a celdas conocidas pertenecen a generadores ($\mli{UF}
\subseteq \mli{CGen}$).

% Para validar la construccion incremental del GVD esta se compara contra la
% construccion no incremental. En el caso de la identificación de objetivos se
% comparan las tres estrategias propuestas en la seccion \ref{sec:pc:idobj}: el
% identificar las fronteras como objetivos, simplicar dichas fronteras a partir
% K-Means, y simplificarlas basandose en el cubrimiento. Por ultimo, la
% consideracion del espacio desconocido en la contruiccion del GVD se valida
% comparandola contra considerar que el espacio desconocido es libre. 
Para validar la construccion incremental del GVD esta se compara contra la
construccion no incremental. En el caso de la identificación de objetivos esta se
comparan contra identificar todas las fronteras como objetivos, y la tecnica
que obtiene los objetivos simplicando dichas fronteras a partir K-Means. Por
ultimo, la consideracion del espacio desconocido en la contruiccion del GVD se
valida comparandola contra considerar que el espacio desconocido es libre. 

Se evaluan entonces cinco soluciones, la que utiliza todos los aspectos a
validar a la vez, y cada una de las que cambian uno de dichos aspectos.

Con el motivo de probar distintas cargas computacionales, cada prueba se repite
cambiando la granularidad de la grilla de ocupacion utlizada para representar
el entorno explorado. La granularidad se indica a traves de las celdas de una
grilla que se corresponden con un metro cuadrado de la realidad. Los valores de
granularidad utilizados para las pruebas son $1,4,9$ y $16$ celdas
por metro cuadrado ($\frac{celdas}{m^2}$). 

Adicionalente debido a que es posible que ejecuciones de un misma simulacion
lleven a distintos resutlados, con el fin de obtener resultados
estadísticamente significativos cada prueba fue ejecutada 20 veces. Siendo 
$5*4*20=400$ el total de pruebas ejecutadas.
% las tres tecnicas, identfica todas las fornteras como objetivos, 

% Debido a esto
% las pruebas se dividen en tres secciones, dependiendo del aspecto que se busca comparar.
% Dado esto las pruebas

% Para cumplir con la hipotesis (IV) planteada en \ref{sec:hip}
\subsection{Entorno}
Las pruebas se realizan en un entrono cerrado que tiene un area explorable de
aproximadamente $10000m^2$. El entorno esta estructurado en habitaciones,
puertas y corredores de varios tamaños. Los unicos obstaculos presentes en él
son paredes y los mismos robots que pueden obstaculizarse entre
sí. Un mapa de este entrono se muestra en la figura
\ref{fig:willow}.

\begin{figure}[H]
  \center
  \includegraphics[width=0.5\linewidth]{imagenes/willow/0_250000mRobots2.png}
  \caption[Mapa del entrono utilizado en las pruebas.]{Mapa del entrono utilizado en las pruebas. En negro se indican las paredes, en blanco el espacio libre y en gris el espacio inaccesible. Las posiciones iniciales de los robots se indican en rojo.}
  \label{fig:willow}
\end{figure} 

El entorno fue construido a partir de un modelo que se encuentra disponible por
defecto en Gazebo, llamado \say{Willow Garage}, el cual se modifico para
reducir el area a exporar y que sea cerrado (sin salidas al exterior).

\subsection{Robots}
Los robots simulados\footnote{Especificación disponible en línea:
\url{https://gitlab.fing.edu.uy/federico.ciuffardi/pioneer_p3dx_model}} en las
pruebas modelan al robot diferencial Pioneer 3-DX \cite{p3dx} (figura
\ref{fig:p3dx}). Cada robot esta equipado con un sensor LiDAR basado en el
modelo URG-04LX-UG01 \cite{hokuyo}, que permite tomar medidas de distancia de
hasta $5.6m$, utilizando una frecuencia de $10hz$ de un maximo de $36hz$.
Adicionalente el sensor fue alterado para tomar medidas en los $360$\textdegree
al rededor del robot y para proporcionar medidas perfectas (sin ruido).

\begin{figure}[H]
  \centerfloat

  \subfloat[Real.]{\includegraphics[clip=true, width=0.33\textwidth]{imagenes/pion/real.png}}
  \qquad
  \subfloat[Simulado.]{\includegraphics[clip=true, width=0.33\textwidth]{imagenes/pion/sim.png}}

  \caption[Robot diferencial Pioneer 3-DX.]{Robot diferencial Pioneer 3-DX.}\label{fig:p3dx}
   % A la izquierda se muestra el robot real, y a la derecha su version simulada.

\end{figure}

La flota de exploración se compone de cinco robots, ubicados en las posiciones
indicadas en la figura \ref{fig:willow}. Las comunicaciones entre los robots de
la flota son sin perdida y de rango infinito.

\subsection{Software}
El software utilizado en las pruebas fue Ubuntu \emph{20.04}, ROS \emph{Noetic} y Gazebo
\emph{11.5.1}. 

\subsection{Hardware}
El simulador junto al resto de procesos necesarios para llevar a cabo una
prueba fueron ejecutados en una computadora personal equipada con un procesador
Intel Core i3-9100F, un procesador gráfico GeForce GTX 660 y 16GB de memoria
RAM.

\section{Metricas}
Con el proposito comparar cuantitativamente los resultados de las pruebas se
establece un conjunto de metricas. 

Una parte de las metricas utlizadas se proponen en \cite{yan2015metrics} y
evaluan el problema de exploración en general, estas son: \emph{tiempo de
exploración}, \emph{distancia total recorrida por la flota}, \emph{completitud
del mapa} y \emph{calidad del mapa}. 

El \emph{tiempo de exploración} refiere al tiempo desde que los robots
comienzan con la primera asignación de tareas (seccion \ref{sec:asigTar}) hasta
que no se detecta mas espacio desconocido por explorar y se da por terminada la exploración.

La \emph{distancia total recorrida por la flota} es la suma las distancias
recorridas por cada robot de la flota a lo largo de la exploración. En
\cite{yan2015metrics} esta metrica se presenta como \emph{costo de
exploración}, ya que los autores la consideran como una buena aproximación del
costo energetico del robot. 

 % mapas de referencia para evaluar los mapas resultantes de las pruebas,

Para calcular las metricas de \emph{completitud} y \emph{calidad} de los mapas
es necesario contar con mapas de referencia considerados como correctos. En
este proyecto estos se representan a traves de grillas de ocupacion al igual
que los generados en la exploración. Se definen cuatro mapas de
referencia, uno por cada nivel de granularidad utlizado en las pruebas, con el
proposito de poder comparar cualquiera de los mapas obtenidos en pruebas con
un mapa de referencia de igual granularidad. Los mapas de referencia se pueden
apreciar en la figura \ref{fig:mref}. 

\begin{figure}[H]
  \centerfloat

  \subfloat[$1  \frac{celda}{m^2}$]{\includegraphics[clip=true, width=0.22\textwidth]{imagenes/willow_ref/1_000000m.png}}
  \qquad
  \subfloat[$4  \frac{celda}{m^2}$]{\includegraphics[clip=true, width=0.22\textwidth]{imagenes/willow_ref/0_500000m.png}}
  \qquad
  \subfloat[$9\frac{celda}{m^2}$]{\includegraphics[clip=true, width=0.22\textwidth]{imagenes/willow_ref/0_333333m.png}}
  \qquad
  \subfloat[$16 \frac{celda}{m^2}$]{\includegraphics[clip=true, width=0.22\textwidth]{imagenes/willow_ref/0_250000m.png}}

  \caption[Mapas de referencia utlizados.]{Mapas de referencia utlizados. En negro se indican las paredes, en blanco el espacio libre y en gris el espacio desconocido por ser inaccesible.}\label{fig:mref}
   % A la izquierda se muestra el robot real, y a la derecha su version simulada.

\end{figure}

Notar que existe espacio desconocido en los mapas de referencia, esto se debe a
que la grilla de ocupación no se ajusta al espacio explorable. En terminos de
las metricas calculadas, el espacio desconocido en los mapas de referencia no es
considerado como parte de los mapas, tanto en los mapas de referencia como en
los obtenidos en las pruebas.

% Es decir las unicas celdas que se consideran como
% parte de los mapas al calcular las metricas son las que tienen un estado
% distinto a desconocido en los mapas de referencia.
% De
% esta forma los mapas de referencia no contienen celdas desconocidas, a
% diferencia de los mapas resultantes de las pruebas los cuales si pueden.

La \emph{completitud del mapa} se define segun (\ref{eq:metComp}) donde
$M_{exp}$ es el area conocida del mapa obtenido en la exploración y $M_{ref}$
es el area del mapa de referencia.

\begin{equation} \label{eq:metComp}
completitud\ del\ mapa = \frac{M_{exp}}{M_{ref}}
\end{equation}

La \emph{calidad del mapa} se establece en (\ref{eq:metCal}) donde se introduce
$E_{exp}$ equivalente al área ocupada por las celdas del mapa obtenido en
la exploración cuyo estado difiere del estado de su celda correpondiente en el
mapa de referencia.

\begin{equation} \label{eq:metCal}
calidad\ del\ mapa = \frac{M_{exp}-E_{exp}}{M_{ref}}
\end{equation}

El resto de las metricas fueron diseñadas especificamente para este proyecto,
estas son: \emph{tiempo promedio en construcción de GVD} y \emph{tiempo
promedio en simplificación de fronteras}.

El \emph{tiempo promedio en construcción de GVD} es el promedio del tiempo que
toma construir el GVD en las etapas de obtención de información en cada
asignación de tareas (seccion \ref{sec:asigTar}).

El \emph{tiempo promedio en simplificación de fronteras} es el tiempo promedio
que se demora en simplificar las fronteras para identificar los objetivos en
cada asignación de tareas.

\section{Resultados}
\newlength{\graphlen}
\setlength{\graphlen}{0.75\textwidth}


Los resultados explerimentales obtenidos son presentados y analizados en esta
seccion. Las tablas que se presentan a lo largo de la sección indican para cada
prueba los promedios y desviaciones estandar de las metricas en las 20
repeticiones que se hacen debido al no determinismo del simulador.

\subsection{Cubrimiento y calidad de los mapas} \label{sec:exp:cubcal}
% Capaz aca puedo mencionar lo que pasa con los errores

% el cubrimiento y la completitud: completitud mayor a 0.999 para todas combinaciones de granularidades y soluciones probadas

% el error y la calidad:

En la tabla \ref{tab:todo3} se muestra la completitud y calidad de los mapas
obtenidos en todas las pruebas realizadas. 

Con respecto a la completitud es posible apreciar que no existen diferencias
mayores entre las variantes de la implemetnación, teniendo todos los mapas
generados una completitud alta. Dado que el criterio de parada es que no se
detecte mas espacio por explorar, estos valores hablan principalmente de la
capacidad de detectar el espacio desconocido explorable que tienen las
soluciones. Todas las soluciones detectan expacio desconocido explorable
mientras exista una un celda $c_1$ libre y otra $c_2$ desconocida tal que $c_1
\in ady_4(c_2)$ (seccion \ref{subsec:Grilla}), principalemente porque en estas
condiciones se asume que los robots no pueden atravezar de $c_1$ a $c_2$. Dicho
esto se presume que la completitud de los mapas no es perfecta debido a la
presencia de celdas ubicadas en esquinas de paredes que que a pesar de no
cumplir con las codiciones de ser detectadas como explorables, lo son. Un
ejempo de este tipo de celda se muestra en la figura \ref{fig:faltaCub}. Sin
embargo los valores de completitud obtenidos indican que este tipo de
situaciones son despreciables.

\begin{figure}[H]
  \centerfloat

  \subfloat[Mapa de referencia.]{\includegraphics[clip=true, width=0.33\textwidth]{imagenes/faltaDeCub/ogred.png}}
  \qquad
  \subfloat[Mapa generado.]{\includegraphics[clip=true, width=0.33\textwidth]{imagenes/faltaDeCub/faltaCubRed}}

  \caption{Esquina superior izquierda del mapa referencia y de uno de los mapas
  obtenidos en las pruebas que presenta una celda explorable que no se reconoce
como tal. Ambos mapas corresponden a una granularidad de una celda por metro cuadrado.}\label{fig:faltaCub}
   % A la izquierda se muestra el robot real, y a la derecha su version simulada.

\end{figure}

Los resultados de calidad de los mapas obtenidos son similares a los de
completitud, los mapas contruidos tienen una calidad alta, sin existir
diferencias significativas entre las variantes. Esto es esperable ya que
los sensores simulados fueron configurados para proporcionar medidas perfectas.
Aunque a pesar de esto la calidad no perfecta, esto puede se explica en primer
lugar por el impacto de la completitud del mapa en su calidad. En segundo lugar
por la existencia de varios robots que pueden detectase como obstaculos entre
sí, generado celdas obstaculizadas que no se encuentran en el mapa de
referencia. Y en tercer lugar porque a pesar de las configuración de los
sensores, sus medidas tienen un ruido pequeño.

Estos resultados de completitud y calidad del mapa permiten concluir que todas
las variantes logran resolver de manera satisfactoria el problema de la
exploración multirobot, en terminos de los mapas contruidos. Las
siguientes secciones se dedican a discutir los impactos que las diferentes
variantes en terminos de tiempo y costo de exploracion.

% De la
% equacion 4 se de deduce que el cubrimiento siempre es mayor o igual que la
% calidad. Por lo tanto 


%%%%%%%%
% Dado que las diferencias son despreciables se puede
% concluir que el impacto de el uso de las distintas tambien despreciable.

% confirman que todas las variantes de la implemetnacion 
% finalizar su ejecucion con un mapa casi completo del entorno.

% Estos resutlados son los esperados ya que
% la mision concluye al no quedar mas espacio por explorar.

% Es posible apreciar que todos los
% mapas cubren casi en su totalidad al entorno y que la calidad de dichos mapas
% es casi perfecta en todos los casos.

\begin{table}[H]
%29/12/2021 18:15:43
\hbadness = 10000
\tolerance=9999
\emergencystretch=10pt
\hyphenpenalty=10000
\exhyphenpenalty=100
\begin{center}

% \begin{adjustbox}{minipage=0.75\paperwidth, center}
\begin{adjustbox}{width=1\textwidth}
\small

\begin{tabularx}{\textwidth}{|X|C{0.80cm}|X|X|}

\hline
Variante & $\frac{celdas}{m^2}$ & Completitud del mapa & Calidad del mapa \\ \hline\hline
\multirow{4}{\linewidth}{\centering Propuesta sin cambios}
& 1 & 0.999725±1.5e-04 & 0.999039±3.1e-04\\ \cline{2-4}
& 4 & 0.999925±6.5e-05 & 0.999703±1.7e-04\\ \cline{2-4}
& 9 & 0.999983±1.2e-05 & 0.999858±5.7e-05\\ \cline{2-4}
& 16 & 0.999998±3.6e-06 & 0.999929±3.4e-05\\ \hline\hline
\multirow{4}{\linewidth}{\centering No incremental}
& 1 & 0.999763±8.9e-05 & 0.999208±2.2e-04\\ \cline{2-4}
& 4 & 0.999957±4.2e-05 & 0.999805±1.3e-04\\ \cline{2-4}
& 9 & 0.999984±1.6e-05 & 0.999864±4.7e-05\\ \cline{2-4}
& 16 & 0.999999±1.9e-06 & 0.999941±3.3e-05\\ \hline\hline
\multirow{4}{\linewidth}{\centering Fronteras significativas basadas en K-Means}
& 1 & 0.999802±1.3e-04 & 0.999179±2.8e-04\\ \cline{2-4}
& 4 & 0.999977±2.7e-05 & 0.999767±8.0e-05\\ \cline{2-4}
& 9 & 0.999976±2.3e-05 & 0.999831±7.4e-05\\ \cline{2-4}
& 16 & 0.999999±2.5e-06 & 0.999930±1.8e-05\\ \hline\hline
\multirow{4}{\linewidth}{\centering Fronteras}
& 1 & 0.999937±7.6e-05 & 0.999493±2.0e-04\\ \cline{2-4}
& 4 & 0.999995±9.8e-06 & 0.999837±4.8e-05\\ \cline{2-4}
& 9 & 0.999983±4.0e-05 & 0.999729±7.6e-05\\ \cline{2-4}
& 16 & 0.999998±4.1e-06 & 0.999921±2.2e-05\\ \hline\hline
\multirow{4}{\linewidth}{\centering Desconocidas se consideran libres}
& 1 & 0.999812±1.1e-04 & 0.999256±3.2e-04\\ \cline{2-4}
& 4 & 0.999972±1.9e-05 & 0.999786±9.7e-05\\ \cline{2-4}
& 9 & 0.999987±1.2e-05 & 0.999840±6.9e-05\\ \cline{2-4}
& 16 & 0.999998±3.5e-06 & 0.999940±2.8e-05\\ \hline
\end{tabularx}
\end{adjustbox}

\caption{Completitud y calidad de los mapas obtenidos en todas las pruebas realizadas.}
\label{tab:todo3}
\end{center}
\end{table}


\subsection{Incrementalidad}\label{sec:exp:inc}
En esta sección se comparan los resultados de las pruebas realizadas con
solución propuesta que construye el GVD de forma incremental contra los de una
variante que solo difiere en que la construcción del GVD es no incremental. Las
métricas a analizar para estas pruebas se encuentran en la tabla \ref{tab:inc1}
y graficadas en las figuras \ref{fig:gra:inc:et}, \ref{fig:gra:inc:ec} y
\ref{fig:gra:inc:gvdt}.

Se puede observar que los resultados obtenidos validan la idea de que la
construcción incremental del GVD tiene mayor eficiencia computacional que su
contraparte no incremental. 

En cada nivel de granularidad la variante incremental reduce un ${\smallsim}94\%$ el
tiempo promedio construcción del GVD con respecto a la variante no incremental. 

El tiempo de construcción de GVD impacta directamente al tiempo que ocurre
desde que un robot pide un tarea y una le es asignada, ya que es una parte 
necesaria y no paralelizada de la etapa de obtención de información
(sección \ref{subsec:obtInfo}) de cada asignación de tareas (sección
\ref{sec:asigTar}). A pesar de esto el impacto sobre los tiempos de exploración
no es directo ya que una asignación de tareas este en curso solo asegura que un
robot estara osicoso, ya que mientras que un robot pide una tarea y la obitene
el resto puede estar completando tareas asignadas previamente. Cuanto mas chica
es la demora en la asignación de tareas mas probable es que solo un robot quede
osicoso, mientras que si dicha demora crece aumenta la probabilidad que en en
transcurso de la asignacion mas robots completen sus tareas y requieran tareas,
quedando ociosos. Esto logra explicar que en todos los niveles de granularidad
la reduccion del tiempo promedio logrado por la variante incremental frente a
la no incremental sea de un ${\smallsim}94\%$ constante mientras que la
reduccion del tiempo de exploración comienza siendo de un ${\smallsim}8\%$ en
el nivel de granularidad mas bajo (tiempos de construcción de GVD más altos),
creciendo mediante se aumenta el nivel de granularidad llegando a una reduccion
de un ${\smallsim}63\%$ en la granularidad mas alta (tiempos de construcción de
GVD mas altos).

% no son
% monótonamente crecientes ni decrecientes respecto a la granularidad

Construir el GVD de forma incremental también parece reducir las distancias
totales recorridas por la flota en todas las granularidades, aunque en este
caso las reducciones no son significativas. Especificamente, las reducciones
estan comprendidas entre $2\%$ y $7\%$, fluctuando al aumentar los
niveles de granularidad y adicionalmente las diferencias entre los promedios de
esta metrica son similares a sus desviaciones estándar. Esto puede deberse a
que tiempos más rapidos de asignación de objetivos hacen mas probable que los
robots cambien su trayectoria a un nuevo objetivo antes de llegar a la
ubicación exacta del objetivo previamente completado. Esto puede reducir la
distancia recorrida por el robot si los caminos hacia el objetivo anterior y el
actual no se solapan.

% En cada nivel de granularidad el tiempo promedio en construcción del GVD de
% variante incremental siempre es menor que el de la variante no incremental,
% aumentando la diferencia a medida que aumentan las celdas por metro cuadrado.
% La reducción del tiempo tiempo de construcción del GVD que se logra al

% construirlo de forma incremental implica una reducción del $55\%$ en el
% porcentaje de tiempo de la obtención de información en la que se construye el
% GVD aproximadamente que se repite en todos los niveles de granularidad.

% Como los tiempos de construcción promedio del GVD son menores
% en las pruebas realizadas con las construcción incremental del GVD,

% al aumentar las celdas por metro cuadrado, el tiempo
% promedio en construcción del GVD, en el caso de la construcción incremental crece de a 

\begin{table}[H]
%24/12/2021 03:06:44
\hbadness = 10000
\tolerance=9999
\emergencystretch=10pt
\hyphenpenalty=10000
\exhyphenpenalty=100
\begin{center}

% \begin{adjustbox}{minipage=0.75\paperwidth, center}
\begin{adjustbox}{width=1\textwidth}
\small

\begin{tabularx}{\textwidth}{|X|C{0.80cm}|X|X|}

\hline
Construcción del GVD & $\frac{celdas}{m^2}$ & Tiempo de exploración $(s)$ & Distancia total recorrida por la flota $(m)$ \\ \hline\hline
\multirow{4}{\linewidth}{\centering Incremental}
& 1 & 480.8±34.7 & 2587.0±186.4\\ \cline{2-4}
& 4 & 498.0±25.3 & 2713.1±107.5\\ \cline{2-4}
& 9 & 553.4±24.9 & 2894.7±124.5\\ \cline{2-4}
& 16 & 689.7±29.0 & 3067.9±121.9\\ \hline\hline
\multirow{4}{\linewidth}{\centering No incremental}
& 1 & 501.9±22.4 & 2653.4±96.5\\ \cline{2-4}
& 4 & 745.7±23.1 & 2851.1±144.0\\ \cline{2-4}
& 9 & 1198.7±35.8 & 3038.1±159.2\\ \cline{2-4}
& 16 & 1856.7±56.2 & 3117.0±170.3\\ \hline
\end{tabularx}
\end{adjustbox}

\caption{Resultados de tiempo y costo de exploración obtenidos en las pruebas realizadas con la construcción incremental y no incremental del GVD.}
\label{tab:inc1}
\end{center}

\end{table}



\begin{figure}[H]
  \centerfloat

  \includegraphics[clip=true, width=\graphlen]{imagenes/graficas_chicas/graficas_histo_num/incrementalidad/exploration_time.png}

  \caption{Grafica de tiempo de exploración en función de celdas por metro cuadrado.}\label{fig:gra:inc:et}

\end{figure}

\begin{figure}[H]
  \centerfloat

  \includegraphics[clip=true, width=\graphlen]{imagenes/graficas_chicas/graficas_histo_num/incrementalidad/exploration_cost.png}

  \caption{Grafica de distancia total recorrida por la flota  en función de celdas por metro cuadrado.}\label{fig:gra:inc:ec}

\end{figure}

\begin{figure}[H]
  \centerfloat

  \includegraphics[clip=true, width=\graphlen]{imagenes/graficas_chicas/graficas_histo_num/incrementalidad/gvd_construction_time_mean.png}

  \caption{Grafica de la sumatoria del tiempo en construcción de GVD en función de celdas por metro cuadrado.}\label{fig:gra:inc:gvdt}

\end{figure}


\subsection{Identificación de objetivos}\label{sec:exp:idobj}
En esta sección se realiza un analisis comparativo de los resultados obtenidos
en pruebas realizadas con tres soluciones distintas que difieren unicamente en
el metodo usado para identificar objetivos (sección \ref{sec:pc:idobj}). Estos
metodos son: la simplificación de fronteas basada en cubrimiento propuesta en
este proyecto, simplificación de fronteras basada en K-Means como es utilizada en
\cite{amorin2019novel} y la identificación de fronteras (sin simplificar)
introducida en \cite{yamauchi1998frontier}. Los resultados de las métricas a
analizar se presentan en la tabla \ref{tab:inc1} y se encuentran graficadas en
las figuras \ref{fig:gra:inc:et}, \ref{fig:gra:inc:ec} y
\ref{fig:gra:inc:gvdt}.

%%% 1 %%%%
% Es posible apreciar que la variante que identifica como objtivos a las
% fronteras sin simplificar (\emph{FSS}) como objetivos obtiene los peores
% resultados en terminos de tiempos de exploración. Esto explica de forma analoga
% a como se explico el comportamiento de esta misma metrica en la sección
% \ref{sec:exp:inc}. En este caso lo que impacta en el tiempo de la asignación de
% tareas es la la existencia de una cantidad mayor de objetivos que causa que las
% etapas de valuación (sección \ref{subsec:MiValSub}) y de resolución (sección
% \ref{subsec:MiResSub}) tomen mas tiempo al ser mayor la cantidad de objtivos a
% valuar y asignar. De los tiempos de exploración obtenidos al utlizar los otros
% dos metodos se deduce que estos consiguen reducir las duraciones de las asignaciónes de
% tareas, cosa que logran invertiendo tiempo en reducir los objetivos
% indentificados al simplificar las fronteras a sus celdas más significativas.
% Especificamente en las pruebas realizadas el simplificar las fronteras reduce
% los tiempos de exploración con respecto a no simplificar, aunque comenzando por
% el nivel mas bajo de granularidad la reducción es despreciable esta crece hasta
% alcanzarse una reducción de ${\smallsim}29\%$ en el nivel de granularidad mas
% alto.
%%% 2 %%%%
% Es posible apreciar que la variante que identifica como objtivos a las
% fronteras sin simplificar (\emph{FSS}) como objetivos obtiene los peores
% resultados en terminos de tiempos de exploración, o lo que es lo mismo, los
% metodos que simplifican las fronteras reducen los tiempos de exploración con
% respecto a no simplificar. Aunque comenzando por el nivel mas bajo de
% granularidad las reducciones son despreciables estas crecen hasta alcanzar una
% reducción de ${\smallsim}29\%$ en el nivel de granularidad mas alto. Esto en
% parte se debe a lo explicado en la sección \ref{sec:exp:inc} con respecto a
% esta metrica. En este caso lo que impacta en el tiempo de la asignación de
% tareas es la cantidad de objetivos identificados, ya que aumentar dicha
% cantidad implica aumentos en la duracion de las etapas de valuación (sección
% \ref{subsec:MiValSub}) y de resolución (sección \ref{subsec:MiResSub}) al ser
% mayor la cantidad de objtivos a valuar y asignar. De los tiempos de exploración
% obtenidos al utlizar los otros dos metodos se deduce que estos consiguen
% reducir las duraciones de las asignaciónes de tareas, cosa que logran
% invertiendo tiempo en reducir los objetivos indentificados al simplificar las
% fronteras a sus celdas más significativas.
%%% 3 %%%%
% Es posible apreciar que los metodos que identifican los objetivos simplificando
% las fronteras reducen los tiempos de exploración con respecto al metodo que
% identifica a las fronteras sin idenficar como obtivos. Aunque en el nivel mas
% bajo de granularidad las reducciones son despreciables estas crecen junto a la
% granularidad, hasta alcanzarse una reducción de ${\smallsim}29\%$ en el nivel
% de granularidad mas alto. Esto debe en parte a los tiempos de asignación de
% tareas, por lo explicado en la sección \ref{sec:exp:inc} con respecto a esta
% metrica. En este caso el principal impacto en
% los tiempos de asignación de tareas es la cantidad de objetivos identificados.
 
% implica reducciones en la duracion de las etapas de valuación (sección
% \ref{subsec:MiValSub}) y de resolución (sección \ref{subsec:MiResSub}) al ser
% menor la cantidad de objetivos a valuar y asignar. Aunque otra razon posible,
% es que el simplificar las fronteras puede coordinar a los robots. Por ejemplo
% de no simplificarse las fronteras se permite que un robot sea asignado a un
% objetivo adyacentes a una pared o que dos robots sean asignados a objetivos
% adyacentes entre si, situaciones en las cuales se desaprovecha capacidad de
% sensado, y se complejiza la navegacion. Al simplificarse las fronteras,
% especialemente cuando se logra el cubrimiento minimizando los objetivos
% resultantes, se evitan las situaciones descritas similares lo cual puede verse
% como una coordinación implicita.


%%% 4 %%%%
% Los metodos que simplifican las fronteras invierten tiempo en simplificar para disminuir dicha la cantidad de objetivos

Es posible apreciar que los metodos que identifican los objetivos simplificando
las fronteras reducen los tiempos de exploración con respecto al metodo que
identifica a las fronteras sin simplificar como objetivos. Aunque en el nivel
de granularidad mas bajo las reducciones son despreciables, estas crecen junto
a la granularidad hasta alcanzarse una reducción de ${\smallsim}29\%$ en el
nivel mas alto. Esto puede deberse al tiempo de asignación de tareas como se
explica en la sección \ref{sec:exp:inc}. En este caso se considera que los
factores con mayor impacto sobre los tiempos de asignación de tareas son dos:
el tiempo de simplificación de fronteras y la cantidad de objetivos
identificados. Mayores tiempos de simplificación fronteras retrasan la
identificación de objetivos, parte necesaria y no paralelizada de la etapa de
obtencion de información (sección \ref{subsec:obtInfo}) en la asignacion de
tareas. Mientras que disminuir la cantidad de objetivos identificados implica
reducciones en la duracion de las etapas de valuación (sección
\ref{subsec:MiValSub}) y de resolución (sección \ref{subsec:MiResSub}) al ser
menor la cantidad de objetivos a valuar y a considerar en la resolución. Entonces lo que se evidencia
en los resultados obtenidos, es que invertir tiempo en disminuir los objetivos
identificados simplificando las fronteras lleva a una reduccion total de los
tiempos de asignación de tareas, y en consecuencia de los tiempos de exploración
frente a no invertir dicho tiempo y utlizar todas las fronteras sin
simplificar. Otra razon posible para las reducciones del tiempo de exploración
es que simplificar las fronteras puede aportar a la coordinación de los robots.
Por ejemplo de no simplificarse las fronteras se permite que un robot sea
asignado a un objetivo adyacente a una pared o que dos robots sean asignados a
objetivos adyacentes entre sí, situaciones en las cuales se desaprovecha
capacidad de sensado, y se complejiza la navegacion. Al simplificarse las
fronteras, especialemente cuando se logra el cubrimiento minimizando los
objetivos resultantes, se evitan situaciones como las descritas lo cual puede
verse como una coordinación pasiva. \todo{o implicita?}

% simplificandose
% la navegación y aprovechandose las capacidades sensoriales de los robots. Esto

% A simple vista este factor parece contradecir los resultados obtenidos, ya
% que el metodo que obtiene los peores resultados es el menos impactado por
% este factor ya que no simplifica las fronteras. Lo que sucede es que el
% tiempo de simplificacion es en realidad una inversion de tiempo que se hace
% para reducir la cantidad de objetivos

% las simplificaciones que minimizan el numero de fronteras significativas
% mientras mantienendo el cubrimiento minimizan el solapamiento de sensado de
% los robots, aumentado la nueva información que se obiente dele entorno. es la
% principal razon por la cual se introducen los metodos basados en
% simplificación, ya que estos la cantidad de objetivos para disminuir el
% tiempo total de la asignación de tareas.

% se mantienen entre un ${\smallsim}9\%$ y ${\smallsim}11\%$, exceptuando la
% granularidad de una celda por metro cuadrado en la simplificación basada en
% K-Means, en la que se obtiene una reduccion del ${\smallsim}7\%$

De acuerdo con los resultados simplificar las fronteras en la identificacion de
objetivos también reduce las distancias totales recorridas por la flota frente
a no simplificarlas. Estas reducciones son en general de ${\smallsim}10\%$, y
pueden ser causadas por la disminución del tiempo de asignación de tareas,
debido a las mismas razones que se explican en la seccion \ref{sec:exp:inc}.
Aunque la coordinación pasiva también puede estar jugando un rol al evitar que
los robots se transladen a objetivos que desaprovechan las capacidades
sensoriales del robot o que puedan llevar a inconvenientes en la navegacion.


% Con respecto a las dos tecnicas de simplificacion los comportamientos en las
% metricas de tiempo de exploración y distancia total recorrida por la flota son
% similares.
% , tienen comportamientos similares tanto en el
% tiempo de exploración como en distancia total recorrida por la flota. 
% omo los niveles de granularidad tienen el proposito
% variar la carga computacional, los resutados mencionados parecen indicar que la
% simplificación basada en el cubrimiento es la mejor cuando la carga
% computacional es baja, siendo superada por la simplificación basada en K-Means cuando
% la carga computacional es alta. 
Con respecto a los dos metodos que simplifican las fornteras para 
identificar objetivos, en el nivel de granularidad mas bajo, la
simplificación basada en cubrimiento logra una reducción con respecto a la
basada en K-Means de ${\smallsim}3\%$ en la distancia total recorrida por la
flota y de ${\smallsim}5\%$ en los tiempo de exploración. Esto se invierte a
medida que aumenta el nivel de granularidad, hasta que en el nivel mas alto en
lugar de reducirse, las metricas aumentan. La distancia total recorrida por la
flota aumenta en un ${\smallsim}3\%$ y el tiempo de exploración en un
${\smallsim}6\%$. Por otro lado segun los resultados obtenidos del tiempo
promedio en simplificación de fronteras, la simplificacion basada en
cubrimiento en todo nivel de granularidad es entre $2$ y
$4$ veces mas lenta que la basada en K-Means. Estos resultados
parecen indicar que la simplificación basada en cubrimiento toma mas tiempo
pero logra mejores reducciones de objetivos que la basada en K-Means. En el
nivel de granularidad mas bajo el aumento de tiempo no es tan relevante como la
reduccion de objetivos, por lo tanto la simplificación basada en cubrimiento
obtiene un mejor redimiento. A medida la granularidad crece el aumento de
tiempo se torna mas relevante que la reduccion de objetivos, por lo que la
simplificación basada en K-Means obtiene un rendimiento superior.  

% costo computacional vs  
%%, cuando los tiempos de simplificación son los mas cortos,
%%

  % Dado que la
% simplificación basada en cubrimiento tiene siempre un mayor costo temporal que
% la basada en K-Means pero logra obtener

% Lo cual implica que el metodo que simplifica basadose en el cubrimiento tiene
% una inversion mayor de tiempo, y a pesar.
% En este caso ambos metodos invierten tiempo en disminuir los
% objetivos identificados a partir de simplificar las fronteras.

% Con respecto a las dos tecnicas de simplificacion estas tienen comportamientos
% similares tanto en el tiempo de exploración como en distancia total recorrida
% por la flota. En el nivel de granularidad mas bajo, la simplificación basada en
% cubrimiento logra una reducción con respecto a la basada en K-Means, de
% ${\smallsim}3\%$ de la distancia total recorrida por la flota y de

  % Esto concuerda con lo
% que ocurre en los niveles de granularidad mas altos, donde la simplificacion
% basada en cubrimiento obtiene peores resultados que la basada en K-Means. Con
% respecto al nivel de granularidad mas bajo, lo que puede estar sucediendo es
% que al ser el nivel con menos costo computacional y por lo tanto 

% ste comportamiento en parte se explica por los tiempos de
% asignación de tareas y su efecto en dichas metricas. En este caso aunque pueden
% existir diferencias entre los numeros de objetivos identificados, estas son
% menos significativas repecto a identificar todas las fronteras como objetivos.
% Dado esto para los metodos que simplifican las fronteras se pasa a considerar
% otro factor de impacto de los tiempos de asignación, el tiempo que demora dicha
% simplificación. 

% En este caso ambos
% metodos reducen la cantidad de objetivos identificados simplificando las
% fronteras, por lo tanto aunque pueden existir diferencias entre los numeros de
% objetivos identificados en estos metodos, estas son menores en comparacion al
% todas las fronteras como objetivos, y en consecuencia este aspecto tiene menor
% impacto sobre el tiempo de la asignación de objetivos. Dado esto se pasa a
% considerar otro factor de impacto de los tiempos de asignación: el tiempo que
% demora la simplificación de fronteras. 

% Dado esto se pasa a
% considerar otro factor de impacto de los tiempos de asignación: el tiempo que
% demora la simplificación de fronteras. Según los resultados el tiemp

% Dado que estos metodos
% solo difieren en como realizan la simplificacion de fronteras, la demora de
% dicha simplificación es uno de los principales factores a tener en cuenta.



% dado que los tiempos de demora de la asignación de tareas se determino como un
% factor, que aunque tenia un peso, no era significativos, y en este caso donde
% las demoras segun los timpos de exploraicon son menores, se obtienen
% diferencias de distancia mas significativas entoces, se puede deteminar que la
% coordinación implicita es un factor de impacto en la distnacia.
% La variante \emph{FSS} también obtiene los peores resultados en terminos de
% distancia total recorrida por la flota. Este resultado se atribuye en primer
% lugar a que al ser todas las celdas fronteras objtivos posibles, esto permite asignaciones

% Y en segundo lugar al amuento del tiempo de asignación de tareas como se
% explica en la sección \ref{sec:exp:inc},  



%aunque esto no justifica que las reducciónes de las
% distancias sean mayores a las obtenidas en las pruebas de dicha seccion a pesar de que las 


\begin{table}[H]
%24/12/2021 03:06:43
\hbadness = 10000
\tolerance=9999
\emergencystretch=10pt
\hyphenpenalty=10000
\exhyphenpenalty=100
\begin{center}

% \begin{adjustbox}{minipage=0.75\paperwidth, center}
\begin{adjustbox}{width=1\textwidth}
\small

\begin{tabularx}{\textwidth}{|X|C{0.80cm}|X|X|}

\hline
Identificación de objetivos & $\frac{celdas}{m^2}$ & Tiempo de exploración $(s)$ & Distancia total recorrida por la flota $(m)$ \\ \hline\hline
\multirow{4}{\linewidth}{\centering Simplificación de fronteras basada en cubrimiento}
& 1 & 480.8±34.7 & 2587.0±186.4\\ \cline{2-4}
& 4 & 498.0±25.3 & 2713.1±107.5\\ \cline{2-4}
& 9 & 553.4±24.9 & 2894.7±124.5\\ \cline{2-4}
& 16 & 689.7±29.0 & 3067.9±121.9\\ \hline\hline
\multirow{4}{\linewidth}{\centering Simplificación de fronteras basada en K-Means}
& 1 & 491.0±25.7 & 2628.0±98.6\\ \cline{2-4}
& 4 & 494.3±28.2 & 2684.8±106.7\\ \cline{2-4}
& 9 & 532.6±21.1 & 2813.0±96.4\\ \cline{2-4}
& 16 & 652.0±32.7 & 2962.4±136.6\\ \hline\hline
\multirow{4}{\linewidth}{\centering Fronteras sin simplificar}
& 1 & 487.9±24.3 & 2750.7±142.3\\ \cline{2-4}
& 4 & 532.8±24.7 & 3028.3±136.1\\ \cline{2-4}
& 9 & 643.3±22.8 & 3216.9±121.3\\ \cline{2-4}
& 16 & 925.8±104.7 & 3352.5±208.7\\ \hline
\end{tabularx}
\end{adjustbox}

\caption{Resultados de tiempo y costo de exploración obtenidos en las pruebas realizadas con los distintos métodos de identificación de objetivos.}
\label{tab:ident_obj1}
\end{center}

\end{table}



\begin{figure}[H]
  \centerfloat

  \includegraphics[clip=true, width=\graphlen]{imagenes/graficas_chicas/graficas_histo_num/ident_obj/exploration_time.png}

  \caption{Grafica de tiempo de exploración en función de celdas por metro cuadrado.}\label{fig:gra:idobj:et}

\end{figure}

\begin{figure}[H]
  \centerfloat

  \includegraphics[clip=true, width=\graphlen]{imagenes/graficas_chicas/graficas_histo_num/ident_obj/exploration_cost.png}

  \caption{Grafica de distancia total recorrida por la flota en función de celdas por metro cuadrado.}\label{fig:gra:idobj:ec}

\end{figure}

\begin{figure}[H]
  \centerfloat

  \includegraphics[clip=true, width=\graphlen]{imagenes/graficas_chicas/graficas_histo_num/ident_obj/obj_id_time_mean.png}

  \caption{Grafica de la sumatoria del tiempo en obtencion de información función de celdas por metro cuadrado.}\label{fig:gra:idobj:iobt}

\end{figure}

\subsection{Consideracion del espacio desconocido}\label{sec:exp:desco}

\begin{table}[H]
%24/12/2021 03:06:43
\hbadness = 10000
\tolerance=9999
\emergencystretch=10pt
\hyphenpenalty=10000
\exhyphenpenalty=100
\begin{center}

% \begin{adjustbox}{minipage=0.75\paperwidth, center}
\begin{adjustbox}{width=1\textwidth}
\small

\begin{tabularx}{\textwidth}{|X|C{0.80cm}|X|X|}

\hline
Consideración del espacio desconocido & $\frac{celdas}{m^2}$ & Tiempo de exploración $(s)$ & Distancia total recorrida por la flota $(m)$ \\ \hline\hline
\multirow{4}{\linewidth}{\centering Desconocidas no propagan olas y $UF \subseteq CGen$}
& 1 & 480.8±34.7 & 2587.0±186.4\\ \cline{2-4}
& 4 & 498.0±25.3 & 2713.1±107.5\\ \cline{2-4}
& 9 & 553.4±24.9 & 2894.7±124.5\\ \cline{2-4}
& 16 & 689.7±29.0 & 3067.9±121.9\\ \hline\hline
\multirow{4}{\linewidth}{\centering Desconocidas se consideran libres}
& 1 & 479.1±34.4 & 2522.2±149.5\\ \cline{2-4}
& 4 & 510.5±27.1 & 2728.9±135.2\\ \cline{2-4}
& 9 & 677.8±26.4 & 3134.2±148.8\\ \cline{2-4}
& 16 & 955.4±42.1 & 3303.6±240.5\\ \hline
\end{tabularx}
\end{adjustbox}

\caption{Resultados de tiempo y costo de exploración obtenidos en las pruebas realizadas con las distintas consideraciones del espacio desconocido al construir el GVD.}
\label{tab:desconocido1}
\end{center}

\end{table}


\begin{figure}[H]
  \centerfloat

  \includegraphics[clip=true, width=\graphlen]{imagenes/graficas_chicas/graficas_histo_num/desconocido/exploration_time.png}

  \caption{Grafica de tiempo de exploración en función de celdas por metro cuadrado.}\label{fig:gra:des:et}

\end{figure}

\begin{figure}[H]
  \centerfloat

  \includegraphics[clip=true, width=\graphlen]{imagenes/graficas_chicas/graficas_histo_num/desconocido/exploration_cost.png}

  \caption{Grafica de distancia total recorrida por la flota en función de celdas por metro cuadrado.}\label{fig:gra:des:ec}

\end{figure}

\begin{figure}[H]
  \centerfloat

  \includegraphics[clip=true, width=\graphlen]{imagenes/graficas_chicas/graficas_histo_num/desconocido/gvd_construction_time_mean.png}

  \caption{Grafica de la sumatoria del tiempo en construcción de GVD en función de celdas por metro cuadrado.}\label{fig:gra:des:gvdt}

\end{figure}



\chapter{Conclusiones y trabajo futuro}\label{cha:concl}
\hfuzz=10pt 
\minitoc
\hfuzz=0pt 

\section{Conclusiones}

\section{Trabajo a futuro}


\begin{appendices}
\chapter{Ejemplos completos}
\section{Simplificacion de fronteras basada en cubrimiento}\label{EXE:sfbc}
\begin{figure}[H]
  \centerfloat
  \subfloat[Estado inicial.]{\includegraphics[clip=true, width=0.40\textwidth]{imagenes/ejemploSimpCub/a1.png}}
  \subfloat[Se inicializa $\mli{FP}$.]{\includegraphics[clip=true, width=0.40\textwidth]{imagenes/ejemploSimpCub/a2.png}}

  \subfloat[Se obtiene $\mli{fp}$ desencolando de $\mli{FP}$.]{\includegraphics[clip=true, width=0.40\textwidth]{imagenes/ejemploSimpCub/b1.png}}
  \subfloat[Existen celdas mas alejadas que $2*rango$, los candidatos se obienen con $radio=rango$. ]{\includegraphics[clip=true, width=0.40\textwidth]{imagenes/ejemploSimpCub/b3.png}}
  \subfloat[Se elige el unico candidato como $\mli{fs}$, se actualiza $\mli{UF}$, $\mli{FS_i}$ y $\mli{FP}$.]{\includegraphics[clip=true, width=0.40\textwidth]{imagenes/ejemploSimpCub/b5.png}}
 \phantomcaption

\end{figure}

\begin{figure}[H]
  \setcounter{subfigure}{5}
  \centerfloat
  \subfloat[Se obtiene $\mli{fp}$ desencolando de $\mli{FP}$.]{\includegraphics[clip=true, width=0.40\textwidth]{imagenes/ejemploSimpCub/c1.png}}
  \subfloat[Existen celdas mas alejadas que $rango$, los candidatos se obienen con $radio=rango$. ]{\includegraphics[clip=true, width=0.40\textwidth]{imagenes/ejemploSimpCub/c3.png}}
  \subfloat[Se elige arbitrariamente un candidato como $\mli{fs}$, se actualiza $\mli{UF}$, $\mli{FS_i}$ y $\mli{FP}$.]{\includegraphics[clip=true, width=0.40\textwidth]{imagenes/ejemploSimpCub/c5.png}}


  \subfloat[Se obtiene $\mli{fp}$ desencolando de $\mli{FP}$.]{\includegraphics[clip=true, width=0.40\textwidth]{imagenes/ejemploSimpCub/d1.png}}
  \subfloat[Existen celdas mas alejadas que $2*rango$, los candidatos se obienen con $radio=rango$. ]{\includegraphics[clip=true, width=0.40\textwidth]{imagenes/ejemploSimpCub/d3.png}}
  \subfloat[Se elige arbitrariamente un candidato como $\mli{fs}$, se actualiza $\mli{UF}$, $\mli{FS_i}$ y $\mli{FP}$.]{\includegraphics[clip=true, width=0.40\textwidth]{imagenes/ejemploSimpCub/d5.png}}

  \subfloat[Se obtiene $\mli{fp}$ desencolando de $\mli{FP}$.]{\includegraphics[clip=true, width=0.40\textwidth]{imagenes/ejemploSimpCub/e1.png}}
  \subfloat[Existen celdas mas alejadas que $2*rango$, los candidatos se obienen con $radio=rango$. ]{\includegraphics[clip=true, width=0.40\textwidth]{imagenes/ejemploSimpCub/e3.png}}
  \subfloat[Se elige arbitrariamente un candidato como $\mli{fs}$, se actualiza $\mli{UF}$, $\mli{FS_i}$ y $\mli{FP}$.]{\includegraphics[clip=true, width=0.40\textwidth]{imagenes/ejemploSimpCub/e5.png}}

 \phantomcaption
\end{figure}

\setcounter{figure}{0}

\begin{figure}[H]
  \setcounter{subfigure}{14}
  \centerfloat

  \subfloat[Se obtiene $\mli{fp}$ desencolando de $\mli{FP}$.]{\includegraphics[clip=true, width=0.40\textwidth]{imagenes/ejemploSimpCub/f1.png}}
  \subfloat[No existen celdas mas alejadas que $2*rango$, los candidatos se obienen con $radio=3.45$ (largos de celda). ]{\includegraphics[clip=true, width=0.40\textwidth]{imagenes/ejemploSimpCub/f3.png}}
  \subfloat[Se elige el unico candidato como $\mli{fs}$, se actualiza $\mli{UF}$, $\mli{FS_i}$ y $\mli{FP}$.]{\includegraphics[clip=true, width=0.40\textwidth]{imagenes/ejemploSimpCub/f4.png}}


  \subfloat[$\mli{UF}=\emptyset$ por lo que el algorimo finaliza.]{\includegraphics[clip=true, width=0.40\textwidth]{imagenes/ejemploSimpCub/zfinal1.png}}
  % \subfloat[$\mli{UF}=\emptyset$ por lo que el algorimo finaliza.]{\includegraphics[clip=true, width=0.40\textwidth]{imagenes/ejemploSimpCub/zfinal3.png}}

  \caption[Proceso de simplificacion de fronteras según cubrimiento.]{Proceso
    de simplificacion de fronteras según cubrimiento. Las fronteras de $F_i$ se
    indican con azul si pertenecen a $\mli{UF}$ y con amarillo de lo contrario.
    Con magenta se indican las celdas en $\mli{FP}$ siendo la numeracion su
    orden en la cola y $\mli{fp}$ la ultima desencolada. Las
    circunferencias rojas centradas en $\mli{fp}$ tienen como radio la distancia
    utilizada para obtener los candidatos $\mli{FSC}$ los cuales se indican con
    naranja. Las fronteras significativas se indican con verde, siendo las
    circunferencias rojas de radio $rango=5.6$ (largos de celda) centradas en estas indicadores de su
  cubrimiento.}\label{fig:ejemploFSCubComp}

\end{figure}

\end{appendices}

% == BIBLOGRAPHY ==
\hbadness=10000
%\begingroup
%\bibliographystyle{IEEEtran}   %facundo
\bibliographystyle{apalike}    %facundo
%\raggedright

\bibliography{references}

\end{document}
