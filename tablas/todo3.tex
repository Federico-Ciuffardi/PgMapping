\begin{table}[H]
%29/12/2021 18:15:43
\hbadness = 10000
\tolerance=9999
\emergencystretch=10pt
\hyphenpenalty=10000
\exhyphenpenalty=100
\begin{center}

% \begin{adjustbox}{minipage=0.75\paperwidth, center}
\begin{adjustbox}{width=1\textwidth}
\small

\begin{tabularx}{\textwidth}{|X|C{0.80cm}|X|X|}

\hline
Variante & $\frac{celdas}{m^2}$ & Completitud del mapa & Calidad del mapa \\ \hline\hline
\multirow{4}{\linewidth}{\centering Propuesta sin cambios}
& 1 & 0.999725±1.5e-04 & 0.999039±3.1e-04\\ \cline{2-4}
& 4 & 0.999925±6.5e-05 & 0.999703±1.7e-04\\ \cline{2-4}
& 9 & 0.999983±1.2e-05 & 0.999858±5.7e-05\\ \cline{2-4}
& 16 & 0.999998±3.6e-06 & 0.999929±3.4e-05\\ \hline\hline
\multirow{4}{\linewidth}{\centering No incremental}
& 1 & 0.999763±8.9e-05 & 0.999208±2.2e-04\\ \cline{2-4}
& 4 & 0.999957±4.2e-05 & 0.999805±1.3e-04\\ \cline{2-4}
& 9 & 0.999984±1.6e-05 & 0.999864±4.7e-05\\ \cline{2-4}
& 16 & 0.999999±1.9e-06 & 0.999941±3.3e-05\\ \hline\hline
\multirow{4}{\linewidth}{\centering Simplificación de fronteras basada en K-Means}
& 1 & 0.999802±1.3e-04 & 0.999179±2.8e-04\\ \cline{2-4}
& 4 & 0.999977±2.7e-05 & 0.999767±8.0e-05\\ \cline{2-4}
& 9 & 0.999976±2.3e-05 & 0.999831±7.4e-05\\ \cline{2-4}
& 16 & 0.999999±2.5e-06 & 0.999930±1.8e-05\\ \hline\hline
\multirow{4}{\linewidth}{\centering Fronteras sin simplificar}
& 1 & 0.999937±7.6e-05 & 0.999493±2.0e-04\\ \cline{2-4}
& 4 & 0.999995±9.8e-06 & 0.999837±4.8e-05\\ \cline{2-4}
& 9 & 0.999983±4.0e-05 & 0.999729±7.6e-05\\ \cline{2-4}
& 16 & 0.999998±4.1e-06 & 0.999921±2.2e-05\\ \hline\hline
\multirow{4}{\linewidth}{\centering Desconocidas se consideran libres}
& 1 & 0.999812±1.1e-04 & 0.999256±3.2e-04\\ \cline{2-4}
& 4 & 0.999972±1.9e-05 & 0.999786±9.7e-05\\ \cline{2-4}
& 9 & 0.999987±1.2e-05 & 0.999840±6.9e-05\\ \cline{2-4}
& 16 & 0.999998±3.5e-06 & 0.999940±2.8e-05\\ \hline
\end{tabularx}
\end{adjustbox}

\caption{Completitud y calidad de los mapas obtenidos en todas las pruebas realizadas.}
\label{tab:todo3}
\end{center}
\end{table}

\todo[inline]{consideraría, para ayudar a visualizar mejor los datos, presentarlos además en una gráfica de barras con std donde cada variante tiene un color. además, se puede graficar juntos la completitud y calidad agrupando horizontalmente las barras de completitud y calidad. \\
sobre el análisis de datos, algo que llama la atención -es una tendencia, es cómo incide positivamente tanto en calidad como en completitud, la granularidad de la grilla. desarrollaría un poco eso.}
